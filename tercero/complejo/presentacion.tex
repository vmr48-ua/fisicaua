\documentclass{beamer}
\usetheme{Madrid}
\usecolortheme{beaver}
\usepackage[utf8]{inputenc}
\usepackage[T1]{fontenc}
\usepackage[spanish]{babel}

\title{Cómo se inventaron los números imaginarios}
\author{Tú Nombre}
\date{\today}

\begin{document}

\frame{\titlepage}

\begin{frame}
\frametitle{Introducción}
En la historia de las matemáticas, los números imaginarios surgieron como una solución a un problema aparentemente insoluble. En un principio, las matemáticas se usaban para cuantificar el mundo real: medir distancias, predecir movimientos astrales y realizar transacciones comerciales. Sin embargo, los matemáticos se encontraron con ecuaciones cúbicas que parecían no tener solución. La clave para resolver este enigma fue separar las matemáticas del mundo real, desvincular el álgebra de la geometría y crear números tan fantásticos que se denominaron "imaginarios".
\end{frame}

\begin{frame}
\frametitle{Luca Pacioli y "Summa de Arithmetica"}
En 1494, Luca Pacioli publicó "Summa de Arithmetica", un compendio de matemáticas renacentistas que incluía ecuaciones cúbicas. Durante miles de años, varias civilizaciones habían intentado resolver estas ecuaciones sin éxito. Pacioli llegó a la conclusión de que no existía una solución general para las ecuaciones cúbicas, lo cual era sorprendente ya que, sin el término $x^3$, la ecuación era simplemente cuadrática y fácilmente resoluble.
\end{frame}

\begin{frame}
\frametitle{Resolución Geométrica de Ecuaciones Cuadráticas}
Los matemáticos antiguos solían resolver ecuaciones cuadráticas mediante visualización geométrica. Por ejemplo, si tenían la ecuación $x^2 + 26x = 27$, la veían como un cuadrado con un lado de longitud $x$ y un rectángulo de $26x$, sumando un área total de $27$. Al cortar el rectángulo en dos y completar el cuadrado, podían resolver la ecuación.
\end{frame}

\begin{frame}
\frametitle{Negación de Números Negativos}
Sin embargo, se resistían a considerar soluciones negativas, ya que no tenían una interpretación realista en términos de longitud, área o volumen. Esto llevó a la negación de números negativos y a la creación de múltiples versiones de ecuaciones cúbicas con coeficientes siempre positivos.
\end{frame}

\begin{frame}
\frametitle{Scipione del Ferro y Antonio Fior}
Hacia el siglo XVI, Scipione del Ferro resolvió ecuaciones cúbicas reducidas, pero mantuvo su método en secreto para asegurar su empleo. Sin embargo, su estudiante Antonio Fior se jactó de sus logros, lo que llevó al famoso desafío de Niccolo Fontana Tartaglia en 1535. Tartaglia resolvió rápidamente las ecuaciones y guardó celosamente su método.
\end{frame}

\begin{frame}
\frametitle{Gerolamo Cardano y "Ars Magna"}
Gerolamo Cardano, un matemático de Milán, persuadió a Tartaglia para que revelara su método en 1539, bajo la promesa de no publicarlo. Cardano mantuvo su promesa, pero después descubrió una solución anterior en un cuaderno de del Ferro. Finalmente, en 1545, Cardano publicó su obra "Ars Magna", que incluía la solución general para ecuaciones cúbicas. Aunque reconoció las contribuciones de otros, la historia se ha centrado en su método, llamándolo "método de Cardano".
\end{frame}

\begin{frame}
\frametitle{Conclusión}
En resumen, los números imaginarios surgieron de la necesidad de resolver ecuaciones cúbicas aparentemente insolubles. Esta historia destaca cómo los matemáticos de la época superaron desafíos y rivalidades para desarrollar nuevas técnicas y abrir el camino hacia una comprensión más profunda de las matemáticas.
\end{frame}

\end{document}
