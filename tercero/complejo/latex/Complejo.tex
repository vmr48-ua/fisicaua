\documentclass{report}
\usepackage[spanish]{babel}
\addto\captionsspanish{
  \renewcommand{\contentsname}%
    {Índice}%
}

\input{preamble}
\input{macros}
\input{letterfonts}

\title{\Huge{Apuntes de análisis de\\ variable compleja}}
\author{}
\date{\number\year}

\begin{document}

\maketitle
\clearpage
\noindent Apuntes de las clases de \textit{Análisis de variable compleja} dadas por \textit{Juan Matías Sepulcre Martínez} y transcritos a \LaTeX
\hspace{0cm} por \textit{Víctor Mira Ramírez} durante el curso 2023-2024 del grado en Física de la \textit{Universidad de Alicante}.
\pagebreak
\tableofcontents
\pagebreak

\chapter{El cuerpo de los números complejos}
  \section{Definiciones básicas}
    \definicion{Número complejo}{
      Un \textbf{número complejo} $z$ es un par ordenado de números reales $a,b$ escrito como
      $\boxed{z=\left(a,b\right)}$ en coordenadas cartesianas. Existe una notación equivalente,
      la forma binómica: $\boxed{z=a+ib}$ siendo $i=\left(0,1\right)$. \\

      El conjunto de los número complejos se denota por: $C:=\left\{(a,b):a,b\in\bbR\right\}$
    }
    \comentario{
      Siempre que $a=0$ sea un número imaginario puro, y $b=0$ sea un número real.
    }
    \definicion{Conjugado}{
      Llamamos conjugado de un número complejo al número denotado $\boxed{\ovl{z}=a-ib}$, siendo
      $z=a+ib$. Geométricamente, podemos decir que el eje real actúa de 'espejo' del número en el plano.
    }
    \comentario{
      Llamamos $\bbC$ al cuerpo de los numeros complejos. $\bbC$ es un cuerpo conmutativo, pero no totalmente ordenado. En cambio, cualquier ecuación algebraica
      tiene solución en los complejos. De todas formas, el teorema fundamental del álgebra nos asegura que tendrá n soluciones en los complejos
    }
    \comentario{
      Cuando los coeficientes de una ecuación algebraica son reales, las soluciones complejas vienen por pares.
    }
    \teorema{Operaciones elementales}{
      \renewcommand{\arraystretch}{1.6}
      \begin{tabular}{lll}
        \textbf{SUMA}& $\left(a+bi\right)+\left(c+di\right)=\left(a+c\right)+\left(b+d\right)i$\\
        \textbf{RESTA}& $\left(a+bi\right)-\left(c+di\right)=\left(a-c\right)+\left(b-d\right)i$\\
        \textbf{PRODUCTO}& $\left(a+bi\right)\cdot\left(c+di\right)=\left(ac-bd\right)+\left(ad+bc\right)i$ &(teniendo en cuenta que $i^2=-1$)\\
        \rule{0pt}{2.3em}\textbf{DIVISIÓN}& $\cfrac{a+bi}{c+di}\ =\ \cfrac{a+bi}{c+di}\cdot\cfrac{c-di}{c-di}\ =\ \cfrac{ac+bd}{c^2+d^2}+\left(\cfrac{bc-ad}{c^2+d^2}\right)i$ &(multiplicando por el conjugado)\\
      \end{tabular}
    }
    \clearpage
    \comentario{
      El elemento unidad es $1+0i$ y el elemento inverso es $\frac{a}{a^2+b^2}-\frac{b}{a^2+b^2}i$. Para que un número complejo tenga elemento inverso, debe ser distinto de cero.
      El producto de un número complejo por su elemento inverso es la unidad.
    }
    \definicion{Componentes de los complejos}{
      Llamamos \textbf{módulo} del número complejo $z=a+bi$ a la cantidad $\boxed{\sqrt{a^2+b^2}}$ denotada $\abs{z}$\\

      \vspace{0.1cm} Llamamos \textbf{argumento} del número complejo $z=a+bi$ al ángulo que forma el semieje positivo de abcisas
      con la recta que contiene el vector $\left(a,b\right)$. Se denota $\text{Arg }z=\alpha$ y se expresa en radianes.\\
      $\boxed{\alpha=\arctan{\left(\frac{b}{a}\right)}}$ si $a\neq 0$
    }
    \definicion{Módulo}{
      Llamamos \textbf{módulo} de un número complejo $z=a+bi$, y lo denotamos $\abs{z}$, a la cantidad
      $$\abs{z}=\sqrt{a^2+b^2}$$
    }
    \definicion{Argumento}{
      Llamamos \textbf{argumento} de un número complejo $z=a+bi$ al ángula que forma el semieje positivo de abcisas con la recta que contiene al vector.
      El argumento de $z$ se representa por Arg($z$)$=\alpha$, y se expresa normalmente en radianes.

      $$\alpha=\arctan{\frac{b}{a}}, \text{si} a\neq0$$
      $$\alpha=\frac{\pi}{2}, \text{si} a=0,b>0$$
      $$\alpha=\frac{3\pi}{2}, \text{si} a=0,b<0$$

      Si el ángulo se encuentra en el intervalo $[-\pi,\pi)$ lo llamaremos argumento principal.
    }
    \comentario{lol
      %Arg z (z/=0) = $\pi/2$ si $a=0$ y $b>0$
      %Arg z (z/=0) = $-\pi/2$ si $a=0$ y $b<0$
      %Arg z (z/=0) = $0$ si $a>0$ y $b=0$
      %Arg z (z/=0) = $-\pi$ si $a<0$ y $b=0$
      %Arg z (z/=0) = $\arctan(b/a)$ si $a>0$ y $b>0$
      %Arg z (z/=0) = $-arctan(-b/a)$ si $a>0$ y $b<0$
      %Arg z (z/=0) = $-arctan(b/-a)+pi$ si $a<0$ y $b>0$
      %Arg z (z/=0) = $-arctan(b/a)-pi$ si $a<0$ y $b<0$
      }
    \comentario{forma exponencial: el desarrollo en serie de la exponencial es: $e^x=\sum_{n=0}\frac{x^n}{n!}=1+x+\frac{x^2}{2}+\frac{x^3}{3}+...$
    si introducimos un número complejo en la exponencial: $e^{iy}=1+(iy)+\frac{(iy)^2}{2}+\frac{(iy)^3}{3!}+...$

    Si analizamos el valor de $i^n$ en función de n, %i= 1 si n=4k, k en Z
                                                        %i si n=4k+1
                                                        %-1 si n=4k+2
                                                        %-i si n=4k+3
    entonces vemos como la exponencial compleja queda ahora como:
    $e^{iy}=1+iy-\frac{y^2}{2}-\frac{iy^3}{3!}+\frac{y^4}{4!}+...=
    \left(1-\frac{y^2}{2!}+\frac{y^4}{4!}+...\right)+i\left(y-\frac{y^3}{3!}+\frac{y^5}{5!}\right)=
    \cos(y)+i\sin(y)$

    $e^z=e^xe^{iy}=e^x\left(\cos(y)+i\sin(y)\right)$ con $z=x+iy$
    }
    \clearpage

    %
    %
    %
    %
  \section{Analiticidad}
    \definicion{Función armónica conjugada}{
      Sea $u\colon\mcD\subset\bbR^2\rightarrow\bbR$ una función armónica en un abierto
      de $\mcD\subset\bbR^2$ diremos que $v\colon\mcD\subset\bbR^2\rightarrow\bbR$ es
      una \textbf{función armónica conjugada} de $u$ en $\mcD$ si $v$ es armónica en
      $\mcD$ y satisfacen las condiciones de \textit{Cauchy-Riemann}, (o equivalentemente
      la función $f(x+iy)=u(x,y)+iv(x,y)$ es holomorfa en $\left\{x+iy\in\bbC\colon (x,y)
      \in\mcD\right\}$)
    }
    \comentario{
      Una función armónica es aquella que satisface la ecuación de Laplace.
    }
    \teorema{}{
      Sea $u(x,y)\colon\mcD\rightarrow\bbR$ es una función armónica de $\mcD$ y
      consideramos $v$ una región rectangular contenida en $\mcD$. Entonces existe una
      conjugada armónica de $u(x,y)$ en $v$.
    }
  \section{Algunas funciones elementales}
    \subsection{Función exponencial}
      \definicion{}{
      $$f(z)=e^z=e^xe^{iy}=e^x\left(\cos y+i\sin y\right)$$
      }
      \teorema{}{
        \begin{enumerate}
          \item $e^z\neq 0\hspace{0.6cm}\forall z\in\bbC$
          \item $\abs{e^z}=e^{Re(z)}\hspace{0.6cm}z\in\bbC$
          \item $arg\left(e^z\right)=\left\{Im(z)+2\pi k,\hspace{0.2cm}
                k\in\bbZ\right\}\hspace{0.8cm}\forall z\in\bbC$
          \item $\ovl{\left(e^z\right)}=e^{\ovl{z}}\hspace{0.8cm}z\in\bbC$
          \item $e^x=1\Leftrightarrow x=0 \hspace{1cm} x\in\bbR$\\
                $e^z=1\Leftrightarrow z=2\pi ki \hspace{0.8cm} z\in\bbC$
          \item $\lim_{x\rightarrow\infty} e^x=\infty\hspace{0.8cm} x\in\bbR$\\
                $\nexists \lim_{\abs{z}\rightarrow\infty} e^z=\infty\hspace{0.8cm} x\in\bbR$
          \item $e^x$ es entera (derivable en todo punto de $\bbC$) $\left(e^z
                \right)^{\prime}=e^z$
          \item $e^{z+\omega}=e^z\cdot e^\omega, \hspace{0.8cm}z,w\in\bbC$\\
                $\left(e^z\right)^n=e^{nz}, \hspace{0.58cm}n\in\bbN, z\in\bbC$
          % (e^iy = cosy + iseny           (e^iy)^n=e^) ... pedir formula de moivre

        \end{enumerate}
      }
      \ejemplo{$e^{iz}-e^{-iz}=4i$}{
        $e^{iz}-e^{-iz}=4i\Longleftrightarrow e^{iz}-e^{-iz}-4i=0\Longleftrightarrow
        e^{2iz}-4ie^{iz}-1=0$\\

        Si $\omega=e^{iz}\Longrightarrow \boxed{\omega^2-4i\omega-1=0}$\\
        \vspace{0.2cm}
        $w=\dfrac{4i\pm\sqrt{-16+4}}{2}=\dfrac{4i\pm\sqrt{-12}}{2}=2i\pm\sqrt{3}=
        2\pm\sqrt{3}i\Longrightarrow \boxed{e^{iz}=(2\pm\sqrt{3})i}$
      }
    \subsection{Función logarítmica}
      \definicion{}{
        Se introduce por la necesidad de solucionar ecuaciones como la anterior.
        $$x=e^y\hspace{0.2cm}\Longleftrightarrow\hspace{0.2cm}y=\log x, \hspace{1cm}
        x\>0, y\in\bbR$$
        Sea $z\in\bbC -{0}$, definimos el logaritmo principal de $z$, y lo denotamos
        por $\log z$, como
        $$\log z=ln\abs{z}+i\cdot Arg(z)$$
        Vemos que $e^{\log z}=e^{log\abs{z}+Arg(z)}=e^{ln\abs{z}}e^{Arg(z)}=\abs{z}
        e^{Arg(z)}=z$\\
        El conjunto de todos los logaritmos de z será:
        $$\log z = \left\{ln\abs{z}+i\left(Arg(z)+2\pi k\right), k\in\bbZ\right\}$$
      }
      \ejemplo{}{
        \begin{enumerate}
          \item Si $z=x>0\Rightarrow \log z = ln\abs{z}+i\cdot Arg(z)= lnx$\\
                $\log z=\left\{ln x + 2\pi k i, k\in\bbZ\right\}$
          \item Si $z=-x>0\Rightarrow \log z = ln x-i\cdot (-\pi)$ (argumento de $z$)\\
                $\log z=\left\{ln x + -(\pi+2\pi k), k\in\bbZ\right\}$
          \item Si $z=ix, x>0\Rightarrow \log z = ln x+i\dfrac{\pi}{2}$\\
                $\log z=\left\{ln x + i\left(\dfrac{\pi}{2}+2\pi k\right), k\in\bbZ
                \right\}$
        \end{enumerate}
      }
      \comentario{
        Retomando la ecuación del ejemplo anterior,

        $$e^{iz}=(2\pm\sqrt{3})i=
        \begin{cases}
          &(2+\sqrt{3}i) \leftrightarrow iz=\log(2+\sqrt{3})i \leftrightarrow 
            z=\left(\dfrac{\pi}{2}+2\pi k\right) - iln(2+\sqrt{3})\\
          &(2-\sqrt{3}i) \leftrightarrow iz=\log(2-\sqrt{3})i \leftrightarrow 
            z=\left(\dfrac{\pi}{2}+2\pi k\right) - iln(2-\sqrt{3})\\
        \end{cases}, k\in\bbZ$$
      }
      \teorema{Propiedades}{
        \begin{enumerate}
          \item Log $z$ es holomorfa en $\bbC - \left[-\infty,0\right]\implies$ 
                de hecho, no es continua en $(-\infty,0]$
          \item $\log_{\theta_0}=z$ es holomorfa en $\bbC-\left\{z\in\bbC, arg(z)=
                \theta_0\right\}$
          \item $e^{\log_{\theta_0}z}=z, \hspace{0.8cm}\forall z\in\bbC, arg(z)=\theta_0$
                y $\left(log_{\theta_0}\right)^{\prime}=\dfrac{1}{z}$
          \item $\log_{\theta_0}e^z=z \hspace{0.8cm}\forall z=x+iy, \theta_0 <= 
                y<\theta_0+2\pi$
                $z=x+iy, e^z=e^xe^{iy}\implies\log_{\theta_0}e^z=z$\\ cuando 
                $y\in\left[\theta_0, \theta_0+2\pi\right]$
        \end{enumerate}
        % PEDIR DEMOSTRACIÓN
      }
      \definicion{}{
        Sea $\theta_0\in\bbR$, tomamos $z\neq0$, $z=re^{i\theta}$, $r>0$, 
        $\theta_0<=\theta=\theta_0+2\pi$ y entonces $log_{\theta_0}z=ln\abs{z}+i\theta$\\
        
        Si $\theta_0=-\pi \implies \log_{\theta_0}z=Log z$\\
        Si $\theta_0=0 \implies log_{0}z=ln\abs{z}+i\theta, \hspace{0.8cm} 0<=\theta<2\pi$

      }



\end{document}
