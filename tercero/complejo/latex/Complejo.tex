\documentclass{report}
\usepackage[spanish]{babel}
\addto\captionsspanish{
  \renewcommand{\contentsname}%
    {Índice}%
}

%%%%%%%%%%%%%%%%%%%%%%%%%%%%%%%%%
% PACKAGE IMPORTS
%%%%%%%%%%%%%%%%%%%%%%%%%%%%%%%%%


\usepackage[tmargin=2cm,rmargin=1in,lmargin=1in,margin=0.85in,bmargin=2cm,footskip=.2in]{geometry}
\usepackage{amsmath,amsfonts,amsthm,amssymb,mathtools}
\usepackage{wrapfig}
\usepackage{subcaption}
\usepackage{afterpage}
\usepackage[varbb]{newpxmath}
\usepackage{xfrac}
\usepackage[T1]{fontenc}
\usepackage{tabularx}
\usepackage{floatrow}
\usepackage{enumerate}
\newfloatcommand{capbtabbox}{table}[][\FBwidth]
\usepackage{amssymb}
%command for alg-closure that automatically adapts its 'bar' to the arg based on the args length (including '\')
\newcommand{\ols}[1]{\mskip.5\thinmuskip\overline{\mskip-.5\thinmuskip {#1} \mskip-.5\thinmuskip}\mskip.5\thinmuskip} % overline short
\newcommand{\olsi}[1]{\,\overline{\!{#1}}} % overline short italic
\makeatletter
\newcommand\ovl[1]{
  \tctestifnum{\count@stringtoks{#1}>1} %checks if number of chars in arg > 1 (including '\')
  {\ols{#1}} %if arg is longer than just one char, e.g. \mathbb{Q}, \mathbb{F},...
  {\olsi{#1}} %if arg is just one char, e.g. K, L,...
}
% FROM TOKCYCLE:
\long\def\count@stringtoks#1{\tc@earg\count@toks{\string#1}}
\long\def\count@toks#1{\the\numexpr-1\count@@toks#1.\tc@endcnt}
\long\def\count@@toks#1#2\tc@endcnt{+1\tc@ifempty{#2}{\relax}{\count@@toks#2\tc@endcnt}}
\def\tc@ifempty#1{\tc@testxifx{\expandafter\relax\detokenize{#1}\relax}}
\long\def\tc@earg#1#2{\expandafter#1\expandafter{#2}}
\long\def\tctestifnum#1{\tctestifcon{\ifnum#1\relax}}
\long\def\tctestifcon#1{#1\expandafter\tc@exfirst\else\expandafter\tc@exsecond\fi}
\long\def\tc@testxifx{\tc@earg\tctestifx}
\long\def\tctestifx#1{\tctestifcon{\ifx#1}}
\long\def\tc@exfirst#1#2{#1}
\long\def\tc@exsecond#1#2{#2}
\makeatother
\usepackage[makeroom]{cancel}
\usepackage{mathtools}
\usepackage{bookmark}
\usepackage{enumitem}
\usepackage{hyperref,theoremref}
\hypersetup{
	pdftitle={Apuntes de Análisis de Varias Variables 2},
	colorlinks=true, linkcolor=doc!90,
	bookmarksnumbered=true,
	bookmarksopen=true
}
\usepackage[most,many,breakable]{tcolorbox}
\usepackage{xcolor}
\usepackage{varwidth}
\usepackage{varwidth}
\usepackage{etoolbox}
%\usepackage{authblk}
\usepackage{nameref}
\usepackage{multicol,array}
\usepackage{tikz-cd}
\usepackage[ruled,vlined,linesnumbered]{algorithm2e}
\usepackage{comment} % enables the use of multi-line comments (\ifx \fi) 
\usepackage{import}
\usepackage{xifthen}
\usepackage{pdfpages}
\usepackage{transparent}
\usepackage{titlesec}
\titleformat{\section}{\normalfont\fontsize{17.28pt}{12pt}\selectfont\bfseries}{\thesection}{1em}{}
\usepackage{empheq} % boxed align


\newcommand\mycommfont[1]{\footnotesize\ttfamily\textcolor{blue}{#1}}
\SetCommentSty{mycommfont}
\newcommand{\incfig}[1]{%
    \def\svgwidth{\columnwidth}
    \import{./figures/}{#1.pdf_tex}
}

\newcommand{\notimplies}{\;\not\!\!\!\implies}

\usepackage{tikzsymbols}
\renewcommand\qedsymbol{$\Laughey$}

\newcommand\phantomarrow[2]{%
  \setbox0=\hbox{$\displaystyle #1\to$}%
  \hbox to \wd0{%
    $#2\mapstochar
     \cleaders\hbox{$\mkern-1mu\relbar\mkern-3mu$}\hfill
     \mkern-7mu\rightarrow$}%
  \,}

%\usepackage{import}
%\usepackage{xifthen}
%\usepackage{pdfpages}
%\usepackage{transparent}


%%%%%%%%%%%%%%%%%%%%%%%%%%%%%%
% SELF MADE COLORS
%%%%%%%%%%%%%%%%%%%%%%%%%%%%%%



\definecolor{myg}{RGB}{56, 140, 70}
\definecolor{myb}{RGB}{45, 111, 177}
\definecolor{myr}{RGB}{199, 68, 64}
\definecolor{mytheorembg}{HTML}{f5e4e1}
\definecolor{mytheoremfr}{HTML}{7B0000}
\definecolor{mylenmabg}{HTML}{FFFAF8}
\definecolor{mylenmafr}{HTML}{983b0f}
\definecolor{mypropbg}{HTML}{f2fbfc}
\definecolor{mypropfr}{HTML}{191971}
\definecolor{myexamplebg}{HTML}{F2FBF8}
\definecolor{myexamplefr}{HTML}{88D6D1}
\definecolor{myexampleti}{HTML}{2A7F7F}
\definecolor{mydefinitbg}{HTML}{E5E5FF}
\definecolor{mydefinitfr}{HTML}{3F3FA3}
\definecolor{notesgreen}{RGB}{0,162,0}
\definecolor{myp}{RGB}{197, 92, 212}
\definecolor{mygr}{HTML}{2C3338}
\definecolor{myred}{RGB}{127,0,0}
\definecolor{myyellow}{RGB}{169,121,69}
\definecolor{myexercisebg}{HTML}{F2FBF8}
\definecolor{myexercisefg}{HTML}{2A7F7F}


%%%%%%%%%%%%%%%%%%%%%%%%%%%%
% TCOLORBOX SETUPS
%%%%%%%%%%%%%%%%%%%%%%%%%%%%

\setlength{\parindent}{1cm}
%================================
% THEOREM BOX
%================================

\tcbuselibrary{theorems,skins,hooks}
\newtcbtheorem[number within=section]{Theorem}{Teorema}
{%
	enhanced,
	breakable,
	colback = mytheorembg,
	frame hidden,
	boxrule = 0sp,
	borderline west = {2pt}{0pt}{mytheoremfr},
	sharp corners,
	detach title,
	before upper = \tcbtitle\par\smallskip,
	coltitle = mytheoremfr,
	fonttitle = \bfseries\sffamily,
	description font = \mdseries,
	separator sign none,
	segmentation style={solid, mytheoremfr},
}
{th}

\tcbuselibrary{theorems,skins,hooks}
\newtcbtheorem[number within=chapter]{theorem}{Teorema}
{%
	enhanced,
	breakable,
	colback = mytheorembg,
	frame hidden,
	boxrule = 0sp,
	borderline west = {2pt}{0pt}{mytheoremfr},
	sharp corners,
	detach title,
	before upper = \tcbtitle\par\smallskip,
	coltitle = mytheoremfr,
	fonttitle = \bfseries\sffamily,
	description font = \mdseries,
	separator sign none,
	segmentation style={solid, mytheoremfr},
}
{th}


\tcbuselibrary{theorems,skins,hooks}
\newtcolorbox{Theoremcon}
{%
	enhanced
	,breakable
	,colback = mytheorembg
	,frame hidden
	,boxrule = 0sp
	,borderline west = {2pt}{0pt}{mytheoremfr}
	,sharp corners
	,description font = \mdseries
	,separator sign none
}

%================================
% Corollary
%================================
\tcbuselibrary{theorems,skins,hooks}
\newtcbtheorem[number within=section]{Corollary}{Corolario}
{%
	enhanced
	,breakable
	,colback = myp!10
	,frame hidden
	,boxrule = 0sp
	,borderline west = {2pt}{0pt}{myp!85!black}
	,sharp corners
	,detach title
	,before upper = \tcbtitle\par\smallskip
	,coltitle = myp!85!black
	,fonttitle = \bfseries\sffamily
	,description font = \mdseries
	,separator sign none
	,segmentation style={solid, myp!85!black}
}
{th}
\tcbuselibrary{theorems,skins,hooks}
\newtcbtheorem[number within=chapter]{corollary}{Corolario}
{%
	enhanced
	,breakable
	,colback = myp!10
	,frame hidden
	,boxrule = 0sp
	,borderline west = {2pt}{0pt}{myp!85!black}
	,sharp corners
	,detach title
	,before upper = \tcbtitle\par\smallskip
	,coltitle = myp!85!black
	,fonttitle = \bfseries\sffamily
	,description font = \mdseries
	,separator sign none
	,segmentation style={solid, myp!85!black}
}
{th}


%================================
% LENMA
%================================

\tcbuselibrary{theorems,skins,hooks}
\newtcbtheorem[number within=section]{Lenma}{Lenma}
{%
	enhanced,
	breakable,
	colback = mylenmabg,
	frame hidden,
	boxrule = 0sp,
	borderline west = {2pt}{0pt}{mylenmafr},
	sharp corners,
	detach title,
	before upper = \tcbtitle\par\smallskip,
	coltitle = mylenmafr,
	fonttitle = \bfseries\sffamily,
	description font = \mdseries,
	separator sign none,
	segmentation style={solid, mylenmafr},
}
{th}

\tcbuselibrary{theorems,skins,hooks}
\newtcbtheorem[number within=chapter]{lenma}{Lenma}
{%
	enhanced,
	breakable,
	colback = mylenmabg,
	frame hidden,
	boxrule = 0sp,
	borderline west = {2pt}{0pt}{mylenmafr},
	sharp corners,
	detach title,
	before upper = \tcbtitle\par\smallskip,
	coltitle = mylenmafr,
	fonttitle = \bfseries\sffamily,
	description font = \mdseries,
	separator sign none,
	segmentation style={solid, mylenmafr},
}
{th}


%================================
% PROPOSITION
%================================

\tcbuselibrary{theorems,skins,hooks}
\newtcbtheorem[number within=section]{Prop}{Proposition}
{%
	enhanced,
	breakable,
	colback = mypropbg,
	frame hidden,
	boxrule = 0sp,
	borderline west = {2pt}{0pt}{mypropfr},
	sharp corners,
	detach title,
	before upper = \tcbtitle\par\smallskip,
	coltitle = mypropfr,
	fonttitle = \bfseries\sffamily,
	description font = \mdseries,
	separator sign none,
	segmentation style={solid, mypropfr},
}
{th}

\tcbuselibrary{theorems,skins,hooks}
\newtcbtheorem[number within=chapter]{prop}{Proposition}
{%
	enhanced,
	breakable,
	colback = mypropbg,
	frame hidden,
	boxrule = 0sp,
	borderline west = {2pt}{0pt}{mypropfr},
	sharp corners,
	detach title,
	before upper = \tcbtitle\par\smallskip,
	coltitle = mypropfr,
	fonttitle = \bfseries\sffamily,
	description font = \mdseries,
	separator sign none,
	segmentation style={solid, mypropfr},
}
{th}


%================================
% CLAIM
%================================

\tcbuselibrary{theorems,skins,hooks}
\newtcbtheorem[number within=section]{claim}{Claim}
{%
	enhanced
	,breakable
	,colback = myg!10
	,frame hidden
	,boxrule = 0sp
	,borderline west = {2pt}{0pt}{myg}
	,sharp corners
	,detach title
	,before upper = \tcbtitle\par\smallskip
	,coltitle = myg!85!black
	,fonttitle = \bfseries\sffamily
	,description font = \mdseries
	,separator sign none
	,segmentation style={solid, myg!85!black}
}
{th}



%================================
% Exercise
%================================

\tcbuselibrary{theorems,skins,hooks}
\newtcbtheorem[number within=section]{Exercise}{Ejercicio}
{%
	enhanced,
	breakable,
	colback = myexercisebg,
	frame hidden,
	boxrule = 0sp,
	borderline west = {2pt}{0pt}{myexercisefg},
	sharp corners,
	detach title,
	before upper = \tcbtitle\par\smallskip,
	coltitle = myexercisefg,
	fonttitle = \bfseries\sffamily,
	description font = \mdseries,
	separator sign none,
	segmentation style={solid, myexercisefg},
}
{th}

\tcbuselibrary{theorems,skins,hooks}
\newtcbtheorem[number within=chapter]{exercise}{Ejercicio}
{%
	enhanced,
	breakable,
	colback = myexercisebg,
	frame hidden,
	boxrule = 0sp,
	borderline west = {2pt}{0pt}{myexercisefg},
	sharp corners,
	detach title,
	before upper = \tcbtitle\par\smallskip,
	coltitle = myexercisefg,
	fonttitle = \bfseries\sffamily,
	description font = \mdseries,
	separator sign none,
	segmentation style={solid, myexercisefg},
}
{th}

%================================
% EXAMPLE BOX
%================================

\newtcbtheorem[number within=section]{Example}{Ejemplo}
{%
	colback = myexamplebg
	,breakable
	,colframe = myexamplefr
	,coltitle = myexampleti
	,boxrule = 1pt
	,sharp corners
	,detach title
	,before upper=\tcbtitle\par\smallskip
	,fonttitle = \bfseries
	,description font = \mdseries
	,separator sign none
	,description delimiters parenthesis
}
{ex}

\newtcbtheorem[number within=chapter]{example}{Ejemplo}
{%
	colback = myexamplebg
	,breakable
	,colframe = myexamplefr
	,coltitle = myexampleti
	,boxrule = 1pt
	,sharp corners
	,detach title
	,before upper=\tcbtitle\par\smallskip
	,fonttitle = \bfseries
	,description font = \mdseries
	,separator sign none
	,description delimiters parenthesis
}
{ex}

%================================
% DEFINITION BOX
%================================

\newtcbtheorem[number within=section]{Definition}{Definición}{enhanced,
	before skip=2mm,after skip=2mm, colback=blue!5,colframe=blue!80!black,boxrule=0.5mm,
	attach boxed title to top left={xshift=1cm,yshift*=1mm-\tcboxedtitleheight}, varwidth boxed title*=-3cm,
	boxed title style={frame code={
					\path[fill=tcbcolback]
					([yshift=-1mm,xshift=-1mm]frame.north west)
					arc[start angle=0,end angle=180,radius=1mm]
					([yshift=-1mm,xshift=1mm]frame.north east)
					arc[start angle=180,end angle=0,radius=1mm];
					\path[left color=tcbcolback!60!black,right color=tcbcolback!60!black,
						middle color=tcbcolback!80!black]
					([xshift=-2mm]frame.north west) -- ([xshift=2mm]frame.north east)
					[rounded corners=1mm]-- ([xshift=1mm,yshift=-1mm]frame.north east)
					-- (frame.south east) -- (frame.south west)
					-- ([xshift=-1mm,yshift=-1mm]frame.north west)
					[sharp corners]-- cycle;
				},interior engine=empty,
		},
	fonttitle=\bfseries,
	title={#2},#1}{def}
\newtcbtheorem[number within=chapter]{definition}{Definición}{enhanced,
	before skip=2mm,after skip=2mm, colback=red!5,colframe=red!80!black,boxrule=0.5mm,
	attach boxed title to top left={xshift=1cm,yshift*=1mm-\tcboxedtitleheight}, varwidth boxed title*=-3cm,
	boxed title style={frame code={
					\path[fill=tcbcolback]
					([yshift=-1mm,xshift=-1mm]frame.north west)
					arc[start angle=0,end angle=180,radius=1mm]
					([yshift=-1mm,xshift=1mm]frame.north east)
					arc[start angle=180,end angle=0,radius=1mm];
					\path[left color=tcbcolback!60!black,right color=tcbcolback!60!black,
						middle color=tcbcolback!80!black]
					([xshift=-2mm]frame.north west) -- ([xshift=2mm]frame.north east)
					[rounded corners=1mm]-- ([xshift=1mm,yshift=-1mm]frame.north east)
					-- (frame.south east) -- (frame.south west)
					-- ([xshift=-1mm,yshift=-1mm]frame.north west)
					[sharp corners]-- cycle;
				},interior engine=empty,
		},
	fonttitle=\bfseries,
	title={#2},#1}{def}



%================================
% Solution BOX
%================================

\makeatletter
\newtcbtheorem{question}{Cuestión}{enhanced,
	breakable,
	colback=white,
	colframe=myb!80!black,
	attach boxed title to top left={yshift*=-\tcboxedtitleheight},
	fonttitle=\bfseries,
	title={#2},
	boxed title size=title,
	boxed title style={%
			sharp corners,
			rounded corners=northwest,
			colback=tcbcolframe,
			boxrule=0pt,
		},
	underlay boxed title={%
			\path[fill=tcbcolframe] (title.south west)--(title.south east)
			to[out=0, in=180] ([xshift=5mm]title.east)--
			(title.center-|frame.east)
			[rounded corners=\kvtcb@arc] |-
			(frame.north) -| cycle;
		},
	#1
}{def}
\makeatother

%================================
% SOLUTION BOX
%================================

\makeatletter
\newtcolorbox{solution}{enhanced,
	breakable,
	colback=white,
	colframe=myg!80!black,
	attach boxed title to top left={yshift*=-\tcboxedtitleheight},
	title=Solution,
	boxed title size=title,
	boxed title style={%
			sharp corners,
			rounded corners=northwest,
			colback=tcbcolframe,
			boxrule=0pt,
		},
	underlay boxed title={%
			\path[fill=tcbcolframe] (title.south west)--(title.south east)
			to[out=0, in=180] ([xshift=5mm]title.east)--
			(title.center-|frame.east)
			[rounded corners=\kvtcb@arc] |-
			(frame.north) -| cycle;
		},
}
\makeatother

%================================
% Question BOX
%================================

\makeatletter
\newtcbtheorem{qstion}{Cuestión}{enhanced,
	breakable,
	colback=white,
	colframe=mygr,
	attach boxed title to top left={yshift*=-\tcboxedtitleheight},
	fonttitle=\bfseries,
	title={#2},
	boxed title size=title,
	boxed title style={%
			sharp corners,
			rounded corners=northwest,
			colback=tcbcolframe,
			boxrule=0pt,
		},
	underlay boxed title={%
			\path[fill=tcbcolframe] (title.south west)--(title.south east)
			to[out=0, in=180] ([xshift=5mm]title.east)--
			(title.center-|frame.east)
			[rounded corners=\kvtcb@arc] |-
			(frame.north) -| cycle;
		},
	#1
}{def}
\makeatother

\newtcbtheorem[number within=chapter]{wconc}{Wrong Concept}{
	breakable,
	enhanced,
	colback=white,
	colframe=myr,
	arc=0pt,
	outer arc=0pt,
	fonttitle=\bfseries\sffamily\large,
	colbacktitle=myr,
	attach boxed title to top left={},
	boxed title style={
			enhanced,
			skin=enhancedfirst jigsaw,
			arc=3pt,
			bottom=0pt,
			interior style={fill=myr}
		},
	#1
}{def}



%================================
% NOTE BOX
%================================

\usetikzlibrary{arrows,calc,shadows.blur}
\tcbuselibrary{skins}
\newtcolorbox{note}[1][]{%
	enhanced jigsaw,
	colback=gray!10!white,%
	colframe=gray!80!black,
	size=small,
	boxrule=1pt,
	title=\textbf{Comentario:},
	halign title=flush center,
	coltitle=black,
	breakable,
	drop shadow=black!50!white,
	attach boxed title to top left={xshift=1cm,yshift=-\tcboxedtitleheight/2,yshifttext=-\tcboxedtitleheight/2},
	minipage boxed title=2.5cm,
	boxed title style={%
			colback=white,
			size=fbox,
			boxrule=1pt,
			boxsep=2pt,
			underlay={%
					\coordinate (dotA) at ($(interior.west) + (-0.5pt,0)$);
					\coordinate (dotB) at ($(interior.east) + (0.5pt,0)$);
					\begin{scope}
						\clip (interior.north west) rectangle ([xshift=3ex]interior.east);
						\filldraw [white, blur shadow={shadow opacity=60, shadow yshift=-.75ex}, rounded corners=2pt] (interior.north west) rectangle (interior.south east);
					\end{scope}
					\begin{scope}[gray!80!black]
						\fill (dotA) circle (2pt);
						\fill (dotB) circle (2pt);
					\end{scope}
				},
		},
	#1,
}

%%%%%%%%%%%%%%%%%%%%%%%%%%%%%%
% SELF MADE COMMANDS
%%%%%%%%%%%%%%%%%%%%%%%%%%%%%%

\newcommand{\ejercicio}[2]{\begin{Exercise}{#1}{}#2\end{Exercise}}
\newcommand{\corolario}[2]{\begin{Corollary}{#1}{}#2\end{Corollary}}
\newcommand{\teorema}[2]{\begin{Theorem}{#1}{}#2\end{Theorem}}
\newcommand{\thmcon}[1]{\begin{Theoremcon}{#1}\end{Theoremcon}}
\newcommand{\ejemplo}[2]{\begin{Example}{#1}{}#2\end{Example}}
\newcommand{\definicion}[2]{\begin{Definition}[colbacktitle=blue!75!black]{#1}{}#2\end{Definition}}
\newcommand{\cuestion}[2]{\begin{question}{#1}{}#2\end{question}}
\newcommand{\comentario}[1]{\begin{note}#1\end{note}}

\newcommand*\circled[1]{\tikz[baseline=(char.base)]{
		\node[shape=circle,draw,inner sep=1pt] (char) {#1};}}
\newcommand\getcurrentref[1]{%
	\ifnumequal{\value{#1}}{0}
	{??}
	{\the\value{#1}}%
}
\newcommand{\getCurrentSectionNumber}{\getcurrentref{section}}
\newenvironment{myproof}[1][\proofname]{%
	\proof[\bfseries #1: ]%
}{\endproof}

\newcommand{\mclm}[2]{\begin{myclaim}[#1]#2\end{myclaim}}
\newenvironment{myclaim}[1][\claimname]{\proof[\bfseries #1: ]}{}

\newcounter{mylabelcounter}

\makeatletter
\newcommand{\setword}[2]{%
	\phantomsection
	#1\def\@currentlabel{\unexpanded{#1}}\label{#2}%
}
\makeatother




\tikzset{
	symbol/.style={
			draw=none,
			every to/.append style={
					edge node={node [sloped, allow upside down, auto=false]{$#1$}}}
		}
}


% deliminators
\DeclarePairedDelimiter{\abs}{\lvert}{\rvert}
\DeclarePairedDelimiter{\norm}{\lVert}{\rVert}

\DeclarePairedDelimiter{\ceil}{\lceil}{\rceil}
\DeclarePairedDelimiter{\floor}{\lfloor}{\rfloor}
\DeclarePairedDelimiter{\round}{\lfloor}{\rceil}

\newsavebox\diffdbox
\newcommand{\slantedromand}{{\mathpalette\makesl{d}}}
\newcommand{\makesl}[2]{%
\begingroup
\sbox{\diffdbox}{$\mathsurround=0pt#1\mathrm{#2}$}%
\pdfsave
\pdfsetmatrix{1 0 0.2 1}%
\rlap{\usebox{\diffdbox}}%
\pdfrestore
\hskip\wd\diffdbox
\endgroup
}
\newcommand{\dd}[1][]{\ensuremath{\mathop{}\!\ifstrempty{#1}{%
\slantedromand\@ifnextchar^{\hspace{0.2ex}}{\hspace{0.1ex}}}%
{\slantedromand\hspace{0.2ex}^{#1}}}}
\ProvideDocumentCommand\dv{o m g}{%
  \ensuremath{%
    \IfValueTF{#3}{%
      \IfNoValueTF{#1}{%
        \frac{\dd #2}{\dd #3}%
      }{%
        \frac{\dd^{#1} #2}{\dd #3^{#1}}%
      }%
    }{%
      \IfNoValueTF{#1}{%
        \frac{\dd}{\dd #2}%
      }{%
        \frac{\dd^{#1}}{\dd #2^{#1}}%
      }%
    }%
  }%
}
\providecommand*{\pdv}[3][]{\frac{\partial^{#1}#2}{\partial#3^{#1}}}
%  - others
\DeclareMathOperator{\Lap}{\mathcal{L}}
\DeclareMathOperator{\Var}{Var} % varience
\DeclareMathOperator{\Cov}{Cov} % covarience
\DeclareMathOperator{\E}{E} % expected

% Since the amsthm package isn't loaded

% I prefer the slanted \leq
\let\oldleq\leq % save them in case they're every wanted
\let\oldgeq\geq
\renewcommand{\leq}{\leqslant}
\renewcommand{\geq}{\geqslant}

% % redefine matrix env to allow for alignment, use r as default
% \renewcommand*\env@matrix[1][r]{\hskip -\arraycolsep
%     \let\@ifnextchar\new@ifnextchar
%     \array{*\c@MaxMatrixCols #1}}


%\usepackage{framed}
%\usepackage{titletoc}
%\usepackage{etoolbox}
%\usepackage{lmodern}


%\patchcmd{\tableofcontents}{\contentsname}{\sffamily\contentsname}{}{}

%\renewenvironment{leftbar}
%{\def\FrameCommand{\hspace{6em}%
%		{\color{myyellow}\vrule width 2pt depth 6pt}\hspace{1em}}%
%	\MakeFramed{\parshape 1 0cm \dimexpr\textwidth-6em\relax\FrameRestore}\vskip2pt%
%}
%{\endMakeFramed}

%\titlecontents{chapter}
%[0em]{\vspace*{2\baselineskip}}
%{\parbox{4.5em}{%
%		\hfill\Huge\sffamily\bfseries\color{myred}\thecontentspage}%
%	\vspace*{-2.3\baselineskip}\leftbar\textsc{\small\chaptername~\thecontentslabel}\\\sffamily}
%{}{\endleftbar}
%\titlecontents{section}
%[8.4em]
%{\sffamily\contentslabel{3em}}{}{}
%{\hspace{0.5em}\nobreak\itshape\color{myred}\contentspage}
%\titlecontents{subsection}
%[8.4em]
%{\sffamily\contentslabel{3em}}{}{}  
%{\hspace{0.5em}\nobreak\itshape\color{myred}\contentspage}



%%%%%%%%%%%%%%%%%%%%%%%%%%%%%%%%%%%%%%%%%%%
% TABLE OF CONTENTS
%%%%%%%%%%%%%%%%%%%%%%%%%%%%%%%%%%%%%%%%%%%

\usepackage{tikz}
\definecolor{doc}{RGB}{0,60,110}
\usepackage{titletoc}
\contentsmargin{0cm}
\titlecontents{chapter}[3.7pc]
{\addvspace{30pt}%
	\begin{tikzpicture}[remember picture, overlay]%
		\draw[fill=doc!60,draw=doc!60] (-7,-.1) rectangle (-0.9,.5);%
		\pgftext[left,x=-3.6cm,y=0.2cm]{\color{white}\Large\sc\bfseries Capítulo\ \thecontentslabel};%
	\end{tikzpicture}\color{doc!60}\large\sc\bfseries}%
{}
{}
{\;\titlerule\;\large\sc\bfseries Página \thecontentspage
	\begin{tikzpicture}[remember picture, overlay]
		\draw[fill=doc!60,draw=doc!60] (2pt,0) rectangle (4,0.1pt);
	\end{tikzpicture}}%
\titlecontents{section}[3.7pc]
{\addvspace{2pt}}
{\contentslabel[\thecontentslabel]{2pc}}
{}
{\hfill\small \thecontentspage}
[]
\titlecontents*{subsection}[3.7pc]
{\addvspace{-1pt}\small}
{}
{}
{\ --- \small\thecontentspage}
[ \textbullet\ ][]

\makeatletter
\renewcommand{\tableofcontents}{%
	\chapter*{%
	  \vspace*{-20\p@}%
	  \begin{tikzpicture}[remember picture, overlay]%
		  \pgftext[right,x=15cm,y=0.2cm]{\color{doc!60}\Huge\sc\bfseries \contentsname};%
		  \draw[fill=doc!60,draw=doc!60] (13,-.75) rectangle (20,1);%
		  \clip (13,-.75) rectangle (20,1);
		  \pgftext[right,x=15cm,y=0.2cm]{\color{white}\Huge\sc\bfseries \contentsname};%
	  \end{tikzpicture}}%
	\@starttoc{toc}}
\makeatother


%From M275 "Topology" at SJSU
\newcommand{\id}{\mathrm{id}}
\newcommand{\taking}[1]{\xrightarrow{#1}}
\newcommand{\inv}{^{-1}}

%From M170 "Introduction to Graph Theory" at SJSU
\DeclareMathOperator{\diam}{diam}
\DeclareMathOperator{\ord}{ord}
\newcommand{\defeq}{\overset{\mathrm{def}}{=}}

%From the USAMO .tex files
\newcommand{\ts}{\textsuperscript}
\newcommand{\dg}{^\circ}
\newcommand{\ii}{\item}

% % From Math 55 and Math 145 at Harvard
% \newenvironment{subproof}[1][Proof]{%
% \begin{proof}[#1] \renewcommand{\qedsymbol}{$\blacksquare$}}%
% {\end{proof}}

\newcommand{\liff}{\leftrightarrow}
\newcommand{\lthen}{\rightarrow}
\newcommand{\opname}{\operatorname}
\newcommand{\surjto}{\twoheadrightarrow}
\newcommand{\injto}{\hookrightarrow}
\newcommand{\On}{\mathrm{On}} % ordinals
\DeclareMathOperator{\img}{im} % Image
\DeclareMathOperator{\Img}{Im} % Image
\DeclareMathOperator{\coker}{coker} % Cokernel
\DeclareMathOperator{\Coker}{Coker} % Cokernel
\DeclareMathOperator{\Ker}{Ker} % Kernel
\DeclareMathOperator{\rank}{rank}
\DeclareMathOperator{\Spec}{Spec} % spectrum
\DeclareMathOperator{\Tr}{Tr} % trace
\DeclareMathOperator{\pr}{pr} % projection
\DeclareMathOperator{\ext}{ext} % extension
\DeclareMathOperator{\pred}{pred} % predecessor
\DeclareMathOperator{\dom}{dom} % domain
\DeclareMathOperator{\ran}{ran} % range
\DeclareMathOperator{\Hom}{Hom} % homomorphism
\DeclareMathOperator{\Mor}{Mor} % morphisms
\DeclareMathOperator{\End}{End} % endomorphism

\newcommand{\eps}{\epsilon}
\newcommand{\veps}{\varepsilon}
\newcommand{\ol}{\overline}
\newcommand{\ul}{\underline}
\newcommand{\wt}{\widetilde}
\newcommand{\wh}{\widehat}
\newcommand{\vocab}[1]{\textbf{\color{blue} #1}}
\providecommand{\half}{\frac{1}{2}}
\newcommand{\dang}{\measuredangle} %% Directed angle
\newcommand{\ray}[1]{\overrightarrow{#1}}
\newcommand{\seg}[1]{\overline{#1}}
\newcommand{\arc}[1]{\wideparen{#1}}
\DeclareMathOperator{\cis}{cis}
\DeclareMathOperator*{\lcm}{lcm}
\DeclareMathOperator*{\argmin}{arg min}
\DeclareMathOperator*{\argmax}{arg max}
\newcommand{\cycsum}{\sum_{\mathrm{cyc}}}
\newcommand{\symsum}{\sum_{\mathrm{sym}}}
\newcommand{\cycprod}{\prod_{\mathrm{cyc}}}
\newcommand{\symprod}{\prod_{\mathrm{sym}}}
\newcommand{\Qed}{\begin{flushright}\qed\end{flushright}}
\newcommand{\parinn}{\setlength{\parindent}{1cm}}
\newcommand{\parinf}{\setlength{\parindent}{0cm}}
% \newcommand{\norm}{\|\cdot\|}
\newcommand{\inorm}{\norm_{\infty}}
\newcommand{\opensets}{\{V_{\alpha}\}_{\alpha\in I}}
\newcommand{\oset}{V_{\alpha}}
\newcommand{\opset}[1]{V_{\alpha_{#1}}}
\newcommand{\lub}{\text{lub}}
\newcommand{\del}[2]{\frac{\partial #1}{\partial #2}}
\newcommand{\Del}[3]{\frac{\partial^{#1} #2}{\partial^{#1} #3}}
\newcommand{\deld}[2]{\dfrac{\partial #1}{\partial #2}}
\newcommand{\Deld}[3]{\dfrac{\partial^{#1} #2}{\partial^{#1} #3}}
\newcommand{\lm}{\lambda}
\newcommand{\uin}{\mathbin{\rotatebox[origin=c]{90}{$\in$}}}
\newcommand{\usubset}{\mathbin{\rotatebox[origin=c]{90}{$\subset$}}}
\newcommand{\lt}{\left}
\newcommand{\rt}{\right}
\newcommand{\bs}[1]{\boldsymbol{#1}}
\newcommand{\exs}{\exists}
\newcommand{\st}{\strut}
\newcommand{\dps}[1]{\displaystyle{#1}}

\newcommand{\sol}{\setlength{\parindent}{0cm}\textbf{\textit{Solution:}}\setlength{\parindent}{1cm} }
\newcommand{\solve}[1]{\setlength{\parindent}{0cm}\textbf{\textit{Solution: }}\setlength{\parindent}{1cm}#1 \Qed}

% Things Lie
\newcommand{\kb}{\mathfrak b}
\newcommand{\kg}{\mathfrak g}
\newcommand{\kh}{\mathfrak h}
\newcommand{\kn}{\mathfrak n}
\newcommand{\ku}{\mathfrak u}
\newcommand{\kz}{\mathfrak z}
\DeclareMathOperator{\Ext}{Ext} % Ext functor
\DeclareMathOperator{\Tor}{Tor} % Tor functor
\newcommand{\gl}{\opname{\mathfrak{gl}}} % frak gl group
\renewcommand{\sl}{\opname{\mathfrak{sl}}} % frak sl group chktex 6

% More script letters etc.
\newcommand{\SA}{\mathcal A}
\newcommand{\SB}{\mathcal B}
\newcommand{\SC}{\mathcal C}
\newcommand{\SF}{\mathcal F}
\newcommand{\SG}{\mathcal G}
\newcommand{\SH}{\mathcal H}
\newcommand{\OO}{\mathcal O}

\newcommand{\SCA}{\mathscr A}
\newcommand{\SCB}{\mathscr B}
\newcommand{\SCC}{\mathscr C}
\newcommand{\SCD}{\mathscr D}
\newcommand{\SCE}{\mathscr E}
\newcommand{\SCF}{\mathscr F}
\newcommand{\SCG}{\mathscr G}
\newcommand{\SCH}{\mathscr H}

% Mathfrak primes
\newcommand{\km}{\mathfrak m}
\newcommand{\kp}{\mathfrak p}
\newcommand{\kq}{\mathfrak q}

% number sets
\newcommand{\RR}[1][]{\ensuremath{\ifstrempty{#1}{\mathbb{R}}{\mathbb{R}^{#1}}}}
\newcommand{\NN}[1][]{\ensuremath{\ifstrempty{#1}{\mathbb{N}}{\mathbb{N}^{#1}}}}
\newcommand{\ZZ}[1][]{\ensuremath{\ifstrempty{#1}{\mathbb{Z}}{\mathbb{Z}^{#1}}}}
\newcommand{\QQ}[1][]{\ensuremath{\ifstrempty{#1}{\mathbb{Q}}{\mathbb{Q}^{#1}}}}
\newcommand{\CC}[1][]{\ensuremath{\ifstrempty{#1}{\mathbb{C}}{\mathbb{C}^{#1}}}}
\newcommand{\PP}[1][]{\ensuremath{\ifstrempty{#1}{\mathbb{P}}{\mathbb{P}^{#1}}}}
\newcommand{\HH}[1][]{\ensuremath{\ifstrempty{#1}{\mathbb{H}}{\mathbb{H}^{#1}}}}
\newcommand{\FF}[1][]{\ensuremath{\ifstrempty{#1}{\mathbb{F}}{\mathbb{F}^{#1}}}}
% expected value
\newcommand{\EE}{\ensuremath{\mathbb{E}}}
\newcommand{\charin}{\text{ char }}
\DeclareMathOperator{\sign}{sign}
\DeclareMathOperator{\Aut}{Aut}
\DeclareMathOperator{\Inn}{Inn}
\DeclareMathOperator{\Syl}{Syl}
\DeclareMathOperator{\Gal}{Gal}
\DeclareMathOperator{\GL}{GL} % General linear group
\DeclareMathOperator{\SL}{SL} % Special linear group

%---------------------------------------
% BlackBoard Math Fonts :-
%---------------------------------------

%Captital Letters
\newcommand{\bbA}{\mathbb{A}}	\newcommand{\bbB}{\mathbb{B}}
\newcommand{\bbC}{\mathbb{C}}	\newcommand{\bbD}{\mathbb{D}}
\newcommand{\bbE}{\mathbb{E}}	\newcommand{\bbF}{\mathbb{F}}
\newcommand{\bbG}{\mathbb{G}}	\newcommand{\bbH}{\mathbb{H}}
\newcommand{\bbI}{\mathbb{I}}	\newcommand{\bbJ}{\mathbb{J}}
\newcommand{\bbK}{\mathbb{K}}	\newcommand{\bbL}{\mathbb{L}}
\newcommand{\bbM}{\mathbb{M}}	\newcommand{\bbN}{\mathbb{N}}
\newcommand{\bbO}{\mathbb{O}}	\newcommand{\bbP}{\mathbb{P}}
\newcommand{\bbQ}{\mathbb{Q}}	\newcommand{\bbR}{\mathbb{R}}
\newcommand{\bbS}{\mathbb{S}}	\newcommand{\bbT}{\mathbb{T}}
\newcommand{\bbU}{\mathbb{U}}	\newcommand{\bbV}{\mathbb{V}}
\newcommand{\bbW}{\mathbb{W}}	\newcommand{\bbX}{\mathbb{X}}
\newcommand{\bbY}{\mathbb{Y}}	\newcommand{\bbZ}{\mathbb{Z}}

%---------------------------------------
% MathCal Fonts :-
%---------------------------------------

%Captital Letters
\newcommand{\mcA}{\mathcal{A}}	\newcommand{\mcB}{\mathcal{B}}
\newcommand{\mcC}{\mathcal{C}}	\newcommand{\mcD}{\mathcal{D}}
\newcommand{\mcE}{\mathcal{E}}	\newcommand{\mcF}{\mathcal{F}}
\newcommand{\mcG}{\mathcal{G}}	\newcommand{\mcH}{\mathcal{H}}
\newcommand{\mcI}{\mathcal{I}}	\newcommand{\mcJ}{\mathcal{J}}
\newcommand{\mcK}{\mathcal{K}}	\newcommand{\mcL}{\mathcal{L}}
\newcommand{\mcM}{\mathcal{M}}	\newcommand{\mcN}{\mathcal{N}}
\newcommand{\mcO}{\mathcal{O}}	\newcommand{\mcP}{\mathcal{P}}
\newcommand{\mcQ}{\mathcal{Q}}	\newcommand{\mcR}{\mathcal{R}}
\newcommand{\mcS}{\mathcal{S}}	\newcommand{\mcT}{\mathcal{T}}
\newcommand{\mcU}{\mathcal{U}}	\newcommand{\mcV}{\mathcal{V}}
\newcommand{\mcW}{\mathcal{W}}	\newcommand{\mcX}{\mathcal{X}}
\newcommand{\mcY}{\mathcal{Y}}	\newcommand{\mcZ}{\mathcal{Z}}


%---------------------------------------
% Bold Math Fonts :-
%---------------------------------------

%Captital Letters
\newcommand{\bmA}{\boldsymbol{A}}	\newcommand{\bmB}{\boldsymbol{B}}
\newcommand{\bmC}{\boldsymbol{C}}	\newcommand{\bmD}{\boldsymbol{D}}
\newcommand{\bmE}{\boldsymbol{E}}	\newcommand{\bmF}{\boldsymbol{F}}
\newcommand{\bmG}{\boldsymbol{G}}	\newcommand{\bmH}{\boldsymbol{H}}
\newcommand{\bmI}{\boldsymbol{I}}	\newcommand{\bmJ}{\boldsymbol{J}}
\newcommand{\bmK}{\boldsymbol{K}}	\newcommand{\bmL}{\boldsymbol{L}}
\newcommand{\bmM}{\boldsymbol{M}}	\newcommand{\bmN}{\boldsymbol{N}}
\newcommand{\bmO}{\boldsymbol{O}}	\newcommand{\bmP}{\boldsymbol{P}}
\newcommand{\bmQ}{\boldsymbol{Q}}	\newcommand{\bmR}{\boldsymbol{R}}
\newcommand{\bmS}{\boldsymbol{S}}	\newcommand{\bmT}{\boldsymbol{T}}
\newcommand{\bmU}{\boldsymbol{U}}	\newcommand{\bmV}{\boldsymbol{V}}
\newcommand{\bmW}{\boldsymbol{W}}	\newcommand{\bmX}{\boldsymbol{X}}
\newcommand{\bmY}{\boldsymbol{Y}}	\newcommand{\bmZ}{\boldsymbol{Z}}
%Small Letters
\newcommand{\bma}{\boldsymbol{a}}	\newcommand{\bmb}{\boldsymbol{b}}
\newcommand{\bmc}{\boldsymbol{c}}	\newcommand{\bmd}{\boldsymbol{d}}
\newcommand{\bme}{\boldsymbol{e}}	\newcommand{\bmf}{\boldsymbol{f}}
\newcommand{\bmg}{\boldsymbol{g}}	\newcommand{\bmh}{\boldsymbol{h}}
\newcommand{\bmi}{\boldsymbol{i}}	\newcommand{\bmj}{\boldsymbol{j}}
\newcommand{\bmk}{\boldsymbol{k}}	\newcommand{\bml}{\boldsymbol{l}}
\newcommand{\bmm}{\boldsymbol{m}}	\newcommand{\bmn}{\boldsymbol{n}}
\newcommand{\bmo}{\boldsymbol{o}}	\newcommand{\bmp}{\boldsymbol{p}}
\newcommand{\bmq}{\boldsymbol{q}}	\newcommand{\bmr}{\boldsymbol{r}}
\newcommand{\bms}{\boldsymbol{s}}	\newcommand{\bmt}{\boldsymbol{t}}
\newcommand{\bmu}{\boldsymbol{u}}	\newcommand{\bmv}{\boldsymbol{v}}
\newcommand{\bmw}{\boldsymbol{w}}	\newcommand{\bmx}{\boldsymbol{x}}
\newcommand{\bmy}{\boldsymbol{y}}	\newcommand{\bmz}{\boldsymbol{z}}

%---------------------------------------
% Scr Math Fonts :-
%---------------------------------------

\newcommand{\sA}{{\mathscr{A}}}   \newcommand{\sB}{{\mathscr{B}}}
\newcommand{\sC}{{\mathscr{C}}}   \newcommand{\sD}{{\mathscr{D}}}
\newcommand{\sE}{{\mathscr{E}}}   \newcommand{\sF}{{\mathscr{F}}}
\newcommand{\sG}{{\mathscr{G}}}   \newcommand{\sH}{{\mathscr{H}}}
\newcommand{\sI}{{\mathscr{I}}}   \newcommand{\sJ}{{\mathscr{J}}}
\newcommand{\sK}{{\mathscr{K}}}   \newcommand{\sL}{{\mathscr{L}}}
\newcommand{\sM}{{\mathscr{M}}}   \newcommand{\sN}{{\mathscr{N}}}
\newcommand{\sO}{{\mathscr{O}}}   \newcommand{\sP}{{\mathscr{P}}}
\newcommand{\sQ}{{\mathscr{Q}}}   \newcommand{\sR}{{\mathscr{R}}}
\newcommand{\sS}{{\mathscr{S}}}   \newcommand{\sT}{{\mathscr{T}}}
\newcommand{\sU}{{\mathscr{U}}}   \newcommand{\sV}{{\mathscr{V}}}
\newcommand{\sW}{{\mathscr{W}}}   \newcommand{\sX}{{\mathscr{X}}}
\newcommand{\sY}{{\mathscr{Y}}}   \newcommand{\sZ}{{\mathscr{Z}}}


%---------------------------------------
% Math Fraktur Font
%---------------------------------------

%Captital Letters
\newcommand{\mfA}{\mathfrak{A}}	\newcommand{\mfB}{\mathfrak{B}}
\newcommand{\mfC}{\mathfrak{C}}	\newcommand{\mfD}{\mathfrak{D}}
\newcommand{\mfE}{\mathfrak{E}}	\newcommand{\mfF}{\mathfrak{F}}
\newcommand{\mfG}{\mathfrak{G}}	\newcommand{\mfH}{\mathfrak{H}}
\newcommand{\mfI}{\mathfrak{I}}	\newcommand{\mfJ}{\mathfrak{J}}
\newcommand{\mfK}{\mathfrak{K}}	\newcommand{\mfL}{\mathfrak{L}}
\newcommand{\mfM}{\mathfrak{M}}	\newcommand{\mfN}{\mathfrak{N}}
\newcommand{\mfO}{\mathfrak{O}}	\newcommand{\mfP}{\mathfrak{P}}
\newcommand{\mfQ}{\mathfrak{Q}}	\newcommand{\mfR}{\mathfrak{R}}
\newcommand{\mfS}{\mathfrak{S}}	\newcommand{\mfT}{\mathfrak{T}}
\newcommand{\mfU}{\mathfrak{U}}	\newcommand{\mfV}{\mathfrak{V}}
\newcommand{\mfW}{\mathfrak{W}}	\newcommand{\mfX}{\mathfrak{X}}
\newcommand{\mfY}{\mathfrak{Y}}	\newcommand{\mfZ}{\mathfrak{Z}}
%Small Letters
\newcommand{\mfa}{\mathfrak{a}}	\newcommand{\mfb}{\mathfrak{b}}
\newcommand{\mfc}{\mathfrak{c}}	\newcommand{\mfd}{\mathfrak{d}}
\newcommand{\mfe}{\mathfrak{e}}	\newcommand{\mff}{\mathfrak{f}}
\newcommand{\mfg}{\mathfrak{g}}	\newcommand{\mfh}{\mathfrak{h}}
\newcommand{\mfi}{\mathfrak{i}}	\newcommand{\mfj}{\mathfrak{j}}
\newcommand{\mfk}{\mathfrak{k}}	\newcommand{\mfl}{\mathfrak{l}}
\newcommand{\mfm}{\mathfrak{m}}	\newcommand{\mfn}{\mathfrak{n}}
\newcommand{\mfo}{\mathfrak{o}}	\newcommand{\mfp}{\mathfrak{p}}
\newcommand{\mfq}{\mathfrak{q}}	\newcommand{\mfr}{\mathfrak{r}}
\newcommand{\mfs}{\mathfrak{s}}	\newcommand{\mft}{\mathfrak{t}}
\newcommand{\mfu}{\mathfrak{u}}	\newcommand{\mfv}{\mathfrak{v}}
\newcommand{\mfw}{\mathfrak{w}}	\newcommand{\mfx}{\mathfrak{x}}
\newcommand{\mfy}{\mathfrak{y}}	\newcommand{\mfz}{\mathfrak{z}}


\title{\Huge{Apuntes de análisis de\\ variable compleja}}
\author{}
\date{\number\year}

\begin{document}

\maketitle
\clearpage
\noindent Apuntes de las clases de \textit{Análisis de variable compleja} dadas por \textit{Juan Matías Sepulcre Martínez} y transcritos a \LaTeX
\hspace{0cm} por \textit{Víctor Mira Ramírez} durante el curso 2023-2024 del grado en Física de la \textit{Universidad de Alicante}.
\pagebreak
\tableofcontents
\pagebreak

\chapter{El cuerpo de los números complejos}
  \section{Definiciones básicas}
    \definicion{Número complejo}{
      Un \textbf{número complejo} $z$ es un par ordenado de números reales $a,b$ escrito como
      $\boxed{z=\left(a,b\right)}$ en coordenadas cartesianas. Existe una notación equivalente,
      la forma binómica: $\boxed{z=a+ib}$ siendo $i=\left(0,1\right)$. \\

      El conjunto de los número complejos se denota por: $C:=\left\{(a,b):a,b\in\bbR\right\}$
    }
    \comentario{
      Siempre que $a=0$ sea un número imaginario puro, y $b=0$ sea un número real.
    }
    \definicion{Conjugado}{
      Llamamos conjugado de un número complejo al número denotado $\boxed{\ovl{z}=a-ib}$, siendo
      $z=a+ib$. Geométricamente, podemos decir que el eje real actúa de 'espejo' del número en el plano.
    }
    \comentario{
      Llamamos $\bbC$ al cuerpo de los numeros complejos. $\bbC$ es un cuerpo conmutativo, pero no totalmente ordenado. En cambio, cualquier ecuación algebraica
      tiene solución en los complejos. De todas formas, el teorema fundamental del álgebra nos asegura que tendrá n soluciones en los complejos
    }
    \comentario{
      Cuando los coeficientes de una ecuación algebraica son reales, las soluciones complejas vienen por pares.
    }
    \teorema{Operaciones elementales}{
      \renewcommand{\arraystretch}{1.6}
      \begin{tabular}{lll}
        \textbf{SUMA}& $\left(a+bi\right)+\left(c+di\right)=\left(a+c\right)+\left(b+d\right)i$\\
        \textbf{RESTA}& $\left(a+bi\right)-\left(c+di\right)=\left(a-c\right)+\left(b-d\right)i$\\
        \textbf{PRODUCTO}& $\left(a+bi\right)\cdot\left(c+di\right)=\left(ac-bd\right)+\left(ad+bc\right)i$ &(teniendo en cuenta que $i^2=-1$)\\
        \rule{0pt}{2.3em}\textbf{DIVISIÓN}& $\cfrac{a+bi}{c+di}\ =\ \cfrac{a+bi}{c+di}\cdot\cfrac{c-di}{c-di}\ =\ \cfrac{ac+bd}{c^2+d^2}+\left(\cfrac{bc-ad}{c^2+d^2}\right)i$ &(multiplicando por el conjugado)\\
      \end{tabular}
    }
    \clearpage
    \comentario{
      El elemento unidad es $1+0i$ y el elemento inverso es $\frac{a}{a^2+b^2}-\frac{b}{a^2+b^2}i$. Para que un número complejo tenga elemento inverso, debe ser distinto de cero.
      El producto de un número complejo por su elemento inverso es la unidad.
    }
    \definicion{Componentes de los complejos}{
      Llamamos \textbf{módulo} del número complejo $z=a+bi$ a la cantidad $\boxed{\sqrt{a^2+b^2}}$ denotada $\abs{z}$\\

      \vspace{0.1cm} Llamamos \textbf{argumento} del número complejo $z=a+bi$ al ángulo que forma el semieje positivo de abcisas
      con la recta que contiene el vector $\left(a,b\right)$. Se denota $\text{Arg }z=\alpha$ y se expresa en radianes.\\
      $\boxed{\alpha=\arctan{\left(\frac{b}{a}\right)}}$ si $a\neq 0$
    }
    \definicion{Módulo}{
      Llamamos \textbf{módulo} de un número complejo $z=a+bi$, y lo denotamos $\abs{z}$, a la cantidad
      $$\abs{z}=\sqrt{a^2+b^2}$$
    }
    \definicion{Argumento}{
      Llamamos \textbf{argumento} de un número complejo $z=a+bi$ al ángula que forma el semieje positivo de abcisas con la recta que contiene al vector.
      El argumento de $z$ se representa por Arg($z$)$=\alpha$, y se expresa normalmente en radianes.

      $$\alpha=\arctan{\frac{b}{a}}, \text{si} a\neq0$$
      $$\alpha=\frac{\pi}{2}, \text{si} a=0,b>0$$
      $$\alpha=\frac{3\pi}{2}, \text{si} a=0,b<0$$

      Si el ángulo se encuentra en el intervalo $[-\pi,\pi)$ lo llamaremos argumento principal.
    }
    \comentario{lol
      %Arg z (z/=0) = $\pi/2$ si $a=0$ y $b>0$
      %Arg z (z/=0) = $-\pi/2$ si $a=0$ y $b<0$
      %Arg z (z/=0) = $0$ si $a>0$ y $b=0$
      %Arg z (z/=0) = $-\pi$ si $a<0$ y $b=0$
      %Arg z (z/=0) = $\arctan(b/a)$ si $a>0$ y $b>0$
      %Arg z (z/=0) = $-arctan(-b/a)$ si $a>0$ y $b<0$
      %Arg z (z/=0) = $-arctan(b/-a)+pi$ si $a<0$ y $b>0$
      %Arg z (z/=0) = $-arctan(b/a)-pi$ si $a<0$ y $b<0$
      }
    \comentario{forma exponencial: el desarrollo en serie de la exponencial es: $e^x=\sum_{n=0}\frac{x^n}{n!}=1+x+\frac{x^2}{2}+\frac{x^3}{3}+...$
    si introducimos un número complejo en la exponencial: $e^{iy}=1+(iy)+\frac{(iy)^2}{2}+\frac{(iy)^3}{3!}+...$

    Si analizamos el valor de $i^n$ en función de n, %i= 1 si n=4k, k en Z
                                                        %i si n=4k+1
                                                        %-1 si n=4k+2
                                                        %-i si n=4k+3
    entonces vemos como la exponencial compleja queda ahora como:
    $e^{iy}=1+iy-\frac{y^2}{2}-\frac{iy^3}{3!}+\frac{y^4}{4!}+...=
    \left(1-\frac{y^2}{2!}+\frac{y^4}{4!}+...\right)+i\left(y-\frac{y^3}{3!}+\frac{y^5}{5!}\right)=
    \cos(y)+i\sin(y)$

    $e^z=e^xe^{iy}=e^x\left(\cos(y)+i\sin(y)\right)$ con $z=x+iy$
    }
    \clearpage

    %
    %
    %
    %
  \section{Analiticidad}
    \definicion{Función armónica conjugada}{
      Sea $u\colon\mcD\subset\bbR^2\rightarrow\bbR$ una función armónica en un abierto
      de $\mcD\subset\bbR^2$ diremos que $v\colon\mcD\subset\bbR^2\rightarrow\bbR$ es
      una \textbf{función armónica conjugada} de $u$ en $\mcD$ si $v$ es armónica en
      $\mcD$ y satisfacen las condiciones de \textit{Cauchy-Riemann}, (o equivalentemente
      la función $f(x+iy)=u(x,y)+iv(x,y)$ es holomorfa en $\left\{x+iy\in\bbC\colon (x,y)
      \in\mcD\right\}$)
    }
    \comentario{
      Una función armónica es aquella que satisface la ecuación de Laplace.
    }
    \teorema{}{
      Sea $u(x,y)\colon\mcD\rightarrow\bbR$ es una función armónica de $\mcD$ y
      consideramos $v$ una región rectangular contenida en $\mcD$. Entonces existe una
      conjugada armónica de $u(x,y)$ en $v$.
    }
  \section{Algunas funciones elementales}
    \subsection{Función exponencial}
      \definicion{}{
      \[f(z)=e^z=e^{x} e^{iy}=e^x\left(\cos y+i\sin y\right)\]
      }
      \teorema{}{
        \begin{enumerate}
          \item $e^z\neq 0\hspace{0.6cm}\forall z\in\bbC$
          \item $\abs{e^z}=e^{Re(z)}\hspace{0.6cm}z\in\bbC$
          \item $arg\left(e^z\right)=\left\{Im(z)+2\pi k,\hspace{0.2cm}
                k\in\bbZ\right\}\hspace{0.8cm}\forall z\in\bbC$
          \item $\ovl{\left(e^z\right)}=e^{\ovl{z}}\hspace{0.8cm}z\in\bbC$
          \item $e^x=1\Leftrightarrow x=0 \hspace{1cm} x\in\bbR$\\
                $e^z=1\Leftrightarrow z=2\pi ki \hspace{0.8cm} z\in\bbC$
          \item $\lim_{x\rightarrow\infty} e^x=\infty\hspace{0.8cm} x\in\bbR$\\
                $\nexists \lim_{\abs{z}\rightarrow\infty} e^z=\infty\hspace{0.8cm} x\in\bbR$
          \item $e^x$ es entera (derivable en todo punto de $\bbC$) $\left(e^z
                \right)^{\prime}=e^z$
          \item $e^{z+\omega}=e^z\cdot e^\omega, \hspace{0.8cm}z,w\in\bbC$\\
                $\left(e^z\right)^n=e^{nz}, \hspace{0.58cm}n\in\bbN, z\in\bbC$
          % (e^iy = cosy + iseny           (e^iy)^n=e^) ... pedir formula de moivre

        \end{enumerate}
      }
      \ejemplo{$e^{iz}-e^{-iz}=4i$}{
        $e^{iz}-e^{-iz}=4i\Longleftrightarrow e^{iz}-e^{-iz}-4i=0\Longleftrightarrow
        e^{2iz}-4ie^{iz}-1=0$\\

        Si $\omega=e^{iz}\Longrightarrow \boxed{\omega^2-4i\omega-1=0}$\\
        \vspace{0.2cm}
        $w=\dfrac{4i\pm\sqrt{-16+4}}{2}=\dfrac{4i\pm\sqrt{-12}}{2}=2i\pm\sqrt{3}=
        2\pm\sqrt{3}i\Longrightarrow \boxed{e^{iz}=(2\pm\sqrt{3})i}$
      }
    \subsection{Función logarítmica}
      \definicion{}{
        Se introduce por la necesidad de solucionar ecuaciones como la anterior.
        \[x=e^y\hspace{0.2cm}\Longleftrightarrow\hspace{0.2cm}y=\log x, \hspace{1cm}
        x\>0, y\in\bbR\]
        Sea $z\in\bbC -{0}$, definimos el logaritmo principal de $z$, y lo denotamos
        por $\log z$, como
        \[\log z=\ln\abs{z}+i\cdot Arg(z)\]
        Vemos que $e^{\log z}=e^{\log\abs{z}+Arg(z)}=e^{\ln\abs{z}}e^{Arg(z)}=\abs{z}
        e^{Arg(z)}=z$\\
        El conjunto de todos los logaritmos de z será:
        \[\log z = \left\{\ln\abs{z}+i\left(Arg(z)+2\pi k\right), k\in\bbZ\right\}\]
      }
      \ejemplo{}{
        \begin{enumerate}
          \item Si $z=x>0\Rightarrow \log z = \ln\abs{z}+i\cdot Arg(z)= lnx$\\
                $\log z=\left\{\ln x + 2\pi k i, k\in\bbZ\right\}$
          \item Si $z=-x>0\Rightarrow \log z = \ln x-i\cdot (-\pi)$ (argumento de $z$)\\
                $\log z=\left\{\ln x + -(\pi+2\pi k), k\in\bbZ\right\}$
          \item Si $z=ix, x>0\Rightarrow \log z = \ln x+i\dfrac{\pi}{2}$\\
                $\log z=\left\{\ln x + i\left(\dfrac{\pi}{2}+2\pi k\right), k\in\bbZ
                \right\}$
        \end{enumerate}
      }
      \comentario{
        Retomando la ecuación del ejemplo anterior,

        \[e^{iz}=(2\pm\sqrt{3})i=
        \begin{cases}
          &(2+\sqrt{3}i) \leftrightarrow iz=\log(2+\sqrt{3})i \leftrightarrow 
            z=\left(\dfrac{\pi}{2}+2\pi k\right) - iln(2+\sqrt{3})\\
          &(2-\sqrt{3}i) \leftrightarrow iz=\log(2-\sqrt{3})i \leftrightarrow 
            z=\left(\dfrac{\pi}{2}+2\pi k\right) - iln(2-\sqrt{3})\\
        \end{cases}, k\in\bbZ\]
      }
      \teorema{Propiedades}{
        \begin{enumerate}
          \item Log $z$ es holomorfa en $\bbC - \left[-\infty,0\right]\implies$ 
                de hecho, no es continua en $(-\infty,0]$
          \item $\log_{\theta_0}=z$ es holomorfa en $\bbC-\left\{z\in\bbC, \arg(z)=
                \theta_0\right\}$
          \item $e^{\log_{\theta_0}z}=z, \hspace{0.8cm}\forall z\in\bbC, \arg(z)=\theta_0$
                y $\left(\log_{\theta_0}\right)^{\prime}=\dfrac{1}{z}$
          \item $\log_{\theta_0}e^z=z \hspace{0.8cm}\forall z=x+iy, \theta_0 <= 
                y<\theta_0+2\pi$
                $z=x+iy, e^z=e^{x} e^{iy}\implies\log_{\theta_0}e^z=z$\\ cuando 
                $y\in\left[\theta_0, \theta_0+2\pi\right]$
        \end{enumerate}
        % PEDIR DEMOSTRACIÓN
      }
      \definicion{}{
        Sea $\theta_0\in\bbR$, tomamos $z\neq0$, $z=re^{i\theta}$, $r>0$, 
        $\theta_0<=\theta=\theta_0+2\pi$ y entonces $\log_{\theta_0}z=\ln\abs{z}+i\theta$\\
        
        Si $\theta_0=-\pi \implies \log_{\theta_0}z=Log z$\\
        Si $\theta_0=0 \implies \log_{0}z=\ln\abs{z}+i\theta, \hspace{0.8cm} 0<=\theta<2\pi$
      }
    \subsection{Función potencia}
      \definicion{Potencia de exponente arbitrario}{
        Sea $z\in\bbC \setminus {0}$ y $\alpha\in\bbC$, tomamos por definición $z^\alpha$, llamada
        potencia de exponente arbitrario como el conjunto de todos los valores dados por:
        \begin{equation}
          z^\alpha = e^{\alpha\log z}
        \end{equation}
        Dohde $\log z$ representa el conjunto de todos los logaritmos de $z$.
      }
      \definicion{Función exponencial general}{
        Sea $a\in\bbC\setminus{0}$, tomaremos por definición $a^z$, llamada función 
        exponencial general como el conjunto de todos los valores dados por:
        \begin{equation}
          a^z = \exp\left(z\log a\right)
        \end{equation}
        Donde $\log a$ es el conjunto de todos los logaritmos de $a$.
      }
      \ejemplo{}{
        \begin{itemize}
          \item $(-2)^{\frac12}=e^{\frac12 \log(-2)}=e^{\frac12\left(\ln 2+i(\pi+2\pi k)\right)}$,
                con $k\in\bbZ$ (tomando la primera definición, $z=-2$ y $\alpha=\frac12$)
          \item $2^i=e^{i\log(2)}=e^{i(\ln 2+2\pi ki)}=e^{-2\pi k}e^{i\ln 2}=
                e^{-2\pi k\left(\cos(\ln 2)+i\sin(\ln 2)\right)}$ con $k\in\bbZ$
          \item $(-1)^{\frac{1}{\pi}}= e^{\frac{1}{\pi}\log(-1)}= e^{\frac{1}{\pi}(\pi+2\pi k)i}
                =e^i\cdot e^{2ki}=e^{(2k+1)i}$, con $k\in\bbZ$
        \end{itemize}
      }
      \ejemplo{Potencia de exponente entero}{
        Sea $\alpha\in\bbZ$, entonces $f(z)=z^\alpha=e^{\alpha\log z}=e^{\alpha(\ln\abs{z}+
        i(Arg(2)+2\pi k))}=e^{\alpha\ln\abs{z}}\cdot e^{i\alpha Arg(z)}\cdot e^{i\alpha 2\pi k}
        =\abs{z}^\alpha \cdot e^{i\alpha Arg(2)}$ función univaluada
      }
      \comentario{
        Cuando tomemos $k=0$ en la función logarítmica, entonces obtenemos la denominada 
        \textbf{rama principal}.
      }
      \ejemplo{Función multiforme/aplicación multivaluada}{
        $f(z)=z^{\frac12}$ con $z\in\bbC\setminus{0}$ Tomando $z=re^{i\theta}$ con $r>0$ y 
        $-\pi\leq\theta <\pi$ La llamada rama principal de $z^\frac12$ es $f_1(z)=
        e^{\frac12\left(\ln(r)+i\theta\right)}=\sqrt{r}\cdot e^\frac{\theta}{2}i$ Otra rama
        con el ¿?. $\arg(z)=\pi$ viene dada por $-f_1(z)=f_2(z)=\sqrt{r}\cdot 
        e^{\frac{\theta +2\pi}{2}i}$\\

        Si $k=2 \rightarrow e^{\frac12\left(\ln r+i(\theta + 4\pi)\right)}=\sqrt{r}\cdot 
        e^{\frac{i\theta}{2}}\cdot e^{2\pi i}=f_1(z)$
      }
    \subsection{Funciones trigonométricas}
        $\cos    z=\dfrac{e^{iz}+e^{-iz}}{2}$ con $z\in\bbC$\\
        $\sin    z=\dfrac{e^{iz}-e^{-iZ}}{2i}$ con $z\in\bbC$\\
        $\tan    z=\dfrac{\sin z}{\cos z}=\dfrac{1}{i}\dfrac{e^{iz}-e^{-iz}}{e^{iz}+e^{-iz}}=
          \dfrac{1}{i}\dfrac{e^{2iz}-1}{e^{2iz}+1}$ es holomorfa en $\bbC$ excepto en 
          $\cos  z=0 \Leftrightarrow z=\frac{\pi}{2} + \pi k$ con $k\in\bbZ$
        $\cot  z=\dfrac{\cos z}{\sin z}=i\dfrac{e^{iz}+e^{-iz}}{e^{iz}-e^{-iz}}=
          i\dfrac{e^{2iz}+1}{e^{2iz}-1}$
        $\sec    z=\dfrac{1}{\cos z}$
        $\cosec  z=\dfrac{1}{\sin z}$
        
        Hiperbólicas

        $\cosh z=\dfrac{e^z+e^{-z}}{2}$
        $\sinh z=\dfrac{e^z-e^{-z}}{2}$
        $\tanh z=\dfrac{\sinh}{\cosh}=\dfrac{e^z-e^{-z}}{e^z+e^{-z}}=\dfrac{e^{2z}-1}{e^{2z}+1}$

        \teorema{Propiedades}{
          \begin{enumerate}
            \item $\ovl{\cos z}=\cos\ovl{z}$
            \item $\ovl{\sin z}=\sin\ovl{z}$
            \item $\cos z=0\Leftrightarrow z=(2\pi +1)\frac{\pi}{2},n\in\bbZ$
            \item $\sin z=0\Leftrightarrow z=n\pi,n\in\bbZ$
            \item $\sinh z=0\Leftrightarrow z=n\pi i,n\in\bbZ$
            \item $\cosh z=0\Leftrightarrow z=(2\pi+1)\frac{\pi}{2}i,nºin\bbZ$
            \item $\sinh(z)=-i\sin(iz)$
            \item $\cosh(z)=\cos(iz)$
          \end{enumerate}
        }
        \comentario{
          Comparación con el caso real:
          \begin{itemize}
            \item $e^z=\sum_{n=0}\dfrac{z^n}{n!}$
            \item $\sin z=\sum_{n=0}\dfrac{(-1)^n z^{2n+1}}{(2n+1)!}$
            \item $\cosh z=\sum_{n=0}\dfrac{(-1)^n z^{2n}}{(2n)!}$
            \item $\sinh z=\sum_{n=0}\dfrac{z^{2n+1}}{(2n+1)!}$
            \item $\cosh z=\sum_{n=0}\dfrac{z^{2n}}{(2n)!}$
          \end{itemize}
        }

\chapter{Integración compleja}
  \section{Preliminares topológicos}
    \definicion{Entorno perforado}{
      Llamamos \textbf{entorno perforado} de un punto $z_0\in\bbC$ a un abierto de la forma
      $\left\{z\in\bbC\colon 0<\abs{z-z_0}<\epsilon\right\}$, con $\epsilon >0$
    }
    \definicion{Tipos de conjuntos}{
      \begin{itemize}
        \item Diremos que dos conjuntos $A$ y $B$ de $\bbC$ están \textbf{espaciados} 
              si $\ovl{A}\cap B=A\cap\ovl{B}=\varnothing$\\ Siendo $\ovl{A}$ la clausura,
              la adherencia de $A$.
        \item Diremos que un conjunto del plano es \textbf{conexo} si no puede ser
              escrito como unión de dos subconjuntos no vaciós y separados.
      
        \item Diremos que un conjunto $P\in\bbC$ es \textbf{poligonalmente conexo} si cada
              par de puntos de $P $ pueden ser unidos mediante una poligonal contenida en $P$.
              (una ponigonal es una unión finita de segmentos).
        \item Un conjunto $E\in\bbC$ es \textbf{estrellado} si existe un punto $a\in E$
              tal que $\left[a,z\right]\subset E \hspace{0.5cm}\forall z\in E$
        \item Un conjunto $C\in\bbC$ es \textbf{convexo} si $\left[z,w\right]\subset C 
              \hspace{0.5cm}\forall z,w\in C$ (cualquier segmento formado por puntos
              del conjunto está dentro del conjunto).

              \textit{Nota: Sea $U\in\bbC$ un conjunto abierto, entonces $U$ es conexo si y sólo
              si es poligonalmente conexo.}
        \item Llamamos \textbf{simplemente conexo} al conjunto $S\subset \bbC$ del cual cada 
              curva cerrada simple (sin autointersecciones) en $S$ puede contraerse dentro
              del conjunto hasta ser un punto (no tiene agujeros).
      \end{itemize}
    }
  \section{Integración sobre caminos}
    \definicion{Curva}{
      Llamamos \textbf{curva} a una aplicación continua $\gamma\colon\left[a,b\right]
      \rightarrow\bbC$ con $a<b$, tal que a un número real $t\in\left[a,b\right]$ le
      corresponde un número complejo $\gamma(t)=x(t)+iy(t)$ donde $x(t)$ e $y(t)$ 
      son funciones reales y continuas.
      \begin{itemize}
        \item La \textbf{traza o trayectoria} de la curva $\gamma(\left[a,b\right])=
              \left\{\gamma(t)\colon a\leq t\leq b\right\}$ será representado por 
              $\gamma^*$
        \item Diremos que la curva es \textbf{cerrada} cuando $\gamma(a)=\gamma(b)$
      \end{itemize}
    }
    \definicion{Camino}{
      Una curva $\gamma\colon\left[a,b\right]\rightarrow\bbC$ es \textbf{diferenciable}
      cuando $\gamma$ es derivable en todo punto de $\left[a,b\right]$. 
      Una curva $\gamma\colon\left[a,b\right]\rightarrow\bbC$ es \textbf{suave} si es
      diferiencable (o de clase $\mcC^1(\left[a,b\right])$), si $\gamma$ es
      derivable en $\left[a,b\right]$ y su derivada es continua.
      Llamamos \textbf{camino} a una curva suave a trozos (diferenciable con continuidad
      a trozos).
    }
    \comentario{
      Los casos de curvas rectificables (aquellas curvas parametrizadas por $\gamma(t)
      =x(t)+iy(t)$), con $t\in[a,b]$,  para las que existe:
      \[\sup\left\{\sum_{j=1}^{n}\sqrt{(x(t_j)-x(t_{j-1}))^2+(y(t_j)-y(t_{j-1}))^2}\right\}\]
      Con $P$ en el conjunto de posibles particiones de $\left[a,b\right]$. En estos
      casos, la longitud de la curva se calcula como: \[L(\gamma)=\displaystyle\int_a^b
      \abs{\gamma^\prime(t)}\ dt=\int_a^b\sqrt{(x^\prime(t)^2+y^\prime(t)^2)}\ dt\]
    }
    \definicion{Integral compleja}{
      Sea $\gamma\colon\left[a,b\right]\rightarrow\bbC$ un camino y $f$ una función 
      continua en $\gamma^*\in\bbC$, definimos la \textbf{integral conmpleja} de $f$
      a lo largo $\gamma$ por:
      \[\int_{\gamma}f(z)\ dz=\int_a^bf\left(\gamma(t)\right)\cdot\gamma^\prime(t)\ dt\]
      \textit{Nota: Cuando no se diga nada, se supondra que el sentido de recorrido 
      sobre un camino cerrado será el antihorario.}
    }
    \ejemplo{$\int_\gamma \ovl{z}\ dz$, donde $\gamma^*$ es el segmento $[1+i,2+4i]$}{
      $\gamma(t)=(1-t)(1+i)+t\cdot(2+4i) \Longrightarrow \gamma^\prime(t)=-(1+i)+2+4i=
      1+3i\hspace{0.3cm}$ con $t\in[0,1]$.\\

      Si separamos en parte real y imaginaria nos queda:\\
      $\gamma(t)=1+t+i(1+3t)\Longrightarrow\int_\gamma\ovl{z}\ dz=\int_0^1(1+t
      -i(1+3t))(1+3i)\ dt=\int_0^1(1+t-i(1+3t)+3i+3it+3+9t)\ dt=\int_0^1 (4+10t+2i)\ dt
      =\left[(4+2i)t+5t^2\right]^1_0=4+2i+5=9+2i$
    }
    \ejemplo{$\int_\gamma \ovl{z}\ dz$, donde $\gamma_2$ es el trozo de parábola que
    une $1+i$ con $2+4i$}{
      $\gamma(t)=t+it^2 \Longrightarrow \gamma^\prime(t)=1+2it$ con $t\in[1,2]
      \Longrightarrow$\\
      $f(\gamma_2(t))=t-it^2\Longrightarrow\int_{\gamma_2}\ovl{z}\ dz=\int_1^2(t-it^2)
      (1+2it)\ dt=9+\frac73 i$
    }
\end{document}
