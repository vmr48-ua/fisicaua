\documentclass{report}
\usepackage[spanish]{babel}
\addto\captionsspanish{
  \renewcommand{\contentsname}%
    {Índice}%
}

\input{preamble}
\input{macros}
\input{letterfonts}

\title{\Huge{Apuntes de análisis de\\ variable compleja}}
\author{}
\date{\number\year}

\begin{document}

\maketitle
\clearpage
\noindent Apuntes de las clases de \textit{Análisis de variable compleja} dadas por \textit{Juan Matías Sepulcre Martínez} y transcritos a \LaTeX
\hspace{0cm} por \textit{Víctor Mira Ramírez} durante el curso 2023-2024 del grado en Física de la \textit{Universidad de Alicante}.
\pagebreak
\tableofcontents
\pagebreak

\chapter{El cuerpo de los números complejos}
  \section{Definiciones básicas}
    \definicion{Número complejo}{
      Un \textbf{número complejo} $z$ es un par ordenado de números reales $a,b$ escrito como
      $\boxed{z=\left(a,b\right)}$ en coordenadas cartesianas. Existe una notación equivalente,
      la forma binómica: $\boxed{z=a+ib}$ siendo $i=\left(0,1\right)$. \\

      El conjunto de los número complejos se denota por: $C:=\left\{(a,b):a,b\in\bbR\right\}$
    }
    \comentario{
      Siempre que $a=0$ sea un número imaginario puro, y $b=0$ sea un número real.
    }
    \definicion{Conjugado}{
      Llamamos conjugado de un número complejo al número denotado $\boxed{\ovl{z}=a-ib}$, siendo
      $z=a+ib$. Geométricamente, podemos decir que el eje real actúa de 'espejo' del número en el plano.
    }
    \comentario{
      Llamamos $\bbC$ al cuerpo de los numeros complejos. $\bbC$ es un cuerpo conmutativo, pero no totalmente ordenado. En cambio, cualquier ecuación algebraica
      tiene solución en los complejos. De todas formas, el teorema fundamental del álgebra nos asegura que tendrá n soluciones en los complejos
    }
    \comentario{
      Cuando los coeficientes de una ecuación algebraica son reales, las soluciones complejas vienen por pares.
    }
    \teorema{Operaciones elementales}{
      \renewcommand{\arraystretch}{1.6}
      \begin{tabular}{lll}
        \textbf{SUMA}& $\left(a+bi\right)+\left(c+di\right)=\left(a+c\right)+\left(b+d\right)i$\\
        \textbf{RESTA}& $\left(a+bi\right)-\left(c+di\right)=\left(a-c\right)+\left(b-d\right)i$\\
        \textbf{PRODUCTO}& $\left(a+bi\right)\cdot\left(c+di\right)=\left(ac-bd\right)+\left(ad+bc\right)i$ &(teniendo en cuenta que $i^2=-1$)\\
        \rule{0pt}{2.3em}\textbf{DIVISIÓN}& $\cfrac{a+bi}{c+di}\ =\ \cfrac{a+bi}{c+di}\cdot\cfrac{c-di}{c-di}\ =\ \cfrac{ac+bd}{c^2+d^2}+\left(\cfrac{bc-ad}{c^2+d^2}\right)i$ &(multiplicando por el conjugado)\\
      \end{tabular}
    }
    \clearpage
    \comentario{
      El elemento unidad es $1+0i$ y el elemento inverso es $\frac{a}{a^2+b^2}-\frac{b}{a^2+b^2}i$. Para que un número complejo tenga elemento inverso, debe ser distinto de cero.
      El producto de un número complejo por su elemento inverso es la unidad.
    }
    \definicion{Componentes de los complejos}{
      Llamamos \textbf{módulo} del número complejo $z=a+bi$ a la cantidad $\boxed{\sqrt{a^2+b^2}}$ denotada $\abs{z}$\\

      \vspace{0.1cm} Llamamos \textbf{argumento} del número complejo $z=a+bi$ al ángulo que forma el semieje positivo de abcisas
      con la recta que contiene el vector $\left(a,b\right)$. Se denota $\text{Arg }z=\alpha$ y se expresa en radianes.\\
      $\boxed{\alpha=\arctan{\left(\frac{b}{a}\right)}}$ si $a\neq 0$
    }
    \definicion{Módulo}{
      Llamamos \textbf{módulo} de un número complejo $z=a+bi$, y lo denotamos $\abs{z}$, a la cantidad
      $$\abs{z}=\sqrt{a^2+b^2}$$
    }
    \definicion{Argumento}{
      Llamamos \textbf{argumento} de un número complejo $z=a+bi$ al ángula que forma el semieje positivo de abcisas con la recta que contiene al vector.
      El argumento de $z$ se representa por Arg($z$)$=\alpha$, y se expresa normalmente en radianes.

      $$\alpha=\arctan{\frac{b}{a}}, \text{si} a\neq0$$
      $$\alpha=\frac{\pi}{2}, \text{si} a=0,b>0$$
      $$\alpha=\frac{3\pi}{2}, \text{si} a=0,b<0$$

      Si el ángulo se encuentra en el intervalo $[-\pi,\pi)$ lo llamaremos argumento principal.
    }
    \comentario{lol
      %Arg z (z/=0) = $\pi/2$ si $a=0$ y $b>0$
      %Arg z (z/=0) = $-\pi/2$ si $a=0$ y $b<0$
      %Arg z (z/=0) = $0$ si $a>0$ y $b=0$
      %Arg z (z/=0) = $-\pi$ si $a<0$ y $b=0$
      %Arg z (z/=0) = $\arctan(b/a)$ si $a>0$ y $b>0$
      %Arg z (z/=0) = $-arctan(-b/a)$ si $a>0$ y $b<0$
      %Arg z (z/=0) = $-arctan(b/-a)+pi$ si $a<0$ y $b>0$
      %Arg z (z/=0) = $-arctan(b/a)-pi$ si $a<0$ y $b<0$
      }
    \comentario{forma exponencial: el desarrollo en serie de la exponencial es: $e^x=\sum_{n=0}\frac{x^n}{n!}=1+x+\frac{x^2}{2}+\frac{x^3}{3}+...$
    si introducimos un número complejo en la exponencial: $e^{iy}=1+(iy)+\frac{(iy)^2}{2}+\frac{(iy)^3}{3!}+...$

    Si analizamos el valor de $i^n$ en función de n, %i= 1 si n=4k, k en Z
                                                        %i si n=4k+1
                                                        %-1 si n=4k+2
                                                        %-i si n=4k+3
    entonces vemos como la exponencial compleja queda ahora como:
    $e^{iy}=1+iy-\frac{y^2}{2}-\frac{iy^3}{3!}+\frac{y^4}{4!}+...=
    \left(1-\frac{y^2}{2!}+\frac{y^4}{4!}+...\right)+i\left(y-\frac{y^3}{3!}+\frac{y^5}{5!}\right)=
    \cos(y)+i\sin(y)$

    $e^z=e^xe^{iy}=e^x\left(\cos(y)+i\sin(y)\right)$ con $z=x+iy$
    }
    \clearpage

    %
    %
    %
    %
  \section{Analiticidad}
    \definicion{Función armónica conjugada}{
      Sea $u\colon\mcD\subset\bbR^2\rightarrow\bbR$ una función armónica en un abierto
      de $\mcD\subset\bbR^2$ diremos que $v\colon\mcD\subset\bbR^2\rightarrow\bbR$ es
      una \textbf{función armónica conjugada} de $u$ en $\mcD$ si $v$ es armónica en
      $\mcD$ y satisfacen las condiciones de \textit{Cauchy-Riemann}, (o equivalentemente
      la función $f(x+iy)=u(x,y)+iv(x,y)$ es holomorfa en $\left\{x+iy\in\bbC\colon (x,y)
      \in\mcD\right\}$)
    }
    \comentario{
      Una función armónica es aquella que satisface la ecuación de Laplace.
    }
    \teorema{}{
      Sea $u(x,y)\colon\mcD\rightarrow\bbR$ es una función armónica de $\mcD$ y
      consideramos $v$ una región rectangular contenida en $\mcD$. Entonces existe una
      conjugada armónica de $u(x,y)$ en $v$.
    }
  \section{Algunas funciones elementales}
    \subsection{Función exponencial}
      \definicion{}{
      \[f(z)=e^z=e^{x} e^{iy}=e^x\left(\cos y+i\sin y\right)\]
      }
      \teorema{}{
        \begin{enumerate}
          \item $e^z\neq 0\hspace{0.6cm}\forall z\in\bbC$
          \item $\abs{e^z}=e^{Re(z)}\hspace{0.6cm}z\in\bbC$
          \item $arg\left(e^z\right)=\left\{Im(z)+2\pi k,\hspace{0.2cm}
                k\in\bbZ\right\}\hspace{0.8cm}\forall z\in\bbC$
          \item $\ovl{\left(e^z\right)}=e^{\ovl{z}}\hspace{0.8cm}z\in\bbC$
          \item $e^x=1\Leftrightarrow x=0 \hspace{1cm} x\in\bbR$\\
                $e^z=1\Leftrightarrow z=2\pi ki \hspace{0.8cm} z\in\bbC$
          \item $\lim_{x\rightarrow\infty} e^x=\infty\hspace{0.8cm} x\in\bbR$\\
                $\nexists \lim_{\abs{z}\rightarrow\infty} e^z=\infty\hspace{0.8cm} x\in\bbR$
          \item $e^x$ es entera (derivable en todo punto de $\bbC$) $\left(e^z
                \right)^{\prime}=e^z$
          \item $e^{z+\omega}=e^z\cdot e^\omega, \hspace{0.8cm}z,w\in\bbC$\\
                $\left(e^z\right)^n=e^{nz}, \hspace{0.58cm}n\in\bbN, z\in\bbC$
          % (e^iy = cosy + iseny           (e^iy)^n=e^) ... pedir formula de moivre

        \end{enumerate}
      }
      \ejemplo{$e^{iz}-e^{-iz}=4i$}{
        $e^{iz}-e^{-iz}=4i\Longleftrightarrow e^{iz}-e^{-iz}-4i=0\Longleftrightarrow
        e^{2iz}-4ie^{iz}-1=0$\\

        Si $\omega=e^{iz}\Longrightarrow \boxed{\omega^2-4i\omega-1=0}$\\
        \vspace{0.2cm}
        $w=\dfrac{4i\pm\sqrt{-16+4}}{2}=\dfrac{4i\pm\sqrt{-12}}{2}=2i\pm\sqrt{3}=
        2\pm\sqrt{3}i\Longrightarrow \boxed{e^{iz}=(2\pm\sqrt{3})i}$
      }
    \subsection{Función logarítmica}
      \definicion{}{
        Se introduce por la necesidad de solucionar ecuaciones como la anterior.
        \[x=e^y\hspace{0.2cm}\Longleftrightarrow\hspace{0.2cm}y=\log x, \hspace{1cm}
        x\>0, y\in\bbR\]
        Sea $z\in\bbC -{0}$, definimos el logaritmo principal de $z$, y lo denotamos
        por $\log z$, como
        \[\log z=\ln\abs{z}+i\cdot Arg(z)\]
        Vemos que $e^{\log z}=e^{\log\abs{z}+Arg(z)}=e^{\ln\abs{z}}e^{Arg(z)}=\abs{z}
        e^{Arg(z)}=z$\\
        El conjunto de todos los logaritmos de z será:
        \[\log z = \left\{\ln\abs{z}+i\left(Arg(z)+2\pi k\right), k\in\bbZ\right\}\]
      }
      \ejemplo{}{
        \begin{enumerate}
          \item Si $z=x>0\Rightarrow \log z = \ln\abs{z}+i\cdot Arg(z)= lnx$\\
                $\log z=\left\{\ln x + 2\pi k i, k\in\bbZ\right\}$
          \item Si $z=-x>0\Rightarrow \log z = \ln x-i\cdot (-\pi)$ (argumento de $z$)\\
                $\log z=\left\{\ln x + -(\pi+2\pi k), k\in\bbZ\right\}$
          \item Si $z=ix, x>0\Rightarrow \log z = \ln x+i\dfrac{\pi}{2}$\\
                $\log z=\left\{\ln x + i\left(\dfrac{\pi}{2}+2\pi k\right), k\in\bbZ
                \right\}$
        \end{enumerate}
      }
      \comentario{
        Retomando la ecuación del ejemplo anterior,

        \[e^{iz}=(2\pm\sqrt{3})i=
        \begin{cases}
          &(2+\sqrt{3}i) \leftrightarrow iz=\log(2+\sqrt{3})i \leftrightarrow 
            z=\left(\dfrac{\pi}{2}+2\pi k\right) - iln(2+\sqrt{3})\\
          &(2-\sqrt{3}i) \leftrightarrow iz=\log(2-\sqrt{3})i \leftrightarrow 
            z=\left(\dfrac{\pi}{2}+2\pi k\right) - iln(2-\sqrt{3})\\
        \end{cases}, k\in\bbZ\]
      }
      \teorema{Propiedades}{
        \begin{enumerate}
          \item Log $z$ es holomorfa en $\bbC - \left[-\infty,0\right]\implies$ 
                de hecho, no es continua en $(-\infty,0]$
          \item $\log_{\theta_0}=z$ es holomorfa en $\bbC-\left\{z\in\bbC, \arg(z)=
                \theta_0\right\}$
          \item $e^{\log_{\theta_0}z}=z, \hspace{0.8cm}\forall z\in\bbC, \arg(z)=\theta_0$
                y $\left(\log_{\theta_0}\right)^{\prime}=\dfrac{1}{z}$
          \item $\log_{\theta_0}e^z=z \hspace{0.8cm}\forall z=x+iy, \theta_0 <= 
                y<\theta_0+2\pi$
                $z=x+iy, e^z=e^{x} e^{iy}\implies\log_{\theta_0}e^z=z$\\ cuando 
                $y\in\left[\theta_0, \theta_0+2\pi\right]$
        \end{enumerate}
        % PEDIR DEMOSTRACIÓN
      }
      \definicion{}{
        Sea $\theta_0\in\bbR$, tomamos $z\neq0$, $z=re^{i\theta}$, $r>0$, 
        $\theta_0<=\theta=\theta_0+2\pi$ y entonces $\log_{\theta_0}z=\ln\abs{z}+i\theta$\\
        
        Si $\theta_0=-\pi \implies \log_{\theta_0}z=Log z$\\
        Si $\theta_0=0 \implies \log_{0}z=\ln\abs{z}+i\theta, \hspace{0.8cm} 0<=\theta<2\pi$
      }
    \subsection{Función potencia}
      \definicion{Potencia de exponente arbitrario}{
        Sea $z\in\bbC \setminus {0}$ y $\alpha\in\bbC$, tomamos por definición $z^\alpha$, llamada
        potencia de exponente arbitrario como el conjunto de todos los valores dados por:
        \begin{equation}
          z^\alpha = e^{\alpha\log z}
        \end{equation}
        Dohde $\log z$ representa el conjunto de todos los logaritmos de $z$.
      }
      \definicion{Función exponencial general}{
        Sea $a\in\bbC\setminus{0}$, tomaremos por definición $a^z$, llamada función 
        exponencial general como el conjunto de todos los valores dados por:
        \begin{equation}
          a^z = \exp\left(z\log a\right)
        \end{equation}
        Donde $\log a$ es el conjunto de todos los logaritmos de $a$.
      }
      \ejemplo{}{
        \begin{itemize}
          \item $(-2)^{\frac12}=e^{\frac12 \log(-2)}=e^{\frac12\left(\ln 2+i(\pi+2\pi k)\right)}$,
                con $k\in\bbZ$ (tomando la primera definición, $z=-2$ y $\alpha=\frac12$)
          \item $2^i=e^{i\log(2)}=e^{i(\ln 2+2\pi ki)}=e^{-2\pi k}e^{i\ln 2}=
                e^{-2\pi k\left(\cos(\ln 2)+i\sin(\ln 2)\right)}$ con $k\in\bbZ$
          \item $(-1)^{\frac{1}{\pi}}= e^{\frac{1}{\pi}\log(-1)}= e^{\frac{1}{\pi}(\pi+2\pi k)i}
                =e^i\cdot e^{2ki}=e^{(2k+1)i}$, con $k\in\bbZ$
        \end{itemize}
      }
      \ejemplo{Potencia de exponente entero}{
        Sea $\alpha\in\bbZ$, entonces $f(z)=z^\alpha=e^{\alpha\log z}=e^{\alpha(\ln\abs{z}+
        i(Arg(2)+2\pi k))}=e^{\alpha\ln\abs{z}}\cdot e^{i\alpha Arg(z)}\cdot e^{i\alpha 2\pi k}
        =\abs{z}^\alpha \cdot e^{i\alpha Arg(2)}$ función univaluada
      }
      \comentario{
        Cuando tomemos $k=0$ en la función logarítmica, entonces obtenemos la denominada 
        \textbf{rama principal}.
      }
      \ejemplo{Función multiforme/aplicación multivaluada}{
        $f(z)=z^{\frac12}$ con $z\in\bbC\setminus{0}$ Tomando $z=re^{i\theta}$ con $r>0$ y 
        $-\pi\leq\theta <\pi$ La llamada rama principal de $z^\frac12$ es $f_1(z)=
        e^{\frac12\left(\ln(r)+i\theta\right)}=\sqrt{r}\cdot e^\frac{\theta}{2}i$ Otra rama
        con el ¿?. $\arg(z)=\pi$ viene dada por $-f_1(z)=f_2(z)=\sqrt{r}\cdot 
        e^{\frac{\theta +2\pi}{2}i}$\\

        Si $k=2 \rightarrow e^{\frac12\left(\ln r+i(\theta + 4\pi)\right)}=\sqrt{r}\cdot 
        e^{\frac{i\theta}{2}}\cdot e^{2\pi i}=f_1(z)$
      }
    \subsection{Funciones trigonométricas}
        $\cos    z=\dfrac{e^{iz}+e^{-iz}}{2}$ con $z\in\bbC$\\
        $\sin    z=\dfrac{e^{iz}-e^{-iZ}}{2i}$ con $z\in\bbC$\\
        $\tan    z=\dfrac{\sin z}{\cos z}=\dfrac{1}{i}\dfrac{e^{iz}-e^{-iz}}{e^{iz}+e^{-iz}}=
          \dfrac{1}{i}\dfrac{e^{2iz}-1}{e^{2iz}+1}$ es holomorfa en $\bbC$ excepto en 
          $\cos  z=0 \Leftrightarrow z=\frac{\pi}{2} + \pi k$ con $k\in\bbZ$
        $\cot  z=\dfrac{\cos z}{\sin z}=i\dfrac{e^{iz}+e^{-iz}}{e^{iz}-e^{-iz}}=
          i\dfrac{e^{2iz}+1}{e^{2iz}-1}$
        $\sec    z=\dfrac{1}{\cos z}$
        $\cosec  z=\dfrac{1}{\sin z}$
        
        Hiperbólicas

        $\cosh z=\dfrac{e^z+e^{-z}}{2}$
        $\sinh z=\dfrac{e^z-e^{-z}}{2}$
        $\tanh z=\dfrac{\sinh}{\cosh}=\dfrac{e^z-e^{-z}}{e^z+e^{-z}}=\dfrac{e^{2z}-1}{e^{2z}+1}$

        \teorema{Propiedades}{
          \begin{enumerate}
            \item $\ovl{\cos z}=\cos\ovl{z}$
            \item $\ovl{\sin z}=\sin\ovl{z}$
            \item $\cos z=0\Leftrightarrow z=(2\pi +1)\frac{\pi}{2},n\in\bbZ$
            \item $\sin z=0\Leftrightarrow z=n\pi,n\in\bbZ$
            \item $\sinh z=0\Leftrightarrow z=n\pi i,n\in\bbZ$
            \item $\cosh z=0\Leftrightarrow z=(2\pi+1)\frac{\pi}{2}i,nºin\bbZ$
            \item $\sinh(z)=-i\sin(iz)$
            \item $\cosh(z)=\cos(iz)$
          \end{enumerate}
        }
        \comentario{
          Comparación con el caso real:
          \begin{itemize}
            \item $e^z=\sum_{n=0}\dfrac{z^n}{n!}$
            \item $\sin z=\sum_{n=0}\dfrac{(-1)^n z^{2n+1}}{(2n+1)!}$
            \item $\cosh z=\sum_{n=0}\dfrac{(-1)^n z^{2n}}{(2n)!}$
            \item $\sinh z=\sum_{n=0}\dfrac{z^{2n+1}}{(2n+1)!}$
            \item $\cosh z=\sum_{n=0}\dfrac{z^{2n}}{(2n)!}$
          \end{itemize}
        }

\chapter{Integración compleja}
  \section{Preliminares topológicos}
    \definicion{Entorno perforado}{
      Llamamos \textbf{entorno perforado} de un punto $z_0\in\bbC$ a un abierto de la forma
      $\left\{z\in\bbC\colon 0<\abs{z-z_0}<\epsilon\right\}$, con $\epsilon >0$
    }
    \definicion{Tipos de conjuntos}{
      \begin{itemize}
        \item Diremos que dos conjuntos $A$ y $B$ de $\bbC$ están \textbf{espaciados} 
              si $\ovl{A}\cap B=A\cap\ovl{B}=\varnothing$\\ Siendo $\ovl{A}$ la clausura,
              la adherencia de $A$.
        \item Diremos que un conjunto del plano es \textbf{conexo} si no puede ser
              escrito como unión de dos subconjuntos no vaciós y separados.
      
        \item Diremos que un conjunto $P\in\bbC$ es \textbf{poligonalmente conexo} si cada
              par de puntos de $P $ pueden ser unidos mediante una poligonal contenida en $P$.
              (una ponigonal es una unión finita de segmentos).
        \item Un conjunto $E\in\bbC$ es \textbf{estrellado} si existe un punto $a\in E$
              tal que $\left[a,z\right]\subset E \hspace{0.5cm}\forall z\in E$
        \item Un conjunto $C\in\bbC$ es \textbf{convexo} si $\left[z,w\right]\subset C 
              \hspace{0.5cm}\forall z,w\in C$ (cualquier segmento formado por puntos
              del conjunto está dentro del conjunto).

              \textit{Nota: Sea $U\in\bbC$ un conjunto abierto, entonces $U$ es conexo si y sólo
              si es poligonalmente conexo.}
        \item Llamamos \textbf{simplemente conexo} al conjunto $S\subset \bbC$ del cual cada 
              curva cerrada simple (sin autointersecciones) en $S$ puede contraerse dentro
              del conjunto hasta ser un punto (no tiene agujeros).
      \end{itemize}
    }
  \section{Integración sobre caminos}
    \definicion{Curva}{
      Llamamos \textbf{curva} a una aplicación continua $\gamma\colon\left[a,b\right]
      \rightarrow\bbC$ con $a<b$, tal que a un número real $t\in\left[a,b\right]$ le
      corresponde un número complejo $\gamma(t)=x(t)+iy(t)$ donde $x(t)$ e $y(t)$ 
      son funciones reales y continuas.
      \begin{itemize}
        \item La \textbf{traza o trayectoria} de la curva $\gamma(\left[a,b\right])=
              \left\{\gamma(t)\colon a\leq t\leq b\right\}$ será representado por 
              $\gamma^*$
        \item Diremos que la curva es \textbf{cerrada} cuando $\gamma(a)=\gamma(b)$
      \end{itemize}
    }
    \definicion{Camino}{
      Una curva $\gamma\colon\left[a,b\right]\rightarrow\bbC$ es \textbf{diferenciable}
      cuando $\gamma$ es derivable en todo punto de $\left[a,b\right]$. 
      Una curva $\gamma\colon\left[a,b\right]\rightarrow\bbC$ es \textbf{suave} si es
      diferiencable (o de clase $\mcC^1(\left[a,b\right])$), si $\gamma$ es
      derivable en $\left[a,b\right]$ y su derivada es continua.
      Llamamos \textbf{camino} a una curva suave a trozos (diferenciable con continuidad
      a trozos).
    }
    \comentario{
      Los casos de curvas rectificables (aquellas curvas parametrizadas por $\gamma(t)
      =x(t)+iy(t)$), con $t\in[a,b]$,  para las que existe:
      \[\sup\left\{\sum_{j=1}^{n}\sqrt{(x(t_j)-x(t_{j-1}))^2+(y(t_j)-y(t_{j-1}))^2}\right\}\]
      Con $P$ en el conjunto de posibles particiones de $\left[a,b\right]$. En estos
      casos, la longitud de la curva se calcula como: \[L(\gamma)=\displaystyle\int_a^b
      \abs{\gamma^\prime(t)}\ dt=\int_a^b\sqrt{(x^\prime(t)^2+y^\prime(t)^2)}\ dt\]
    }
    \definicion{Integral compleja}{
      Sea $\gamma\colon\left[a,b\right]\rightarrow\bbC$ un camino y $f$ una función 
      continua en $\gamma^*\in\bbC$, definimos la \textbf{integral conmpleja} de $f$
      a lo largo $\gamma$ por:
      \[\int_{\gamma}f(z)\ dz=\int_a^bf\left(\gamma(t)\right)\cdot\gamma^\prime(t)\ dt\]
      \textit{Nota: Cuando no se diga nada, se supondra que el sentido de recorrido 
      sobre un camino cerrado será el antihorario.}
    }
    \ejemplo{$\int_\gamma \ovl{z}\ dz$, donde $\gamma^*$ es el segmento $[1+i,2+4i]$}{
      $\gamma(t)=(1-t)(1+i)+t\cdot(2+4i) \Longrightarrow \gamma^\prime(t)=-(1+i)+2+4i=
      1+3i\hspace{0.3cm}$ con $t\in[0,1]$.\\

      Si separamos en parte real y imaginaria nos queda:\\
      $\gamma(t)=1+t+i(1+3t)\Longrightarrow\int_\gamma\ovl{z}\ dz=\int_0^1(1+t
      -i(1+3t))(1+3i)\ dt=\int_0^1(1+t-i(1+3t)+3i+3it+3+9t)\ dt=\int_0^1 (4+10t+2i)\ dt
      =\left[(4+2i)t+5t^2\right]^1_0=4+2i+5=9+2i$
    }
    \ejemplo{$\int_\gamma \ovl{z}\ dz$, donde $\gamma_2$ es el trozo de parábola que
    une $1+i$ con $2+4i$}{
      $\gamma(t)=t+it^2 \Longrightarrow \gamma^\prime(t)=1+2it$ con $t\in[1,2]
      \Longrightarrow$\\
      $f(\gamma_2(t))=t-it^2\Longrightarrow\int_{\gamma_2}\ovl{z}\ dz=\int_1^2(t-it^2)
      (1+2it)\ dt=9+\frac73 i$
    }
    \teorema{Propiedades básicas de integrabilidad}{
      Sea $\gamma\colon[a,b]\rightarrow\bbC$ un camino, y consideremos dos funciones
      continuas $f$ y $g$ continuas en $\gamma^*$, entonces se cumplen las siguientes
      propiedades:
      \begin{enumerate}
        \item $\int_\gamma \left(f(z)+g(z)\right)\ dz=\int_\gamma f(z)\ dz+\int_\gamma g(z)\ dz$
        \item $\int_\gamma cf(z)\ dz=c\int_\gamma f(z)\ dz\hspace{0.3cm}\forall c\in\bbC$ 
        \item $\int_{-\gamma} f(z)\ dz= -\int_\gamma f(z)\ dz$, donde $-\gamma$ es el camino
              opuesto a $\gamma$: $(-\gamma)(t)=\gamma(b+a-t)$ con $t\in[a,b]$
        \item Si $\beta$ es otro camino tal que $\gamma+\beta$ está definido (el punto inicial
              de $\beta$ coincide con el punto final de $\gamma$) y la función $f$ es también 
              continua en $\beta^*$ y $(\gamma+\beta)^*$, entonces:\\
              $\int_{\beta+\gamma}f(z)\ dz=\int_\gamma f(z)\ dz+\int_\beta f(z)\ dz$
        \item La integral sobre un camino es invariante bajo una parametrización.
      \end{enumerate}
    }
    \teorema{Propiedades}{
      Sea $\gamma\colon[a,b]\rightarrow\bbC$ un camino y $f$ una función continua en $\gamma^*$
      entonces $\abs{\int_\gamma f(z)\ dz}\leq M_f(\gamma)L(\gamma)$ siendo $M_f(\gamma)=
      \max \left\{\abs{f(z)}, z\in\gamma^*\right\}$ y $L(\gamma)$ la longitud de $\gamma$.
    }
    \subsubsection*{Prueba}
      $\displaystyle\abs{\int_\gamma f(z)\ dz}=\abs{\int_a^b f(\gamma(t))\cdot\gamma^\prime(t)\ dt}
      \leq\int_a^b \abs{f(\gamma(t))}\abs{\gamma^\prime(t)}\ dt\leq M_f(\gamma)\int_a^b\abs{\gamma^\prime(t)}\ dt
      =M_f(\gamma)L(\gamma)$
    \ejemplo{}{
      Si $\gamma(t)=2e^{it}$ para $\dfrac{-\pi}{6}\leq t\leq\dfrac{\pi}{6}$ entonces
      $\abs{\displaystyle\int_\gamma \dfrac{1}{z^3+1}\ dz}\leq\dfrac{2\pi}{21}\Longleftrightarrow
      \dfrac{1}{\abs{z^3+1}}\leq\dfrac{1}{\abs{z^3}+1}\leq\dfrac{1}{8-1}=\dfrac{1}{7}\hspace{0.4cm}
      \forall z\in\gamma^*$\\
      
      Usando la propiedad $\abs{z^3+1}\geq\abs{z^3}-1$.\\

      Finalmente: $\abs{\int_\gamma\dfrac{1}{z^3+1}\ dz}\leq\dfrac{1}{7}\dfrac{2\pi}{3}=\dfrac{2\pi}{21}$
    }
    \definicion{Primitiva compleja de $f$}{
      Sea $f\colon U\rightarrow\bbC$ una función definida en un conjunto abierto $U$, diremos
      que $F\colon U\rightarrow\bbC$ es una primitiva de $f$ en $U$ si $F$ es holomorfa en $U$
      y $F^\prime(z)=f(z)\hspace{0.3cm}\forall z\in U$.
    }
    \ejemplo{}{
      Tomando la función $f\colon U=\bbC\setminus(-\infty,0]\rightarrow\bbC$ tal que
      $f(z)=\frac{1}{z}$, entonces $F(z)=Log z+ C$ con $c\in\bbC$ es una primitiva de 
      $f$ en $U$.
    }
    \teorema{Extensión del 2º Teorema Fundamental del Cálculo}{
      Supongamos que $f\colon U\rightarrow\bbC$ es una función continua en un conjunto abierto
      $U\subset\bbC$ y que $F$ es un aprimitiva de $f$ en $U$. Si $\gamma\colon[a,b]\rightarrow U$
      es un camino en $U$, entonces:
      \[\int_\gamma f(z)\ dz=F(z)|^{\gamma(b)}_{\gamma(a)}\]
      En particular, bajo las hipótesis anteriories tenemos que $\int_\gamma f(z)\ dz=0\hspace{0.4cm}
      \forall\gamma$ camino cerrado en $U$.
    }
    \subsubsection*{Prueba}
      \noindent Definamos $G(t)=F(\gamma(t))$ con $t\in[a,b]$, que es una función continua en $[a,b]
      \Longrightarrow G^\prime(t)=F^\prime(\gamma(t))\cdot\gamma^\prime(t)$ (excepto a la suma,
      una contidad finita de puntos).\\

      \noindent Por el 2º TFC, tenemos que: $\int_\gamma f(z)\ dz=\int_a^b f(\gamma(t))\cdot\gamma^\prime (t)\ dt=
      \int_a^b G(t)\ dt=G(b)-G(a)=F(\gamma(b))-F(\gamma(a))$\\
    \corolario{}{
      Supongamos que $f\colon U\rightarrow\bbC$ es una función holomorfa en un conjunto
      abierto y conexo $U$ y además $f^\prime(z)=0\hspace{0.3cm}\forall z\in U$, entonces
      $f$ es constante en $U$.
    }
    \subsubsection{Prueba}
      \noindent Tomamos $F(z)=f(z)$: Como $f^\prime(z)=0$ es continua en $U$, tenemos que
      $0=\int\gamma f^\prime(z)\ dz=f(z_2)-f(z_1)\hspace{0.3cm}\forall z_1,z_2\in U$.\\
      $\gamma$ une $z_1$ y $z_2$ $\Rightarrow$ $f$ es constante.

    \vspace{0.3cm}
    \ejemplo{}{
      Si $\gamma$ representa cualquier camino que conecte los puntos $0$ y $1+i$, por
      el teorema anterior:\\
      $\int_\gamma z^2\ dz=\dfrac{2^3}{3}|^{1+i}_{0}=\dfrac{(1+i)^3}{3}=\dfrac{2}{3}(-1+i)$
    }
    \ejemplo{}{
      $\int_\gamma P(z)\ dz=0$ cuando $\gamma$ es un camino cerrado y $P$ es un polinomio.
    }
    \teorema{Independencia del camino}{
      Supongamos que $f\colon U\rightarrow\bbC$ es una función continua en un conjunto
      abierto y conexo $U\subset\bbC$, las siguientes propiedades son equivalentes:
      \begin{itemize}
        \item $\int_\gamma f(z)\ dz$ es independiente del camino (por dos puntos prefijados).
        \item $\int_\gamma f(z)\ dz=0$ \hspace{0.3cm}$\forall \gamma$ camino cerrado en $U$.
        \item $f$ admite una primitiva en $U$.
      \end{itemize}
    }
    
    \comentario{
      Una propiedad importante de las integrales de línea para el caso de campos vectoriales
      es la vinculada con los llamados \textbf{campos conservativos} (los que admiten 
      función potencial). El   principio físico que hay detrás de estos campos es que el 
      trabajo realizado cuando una partícula recorre la trayectoria de la curva sometida
      al campo de fuerzas es igual del potencial de campo entre los extremos de la trayectoria.\\

      \noindent En el caso de la integral compleja, el valor depende generalmente del camino
      elegido, sin embargo, el teorema anterior afirma que la independencia del camino se 
      puede caracterizar a través de la existencia de una primitiva del integrando 
      continuo $f$ en el conjunto $U$, esto es, la existencia de una función $F$ holomorfa
      en $U$ con $F^\prime=f(z)\hspace{0.2cm}\forall z\in U$. Además, sabemos que
      $F^\prime(z)=\frac{\partial U}{\partial x}(x,y)-i\frac{\partial U}{\partial y}(x,y)
      =f(z)$.\\ Por tanto, $\ovl{F^\prime(z)}=\frac{\partial U}{\partial x}(x,y)+i\frac
      {\partial U}{\partial y}(x,y)=U(x,y)-iU(x,y)$ y deducimos que $\Delta U=\ovl{f}$,
      $\hspace{0.2cm}$ con $\ovl{f}=(u,-v)$

    }
\end{document}
