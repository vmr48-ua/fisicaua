\documentclass{report}
\usepackage[spanish]{babel}
\addto\captionsspanish{
  \renewcommand{\contentsname}%
    {Índice}%
}

\input{preamble}
\input{macros}
\input{letterfonts}

\title{\Huge{Apuntes de mecánica\\ cuántica 1}}
\author{}
\date{\number\year}

\begin{document}

\maketitle
\clearpage
\noindent Apuntes de las clases de \textit{Mecánica Cuántica I} dadas por \textit{María José Caturla} y transcritos a \LaTeX
\hspace{0cm} por \textit{Víctor Mira Ramírez} durante el curso 2023-2024 del grado en Física de la \textit{Universidad de Alicante}.
\pagebreak
\tableofcontents
\pagebreak

\chapter{Introducción}
  \section{Radiación de cuerpo negro}
    \noindent En el siglo XIX se buscaba el lfilamento que más radiación emitía (más luz), lo que propició el desarrollo de la física relacionada.
    Era posible medir ucar era el espectro de radiación de distintos cuerpos para estudiarlos. Aquí nace el concepto de cuerpo negro, un cuerpo
    que absorbe todas las radiaciones idílicamente. La emisión de radiación es una gráfica de energía frente a longitud de onda. 
    Experimentalmente se obtuvieron dos leyes:
    \definicion{Ley de desplazamiento de Wien}{
      La longitud de onda emitida depende de la temperatura del cuerpo emisor. A mayor temperatura, la longitud de onda emitida será menor (el pico). 
      $\boxed{\lambda_m\cdot T=B}$ donde $B$ es la \textit{Constante de Wien} $B=2.898\times 10^{-3}m\cdot k$}
    \comentario{
      Si $\lambda_m=500$, ¿Cuál es la temperatura de la superficie del Sol? $T=6000 k$
      }
    \definicion{Ley de Stefan-Boltzmann}{
      $$P\alpha T^4$$
      Permite calcular la potencia irradiada en función de la temperatura.
    }
    \noindent Una manera de teorizar un cuerpo negro es considerando una esfera con un agujero muy pequeño. La luz que entra por un agujero se ve atrapada dentro,
    simulando un cuerpo negro. Cuando calcularon la densidad energética de este modelo, sucedía un problema: la \textbf{Catástrofe Ultravioleta}. El modelo no podía
    simular las longitudes de onda más pequeñas, la energía tendía a infinito con una asíntota vertical. Clásicamente se denomina a este modelo el modelo de 
    Rayleigh-Jeans: $u(\lambda)d\lambda=8\pi\lambda^{-4}KTd\lambda$. 

    \vspace{0.4cm} \noindent Para resolver este problema, Max Planck tuvo un enfoque mucho más pragmático. Buscó una función que se aproximara a los datos experimentales para después deducir
    una posible teoría física de ahí. Él sabía que era necesario cambiar la constante de Boltzmann por otra que se acomodara mejor. Es en este punto donde introducie
    el concepto de la energía como cuantos. Nace aquí la famosa ecuación $\epsilon = h\cdot f$. Hoy en día ya le hemos puesto nombre a esta constante, la 
    \textbf{Constante de Planck}. Como la energía es discreta, no se puede usar una integral, debemos hacer uso de sumatorios, lo cual resuelve la catástrofe.

    \vspace{0.4cm} \noindent El enfoque cuántico de Planck: $u(\lambda)d\lambda=\dfrac{8\pi\lambda^{-5}hc}{\exp\left(\dfrac{hc}{\lambda k T}\right)-1}d\lambda$. 
    
    \noindent Planck no se contentó con esta explacación teórica y dedicó su vida a tratar de buscar teorías alternativas.
    La constante de Planck es: $h=6.626\times10^{-34}J\cdot s$. El hecho de que la constante sea tan pequeña explica porqué a nivel 
    macroscópico no observamos fenómenos cuánicos.
  \section{Efecto fotoeléctrico}
    \noindent El primero en observar este fenómeno fue Hertz, mientras realizaba un experiemento en el que trataba de probar que la luz era una onda. En dicho experimento,
    encontró un fenómeno que parecía demostrar lo contrario, que la luz se comportaba como una partícula: el efecto fotoeléctrico. Es remarcable el hecho de que Hertz
    publicara sus resultados a pesar de ser contradictorios.

    \vspace{0.4cm} \noindent El efecto fotoeléctrico se basa en que al irradiar con una onda electromagnética un metal se emiten electrones (o fotoelectrones) en algunas condiciones concretas.
    Al conectar un ánodo con una diferencia de potencial cercano al metal que irradiamos, observamos una corriente eléctrica generada, positiva si colocamos un 
    potencial positivo, y al ir reduciendo el potencial llegamos a un potencial límite donde deja de haber corriente eléctrica, el llamado \textbf{Potencial de
    frenado} ($V_0$). La frecuencia de la luz incidente es directamente proporcional con la intensidad de corriente. 
    
    \teorema{Problemas clásicos del \textbf{efecto fotoeléctrico}}{
    \begin{enumerate}
      \item No se entendía cómo el potencial de frenado es independiente de la intensidad de la luz. 
    
      \item Tampoco podía explicarse clásicamente cómo la emisión de
      los electrones no sucedía para cualquier longitud de onda, independientemente de la intensidad. Llamamos a la frecuencia por debajo de la cual no se producen
      fotoelectrones \textbf{frecuencia umbral}.

      \item Por último, la ausenncia de tiempo de retardo entre la llegada de la radiación al metal y la producción de fotoelectrones era inexplicable también.
    \end{enumerate}
    }

    \vspace{0.4cm} \noindent Fue Einstein quien resolvió este problema, basándose en la teoría de Planck. Einstein postuló que para poder empezar a arrancar
    electrones de un metal, debíamos superar una función de trabajo $\Phi$ dependiente del metal y del tratamiento de su superficie.
    El rigor en los experimentos era impescindible para mantener una coherencia en los resultados.

    \vspace{0.4cm} \noindent La energia incidente era proporcional a la constante de planck multiplicado por la frecuencia, $h f$. La energía cinética que 
    tendrán los electrones emitidos será $h f$ menos la función de trabajo: $E_c=hf-\Phi$ y como $E_c=e\abs{V_0}$, entonces 
    $e\abs{V_0}=hf-\Phi$. La frecuencia umbral será cuando la energía incidente sea exactamente igual a la función de trabajo,
    pudiendo despejar la frecuencia umbral $h f_u=\frac{\Phi}{h}$

  \section{Espectro de rayos X}
    \noindent Fueron descubiertos también en el siglo XIX de forma accidental mietras estudiaba rayos catódicos. Se dió cuenta que al hacer 
    incidir electrones sobre un material, aparecía una radiacion que no supo describir, y la bautizó como \textbf{Rayos X}. Se observó
    como estos rayos atravesaban materiales en mayor o menor grado en función de su densidad.

    \vspace{0.4cm} \noindent No se entendían los picos de la gráfica, así como la aparición de una frecuencia umbral. Fue Einstein quien se dió cuenta de la 
    similitud de esta gráfica con la del efecto fotoeléctrico. Al acelerar electrones, obtenemos una radiación asociada, siguiendo
    la ecuación anterior: $e\abs{V_0}=hf-\Phi$. 

    \definicion{Ley de Duane-Hunt}{
      Longitud de onda mínima a partir de la cuál aparece una radiación X asociada: 
      $$\lambda_m=\dfrac{1.24\times10^3}{V}m$$ 
      
      Suponiendo:\hspace{0.4cm} $e\abs{V_0} \geq \Phi$\hspace{0.4cm} $e\abs{V_0}=hf$ \hspace{0.4cm} $c=\lambda f$
    }
  \section{Espectros atómicos}
    \noindent Se aplicaba a ciertos gases un voltaje para que emitiesen luz. No se veía todo el espectro, sino solo líneas discretas (luz de forma no continua sino discreta). 
    Estos eran los espectros de emisión, también en espectros de absorción: luz blanca impacta en un material y al reflejarse (según el material) faltaban ciertas 
    líneas que correspondían a lo que absorbía y reflejaba el material. 

    \vspace{0.4cm} \noindent Para cada material se tienen determinadas longitudes de onda. En el estudio de esta sucesión de longitudes de onda se llega a la siguiente expresión 
    que puede describir de forma empírica la posición de dichas líneas: (Por ejemplo la serie del hidrógeno sólo tenía 4 líneas).

    \definicion{Ecuación de Balmer}{
      $$\lambda_m=\frac{m^2}{m^2-4}\cdot 364.6nm$$
      Donde $m$ es un entero. Sea $m=3,4,5,6...$ esto funcionaba fenomenal. No sólo daba las líneas ya conocidas, sino que predecía las que se descubrieron posteriormente.
    }

    \vspace{0.4cm} \noindent De forma paralela se consigue otra expresión válida para elementos alcalinos:

    \definicion{Ecuación de Rydberg-Ritz}{
      $$\frac{1}{\lambda_n}=R\left(\frac{1}{m^2}-\frac{1}{n^2}\right)\hspace{1cm}n>m\hspace{0.4cm}n,m=1,2,3...$$
      Donde $R$ es la \textbf{Constante de Rydberg}, casi independiente del material. ($R_H=1.096776\cdot10^7 m^{-1}$, $R>\inf =1.097373\cdot10^7 m^{-1}$)
    }

    \vspace{0.4cm} \noindent Esto funciona con el hidrógeno: el espectro es visible, pero además también puede predecir las líneas espectrales que se hallan en el espectro no visible (infrarrojos, UV). 
    La ecuación de Balmer se puede deducir a partir de esta.
    
    \vspace{0.4cm} \noindent Todo esto lleva a que Bohr desarrolle su teoría atómica, pero había dos problemas: 
    \begin{itemize}
      \item El momento angular de los electrones de las órbitas había de estar cuantizado. 
      \item Al estar el electrón orbitando alrededor de una carga, se pierde energía, cayendo en espiral y chocando contra el núcleo. 
    \end{itemize}

    \teorema{Postulados del \textbf{Modelo atómico de Bohr}}{
      \begin{enumerate}
        \item Los electrones orbitan alrededor del núcleo en estados estacionarios donde no se emite radiación (no siguiendo la mecánica clásica), es decir, son estables.
        \item La diferencia de energía entre niveles será aquella de la radiación que se emita. 
        \item El momento angular del electrón está cuantizado. Consecuentemente, la energía de los estados está cuantizada. 
      \end{enumerate}
    }

  \noindent A consecuencia de la cuantización de los radios vamos a demostrar la cuantización de la energía de los electrones
  $E_c=K=\frac12 mv^2=\frac12 m\left(\frac{kze^2}{mr}\right)=\frac12 \frac{kze^2}{r}=-\frac12 u \Longrightarrow E=U+E_c=U-\frac12 U=\frac12 U=-\frac12 \frac{kze^2}{r}$
  De aquí obtenemos una ecuación para la energía en función de n:
  $$\boxed{E_n=-\frac12 \frac{kze^2}{r_n}=-\frac12 \frac{mk^2e^4}{\hbar^2}\frac{z^2}{n^2}}$$
  Si llamamos $E_0$ a $-\frac12 \frac{kze^2}{r_n}=-\frac12 \frac{mk^2e^4}{\hbar^2}$, entonces obtenemos que 
  $E_n=-E_0\frac{z^2}{n^2}$ y como $h\nu=E_i-E_j \Rightarrow h\nu=-E_0\frac{z^2}{n^2_i}+E_0\frac{z^2}{n^2_f}=E_0z^2\left(\frac{1}{n^2_f}-\frac{1}{n^2_i}\right) \Rightarrow 
  \frac{1}{\lambda}=\frac{E_0z^2}{hc}\left(\frac{1}{n^2_f}-\frac{1}{n^2_i}\right)$ notando que $R_H=\frac{E_0}{hc}$ es la constante de Raylberg. Esto validó el postulado de Bohr
\chapter{La ecuación de Schrödinger I}
\chapter{La ecuación de Schrödinger II}

\end{document}