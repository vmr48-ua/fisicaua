\documentclass{report}
\usepackage[spanish]{babel}
\addto\captionsspanish{
  \renewcommand{\contentsname}%
    {Índice}%
}

\input{preamble}
\input{macros}
\input{letterfonts}

\title{\Huge{Apuntes de mecánica\\ cuántica 1}}
\author{}
\date{\number\year}

\begin{document}

\maketitle
\clearpage
\noindent Apuntes de las clases de \textit{Mecánica Cuántica I} dadas por \textit{María José Caturla} y transcritos a \LaTeX
\hspace{0cm} por \textit{Víctor Mira Ramírez} durante el curso 2023-2024 del grado en Física de la \textit{Universidad de Alicante}.
\pagebreak
\tableofcontents
\pagebreak

\chapter{Introducción}
  \vspace{-0.8cm}
  \section{Radiación de cuerpo negro}
    \noindent En el siglo XIX se buscaba el lfilamento que más radiación emitía (más luz), lo que propició el desarrollo de la física relacionada.
    Era posible medir ucar era el espectro de radiación de distintos cuerpos para estudiarlos. Aquí nace el concepto de cuerpo negro, un cuerpo
    que absorbe todas las radiaciones idílicamente. La emisión de radiación es una gráfica de energía frente a longitud de onda. 
    Experimentalmente se obtuvieron dos leyes:
    \definicion{Ley de desplazamiento de Wien}{
      La longitud de onda emitida depende de la temperatura del cuerpo emisor. A mayor temperatura, la longitud de onda emitida será menor (el pico). 
      \begin{equation}
        \lambda_m\cdot T=B
        \label{eq:LeyDesplazamientoWien}
      \end{equation}
      donde $B$ es la \textit{Constante de Wien} $B=2.898\times 10^{-3}m\cdot k$}
    \comentario{
      Si $\lambda_m=500$, ¿Cuál es la temperatura de la superficie del Sol? $T=6000 k$
      }
    \definicion{Ley de Stefan-Boltzmann}{
      \begin{equation}
        P\alpha T^4
        \label{eq:LeyStefan-Boltzmann}
      \end{equation}
      Permite calcular la potencia irradiada en función de la temperatura.
    }
    \noindent Una manera de teorizar un cuerpo negro es considerando una esfera con un agujero muy pequeño. La luz que entra por un agujero se ve atrapada dentro,
    simulando un cuerpo negro. Cuando calcularon la densidad energética de este modelo, sucedía un problema: la \textbf{Catástrofe Ultravioleta}. El modelo no podía
    simular las longitudes de onda más pequeñas, la energía tendía a infinito con una asíntota vertical. Clásicamente se denomina a este modelo el modelo de 
    Rayleigh-Jeans: $u(\lambda)d\lambda=8\pi\lambda^{-4}KTd\lambda$. 

    \vspace{0.4cm} \noindent Para resolver este problema, Max Planck tuvo un enfoque mucho más pragmático. Buscó una función que se aproximara a los datos experimentales para después deducir
    una posible teoría física de ahí. Él sabía que era necesario cambiar la constante de Boltzmann por otra que se acomodara mejor. Es en este punto donde introducie
    el concepto de la energía como cuantos. Nace aquí la famosa ecuación $\epsilon = h\cdot f$. Hoy en día ya le hemos puesto nombre a esta constante, la 
    \textbf{Constante de Planck}. Como la energía es discreta, no se puede usar una integral, debemos hacer uso de sumatorios, lo cual resuelve la catástrofe.

    \vspace{0.4cm} \noindent El enfoque cuántico de Planck: $u(\lambda)d\lambda=\dfrac{8\pi\lambda^{-5}hc}{\exp\left(\dfrac{hc}{\lambda k T}\right)-1}d\lambda$. 
    
    \vspace{0.1cm}\noindent Planck no se contentó con esta explacación teórica y dedicó su vida a tratar de buscar teorías alternativas.
    La constante de Planck es: $h=6.626\times10^{-34}J\cdot s$. El hecho de que la constante sea tan pequeña explica porqué a nivel 
    macroscópico no observamos fenómenos cuánicos.
  \section{Efecto fotoeléctrico}
    \noindent El primero en observar este fenómeno fue Hertz, mientras realizaba un experiemento en el que trataba de probar que la luz era una onda. En dicho experimento,
    encontró un fenómeno que parecía demostrar lo contrario, que la luz se comportaba como una partícula: el efecto fotoeléctrico. Es remarcable el hecho de que Hertz
    publicara sus resultados a pesar de ser contradictorios.

    \vspace{0.4cm} \noindent El efecto fotoeléctrico se basa en que al irradiar con una onda electromagnética un metal se emiten electrones (o fotoelectrones) en algunas condiciones concretas.
    Al conectar un ánodo con una diferencia de potencial cercano al metal que irradiamos, observamos una corriente eléctrica generada, positiva si colocamos un 
    potencial positivo, y al ir reduciendo el potencial llegamos a un potencial límite donde deja de haber corriente eléctrica, el llamado \textbf{Potencial de
    frenado} ($V_0$). La frecuencia de la luz incidente es directamente proporcional con la intensidad de corriente. 
    
    \teorema{Problemas clásicos del \textbf{efecto fotoeléctrico}}{
    \begin{enumerate}
      \item No se entendía cómo el potencial de frenado es independiente de la intensidad de la luz. 
    
      \item Tampoco podía explicarse clásicamente cómo la emisión de
      los electrones no sucedía para cualquier longitud de onda, independientemente de la intensidad. Llamamos a la frecuencia por debajo de la cual no se producen
      fotoelectrones \textbf{frecuencia umbral}.

      \item Por último, la ausenncia de tiempo de retardo entre la llegada de la radiación al metal y la producción de fotoelectrones era inexplicable también.
    \end{enumerate}
    }

    \vspace{0.4cm} \noindent Fue Einstein quien resolvió este problema, basándose en la teoría de Planck. Einstein postuló que para poder empezar a arrancar
    electrones de un metal, debíamos superar una función de trabajo $\Phi$ dependiente del metal y del tratamiento de su superficie.
    El rigor en los experimentos era impescindible para mantener una coherencia en los resultados.

    \vspace{0.4cm} \noindent La energia incidente era proporcional a la constante de planck multiplicado por la frecuencia, $h f$. La energía cinética que 
    tendrán los electrones emitidos será $h f$ menos la función de trabajo: $E_c=hf-\Phi$ y como $E_c=e\abs{V_0}$, entonces 
    $e\abs{V_0}=hf-\Phi$. La frecuencia umbral será cuando la energía incidente sea exactamente igual a la función de trabajo,
    pudiendo despejar la frecuencia umbral $h f_u=\frac{\Phi}{h}$

  \section{Espectro de rayos X}
    \noindent Fueron descubiertos también en el siglo XIX de forma accidental mietras estudiaba rayos catódicos. Se dió cuenta que al hacer 
    incidir electrones sobre un material, aparecía una radiacion que no supo describir, y la bautizó como \textbf{Rayos X}. Se observó
    como estos rayos atravesaban materiales en mayor o menor grado en función de su densidad.

    \vspace{0.4cm} \noindent No se entendían los picos de la gráfica, así como la aparición de una frecuencia umbral. Fue Einstein quien se dió cuenta de la 
    similitud de esta gráfica con la del efecto fotoeléctrico. Al acelerar electrones, obtenemos una radiación asociada, siguiendo
    la ecuación anterior: $e\abs{V_0}=hf-\Phi$. 

    \definicion{Ley de Duane-Hunt}{
      Longitud de onda mínima a partir de la cuál aparece una radiación X asociada: 
      \begin{equation}
        \lambda_m=\dfrac{1.24\times10^3}{V}m
        \label{eq:LeyDuane-Hunt}
      \end{equation}
      
      Suponiendo:\hspace{0.4cm} $e\abs{V_0} \geq \Phi$\hspace{0.4cm} $e\abs{V_0}=hf$ \hspace{0.4cm} $c=\lambda f$
    }
  \section{Espectros atómicos}
    \noindent Se aplicaba a ciertos gases un voltaje para que emitiesen luz. No se veía todo el espectro, sino solo líneas discretas (luz de forma no continua sino discreta). 
    Estos eran los espectros de emisión, también en espectros de absorción: luz blanca impacta en un material y al reflejarse (según el material) faltaban ciertas 
    líneas que correspondían a lo que absorbía y reflejaba el material. 

    \vspace{0.4cm} \noindent Para cada material se tienen determinadas longitudes de onda. En el estudio de esta sucesión de longitudes de onda se llega a la siguiente expresión 
    que puede describir de forma empírica la posición de dichas líneas: (Por ejemplo la serie del hidrógeno sólo tenía 4 líneas).

    \definicion{Ecuación de Balmer}{
      \begin{equation}
        \lambda_m=\frac{m^2}{m^2-4}\cdot 364.6nm
        \label{eq:Balmer}
      \end{equation}
      Donde $m$ es un entero. Sea $m=3,4,5,6...$ esto funcionaba fenomenal. No sólo daba las líneas ya conocidas, sino que predecía las que se descubrieron posteriormente.
    }

    \vspace{0.4cm} \noindent De forma paralela se consigue otra expresión válida para elementos alcalinos:

    \definicion{Ecuación de Rydberg-Ritz}{
      \begin{equation}
        \frac{1}{\lambda_n}=R\left(\frac{1}{m^2}-\frac{1}{n^2}\right)\hspace{1cm}n>m\hspace{0.4cm}n,m=1,2,3...
        \label{eq:Rydberg}
      \end{equation}
      Donde $R$ es la \textbf{Constante de Rydberg}, casi independiente del material. ($R_H=1.096776\cdot10^7 m^{-1}$, $R>\inf =1.097373\cdot10^7 m^{-1}$)
    }

    \vspace{0.4cm} \noindent Esto funciona con el hidrógeno: el espectro es visible, pero además también puede predecir las líneas espectrales que se hallan en el espectro no visible (infrarrojos, UV). 
    La ecuación de Balmer se puede deducir a partir de esta.
    
    \vspace{0.4cm} \noindent Todo esto lleva a que Bohr desarrolle su teoría atómica, pero había dos problemas: 
    \begin{itemize}
      \item El momento angular de los electrones de las órbitas había de estar cuantizado. 
      \item Al estar el electrón orbitando alrededor de una carga, se pierde energía, cayendo en espiral y chocando contra el núcleo. 
    \end{itemize}

    \teorema{Postulados del \textbf{Modelo atómico de Bohr}}{
      \begin{enumerate}
        \item Los electrones orbitan alrededor del núcleo en estados estacionarios donde no se emite radiación (no siguiendo la mecánica clásica), es decir, son estables.
        \item La diferencia de energía entre niveles será aquella de la radiación que se emita. 
        \item El momento angular del electrón está cuantizado. Consecuentemente, la energía de los estados está cuantizada. 
      \end{enumerate}
    }

    \noindent A consecuencia de la cuantización de los radios vamos a demostrar la cuantización de la energía de los electrones
    $E_c=K=\frac12 mv^2=\frac12 m\left(\frac{kze^2}{mr}\right)=\frac12 \frac{kze^2}{r}=-\frac12 U \Longrightarrow E=U+E_c=U-\frac12 U=\frac12 U=-\frac12 \frac{kze^2}{r}$
    
    \noindent De aquí obtenemos una ecuación para la energía en función de n:
    \begin{equation}
      \boxed{E_n=-\frac12 \frac{kze^2}{r_n}=-\frac12 \frac{mk^2e^4}{\hbar^2}\frac{z^2}{n^2}}
      \label{eq:EnergiaOrbitalBohr}
    \end{equation}

    \clearpage \noindent Si llamamos $E_0$ a $-\dfrac12 \dfrac{kze^2}{r_n}=-\dfrac12 \dfrac{mk^2e^4}{\hbar^2}$, entonces obtenemos que 
    $E_n=-E_0\dfrac{z^2}{n^2}$ y como $h\nu=E_i-E_j \Rightarrow h\nu=-E_0\dfrac{z^2}{n^2_i}+E_0\dfrac{z^2}{n^2_f}=E_0z^2\left(\dfrac{1}{n^2_f}-\dfrac{1}{n^2_i}\right) \Rightarrow 
    \dfrac{1}{\lambda}=\dfrac{E_0z^2}{hc}\left(\dfrac{1}{n^2_f}-\dfrac{1}{n^2_i}\right)$ notando que $R_H=\dfrac{E_0}{hc}$ es la constante de Raylberg.\vspace{0.2cm}
    
    \noindent Esto validó el postulado de Bohr.\\

    \noindent Hemos llegado a la misma expresión que teníamos antes. Esto 
    implica que tenemos distintos niveles energéticos estacionarios, es decir, que todas 
    las transiciones de $n>2$ van a emitir fotones con cierta longitud de onda discreta 
    (esto ya se veía experimentalmente, espectro de onda visible).\\

    \noindent La teoría de Bohr de nuevo es capaz de predecir el espectro,
    así como las longitudes de onda en el espectro no visible.

  \section{Experimento de Franck-Hertz}
    \noindent Cápsula con dos placas: cátodo que se calienta y de donde se desprenden 
    electrones, se acelera con cierta diferencia de potencial a una red metálica y 
    detrás otra placa para detectar electrones. Con esto se generaba una corriente 
    eléctrica que se medía con un amperímetro.\\

    \begin{wrapfigure}[10]{l}{.53\textwidth}
      \vspace{-0.6cm}
      \includegraphics[width=\textwidth]{fotos/T1.5.Franck-Hertz.png}
    \end{wrapfigure}

    \noindent Conforme se aumenta el potencial, más electrones llegan a la placa que 
    registra y mayor la corriente. Se observa que a cierto valor de voltaje, baja: 
    tenemos el metal a estudiar en gas, y lo que ocurre esq tenemos T suficiente para 
    que los e saltasen. 

    \vspace{0.4cm} \noindent Por tanto, la energía que gana la pierde al pasar de un 
    estado 1 a 2, los electrones salen sin T y por haber potencial que los repele no 
    llegan a la placa. Con esto se demuestra la cuantización de los estados. 

  \section{Longitud de onda de De Broglie}
    \noindent Llegado a este punto entra De Broglie, que propone que no solo las ondas 
    se pueden comportar como corpúsculo, sino corpúsculo como onda.\\
    
    
    \noindent 
    \begin{equation}
      \begin{rcases}
        &\text{ENERGÍA}\\
        &\text{FOTÓN}
      \end{rcases}
      E=pc\hspace{0.3cm}E=h\nu=\dfrac{hc}{\lambda}\hspace{0.2cm}\implies\hspace{0.2cm}
      \dfrac{hc}{\lambda}=pc\hspace{1cm}\boxed{\lambda=\dfrac{h}{p}}
      \label{eq:deBroglie}
    \end{equation}

    \vspace{0.4cm}
    \noindent Propuso que las partículas se pueden comportar como ondas y que su 
    longitud de onda viene determinada por su momento lineas (y por ello su velocidad).

    \vspace{0.8cm}
    \noindent Nos queda ver cómo los postulados de Bohr solucionan el hecho de que las 
    órbitas de los electrones no irradian.

    \noindent $L_n=n\hbar\hspace{1cm}mrv=n\hbar\hspace{1cm}2\pi r(mv)=nh
    \xRightarrow{\ref{eq:deBroglie}}2\pi r \cdot \dfrac{h}{\lambda}=nh
    \hspace{1cm}\boxed{2\pi r =n\lambda}$ 
    \hspace{0.6cm} \textit{(la $\lambda$ es proporcional a $r$).}

    \vspace{0.8cm}
    \noindent Podemos decir que la onda que forma el electron es \textbf{estacionaria} 
    (el origen y final están quietos, si los unimos la onda es circular).

    \noindent \[\text{VECTOR DE ONDA}\hspace{0.2cm} \rightarrow \hspace{1cm} k=
    \dfrac{2\pi}{\lambda}<   \hspace{1cm}\vec{k}=[\vec{k}]=\dfrac{2\pi}
    {\lambda}\]

\clearpage

  \section{Ondas Clásicas}
    \noindent Vamos a trabajar con una onda sinusoidal de longitud de onda $\lambda$, 
    periodo $T$ y fase inicial en una única dimensión. Las expresiones que describen 
    este movimiento son las siguientes:

    \begin{equation}
      y(x,t)=y_0\cos\left(kx\pm \omega t+\varphi_0\right) \hspace{0.4cm} 
      \text{CONVENIO DE SIGNOS: <\ 0 mueve hacia la derecha}
    \end{equation}

    \vspace{0.15cm}
    \teorema{}{
      \def\arraystretch{2}
      \begin{tabular}{lll}
        &Número de onda: &$k=\dfrac{2\pi}{\lambda}$\\
        &Frecuencia angular: &$\omega=2\pi\nu=\dfrac{2\pi}{T}$\\
        &Velocidad/fase de propagación: &$v_f=\lambda\nu=\dfrac{\omega}{k}$
      \end{tabular}
    }

    \noindent Se puede comprobar que la expresión dada cumple la ecuación de onda, 
    lo cual una vez aplicado De Broglie implica que el momento $p$ está bien definido:

    \begin{equation}
      \text{ECUACIÓN DE ONDA}\hspace{1cm} \dfrac{\partial^2y}{\partial x^2}=\dfrac{1}{v_f^2}\cdot
      \dfrac{\partial y^2}{\partial t^2}
    \end{equation}
    
    \noindent Aquí hallamos el principio de incertidumbre de Heisemberg, puesto que 
    conocemos su momento lineal $p$ pero no su posición, ergo la onda está 
    deslocalizada.

    \definicion{Onda plana}{
      Llamamos onda plana a aquella onda cuyo frente de onda se ve en un plano.

      \begin{wrapfigure}[9]{l}{.23\textwidth}
        \includegraphics[width=\textwidth]{fotos/ondaPlana.png}
      \end{wrapfigure}
      $$y(x,y)=a_0e^{i(kx\pm\omega t)}\hspace{1cm}a_0=y_0e^{il_0}$$
      Está expresada en forma compleja (notación que usaremos de ahora en adelante,
      puesto que la parte cqmpleja cobra sentido físico en la mecánica cuántica), y 
      para obtener la onda clásica nos quedamos solo con la parte real:
      $$Re\left[y(x,y)\right]=Re\left[y_0 e^{kx\pm\omega t+l_0}\right]=y_0\cos
      (kx\pm\omega t+l_0)$$
    }
    \teorema{Principio de superposición}{
      Si $y_1$ e $y_2$ son soluciones de la ecuación de onda, una combinación lineal
      de ellas también lo será.
    }
    \definicion{Ondas estacionarias}{
      \begin{wrapfigure}[5]{l}{.23\textwidth}
        \vspace{-0.5cm}
        \includegraphics[width=\textwidth]{fotos/nodos.png}
      \end{wrapfigure}

      Se llama onnda estacionaria a aquella onda cuyos nodos permanecen inmóviles.
      Llamamos nodos a los puntos de la onda cruzan el eje de abcisas y son por tanto
      estáticos en las ondas estacionarias.
    } 
    \teorema{Interferencia de una onda consigo misma}{
      \begin{itemize}
        \item Hacia la derecha:$\hspace{0.35cm}y_1=y_0\sin(kx-\omega t)$
        \item Hacia la izquierda: $y_2=y_0\sin(kx+\omega t)$
      \end{itemize}
    }

\clearpage

    \noindent Para ver la onda resultante de la interferencia de ambas, las sumamos:
    $y=y_1+y_2=y_0\left[\sin(kx-\omega t)+\sin(kx+\omega t)\right]=2y_0\sin\left(
    \dfrac{kx-\omega t+kx+\omega t}{2}\right)\cdot\left(\dfrac{kx-\omega t-kx+\omega 
    t}{2}\right)\Leftrightarrow \boxed{y=2y_0\sin(kx)\cos(\omega t)}$\\ 
    \vspace{0.8cm} Sabiendo que $\sin\alpha+\sin\beta=2\sin\left(\dfrac{\alpha+\beta}{2}
    \right)\cdot\left(\dfrac{\alpha-\beta}{2}\right)$

    \noindent Ahora, aplicamos las condiciones de contorno:
    $t=0, y(x=L) \Longleftrightarrow \boxed{k\cdot L = n\pi\hspace{0.6cm}n=0,1,2...}$\\

    \noindent Donde cada valor de $n$ se le llama \textbf{armónico}. Esta relación 
    puede cambiar si se dan condiciones de contorno distintas.\\
      
    \noindent Como $k=\dfrac{2\pi}{\lambda}$ y $\dfrac{2\pi}{\lambda}\cdot L=n\pi$
    entonces, $\boxed{\lambda=\dfrac{2\pi}{n}}$
    
    \vspace{0.4cm}
    \definicion{Paquete de ondas}{
      Llamamos \textbf{paquete de ondas} a un conjunto de ondas sinusoidales de igual 
      amplitud pero distinta longitud de onda y frecuencia.

      $$\begin{rcases}
        &y_1=A\cos(k_1x-\omega_1 t)\\
        &y_2=A\cos(k_2x-\omega_2 t)
      \end{rcases}$$

      \begin{align}
        \begin{split}
          y(x,y)=y_1+y_2=A\left[\cos(k_1x-\omega_1 t)+\cos(k_2x -\omega_2 t)\right]=\\
          2A\cos\left(\dfrac{k_1+k_2}{2}x-\dfrac{\omega_1+\omega_2}{2}t\right)\cos\left(
          \dfrac{k_1-k_2}{2}x-\dfrac{\omega_1-\omega_2}{2}t\right)
          \label{eq:paqueteDeOndas}
        \end{split}
      \end{align}

      \noindent Siendo los \textbf{valores medios}: $\langle k\rangle=\dfrac{k_1+k_2}
      {2},\hspace{0.6cm}\langle w\rangle=\dfrac{\omega_1+\omega_2}{2}$

      \vspace{0.4cm} \noindent Cada onda se mueve con velocidad $v_1 = \dfrac{\omega_1}{k_1}$ y 
      $v_2 = \dfrac{\omega_2}{k_2}$ respectivamente. Además, podemos calcular la 
      velocidad de fase con los valoores medios que acabamos de definir:
      \[v_f=\dfrac{\langle\omega\rangle}{\langle k\rangle}\]

      \noindent También podemos calcular la velocidad de grupo, que es la velocidad a
      la que se mueve el paquete de ondas:
      \[v_g=\dfrac{\Delta\omega}{\Delta k}\]

      \noindent Cuando hacemos una superposición sí que podemos localizar la partícula,
      pero perdemos información sobre su momento. Esto nos vuelve a llevar otra vez
      al principio de incertidumbre de Heisemberg.
    }

\clearpage

    \ejemplo{Consideramos las siguientes tres ondas y calculamos la superposición de su
    interferencia (sumándolas)}{
      \begin{equation*}
        \begin{rcases}
          \psi_1(x)=\psi_0 e^{ik_0x}\\
          \psi_2(x)=\frac12\psi_0e^{i\left(k_0-\frac{\Delta k}{2}\right)x}\\
          \psi_3(x)=\frac12\psi_0e^{i\left(k_0+\frac{\Delta k}{2}\right)x}
        \end{rcases}
        \psi(x)=\psi_0e^{ix_0x}\left(1+\frac12 e^{-i\frac{\Delta k}{2}x}+
        \frac12 e^{i\frac{\Delta k}{2}x}\right) \Rightarrow 
        \begin{rcases}
          e^{i\theta}=\cos\theta + i\sin\theta\\
          e^{-i\theta}=\cos\theta - i\sin\theta
        \end{rcases}
        2\cos\theta =e^{i\theta}+e^{-i\theta}
      \end{equation*}

      \[\boxed{\psi(x)=\psi_0e^{ik_0x}\left(1+\cos\left(\dfrac{\Delta k}{2}x\right)
      \right)}\]

      \noindent En los extremos tenemos que la función se anula y en el centro, que 
      tiene un máximo. Se dice que la onda está localizada y tiene anchura $\Delta x$.

      \noindent La representación de una partícula se puede dar como un paquete de
      ondas, donde se sabe que la partícula está en $\Delta x$ (relación con 
      Schrödinger porue perdemos informacińo sobre el momento).

      \noindent Podemos construir un paquete de ondas sumando tantas ondas como queramos.
      Una suma infinita de ondas planas con distintios valores de número de onda o de 
      momento se denomina \textit{Serie de Fourier}.
    }
    \definicion{Series de Fourier}{
      Cualquier función periódica se puede escribir como suma de senos y cosenos.
      \begin{equation}
        \psi(x)=\sum_{n=1}^{\infty}A_ne^{ik_n x}
        \label{eq:SerieFourier}
      \end{equation}
      Donde $A_n$ son las amplitudos de las ondas, el peso que cada onda tiene 
      en la función.
    }
    \noindent Vamos a ver que $k$ puede tomar cualquier valor continuo no discreto:
    \begin{equation}
      k\text{ continua}\implies \boxed{\psi(x)=\dfrac{1}{\sqrt{2\pi}}\int g(k)e^
      {ikx}\ dk}
      \label{eq:SerieFourierContinuo}
    \end{equation}
    \noindent $g(k)$ tiene el mismo papel que los coeficientes $A_n$ en el 
    caso discreto. $g(k)$ es la \textit{transformada de Fourier} de $\psi(x)$.
    \teorema{Plancherel}{
      \begin{equation}
        g(k)=\dfrac{1}{\sqrt{2\pi}}\int_{-\infty}^{+\infty}\psi(x)e^{-ikx}\ dx
        \label{eq:TeoremaPlancehrel}
      \end{equation}
      El valor de k estará relacionado con la energía de la partícula, será continua
      en el caso de una partícula libre (puesto que no tiene ningún tipo de 
      restricción). Si quiero hacer un paquete de onda para esa partícula libre, se 
      ha de hacer una integral.
    }
    \noindent Hemos definido dos velocidades distintas para los paquetes de ondas, 
    una velocidad de fase $v_f$ (ondas dentro del paquete) y una de grupo $v_g$ (de 
    todo el paquete). ¿Cómo las calculamos si tenemos un caso donde $k$ es continuo?

    \begin{itemize}
      \item Velocidad de grupo: $v_g=\dfrac{\Delta\omega}{\Delta k}\xrightarrow{k 
            \text{ continua}}v_g=\dfrac{d\omega}{dk}$\\ 
            
            \vspace{-0.2cm}A esta relación entre $\omega$ y $k$ se le llama \textbf{Relación 
            de dispersión}.
      \item Relación entre la velocidad de grupo y la velocidad de fase:\\
            $v_f=\dfrac{\omega}{k},\hspace{0.6cm}\omega=v_f\cdot k\Rightarrow v_g=
            \dfrac{d\omega}{dk}=\dfrac{d(v_f\cdot k)}{dk}=v_f+k\cdot\dfrac{dv_f}{dk}
            \Rightarrow\boxed{v_g=v_f+k\cdot\dfrac{dv_f}{dk}}$\\

            \vspace{-0.2cm}Si $v_f$ no varía, la onda no cambia de forma, pero si varía, se le  
            llamará \textbf{Onda Dispersiva}.
    \end{itemize}
  \section{Difriacción, doble rendija de Young}
    \begin{wrapfigure}[7]{l}{.3\textwidth}
      \vspace{-0.6cm}
      \includegraphics[width=\textwidth]{fotos/difracción.png}
    \end{wrapfigure}
    \noindent Dos rendijas separadas una distancia $d$ y una pantalla a una distancia $D$ tal
    que $D\textgreater\textgreater d$. Cuando se llega a la doble rendija, se producen nuevas ondas que interfieren
    entre ellas. Se espera encontrar un patrón de interferencia. Para que se produzca un 
    máximo, la diferencia de caminos debe ser un múltiplo de la longitud de onda $\lambda$.
    \[\Delta x=x_2-x_1=n\lambda\hspace{1cm}n\in\bbZ\]

    \noindent Vamos a ver la distancia entre dos máximos $\Delta y$ en función de algún dato
    que nos sea conocido.

    \noindent $\Delta x = d\cdot\sin\theta\Rightarrow$ Condición máximo: $\boxed{d\cdot\sin
    \theta=n\lambda}$ \hspace{1.5cm} $\dfrac{\Delta y}{D}=\tan\theta\simeq\sin\theta,\hspace{1cm}
    \boxed{\Delta y=\dfrac{n\lambda D}{d}}$\\

    \vspace{0.4cm}\noindent Vamos a estudiar cómo se presenta el patrón de interferencia:
    \begin{align}
      \begin{split}
        \text{Intensidad }\propto \abs{\psi}^2\Rightarrow\\
        \begin{rcases}
          \psi_1=\psi_0e^{i(kx_1-\omega t)}\\
          \psi_2=\psi_0e^{i(kx_2-\omega t)}
        \end{rcases}
        \Rightarrow\abs{\psi}^2=(\psi_1+\psi_2)\cdot(\psi_1+\psi_2)=\psi_1\cdot\psi_1+\psi_2\cdot
        \psi_2+\psi_1\cdot\psi_2+\psi_2\cdot\psi_1=\\ \abs{\psi_0}^2\left(e^{-i(kx_1-\omega t)}\cdot
        e^{i(kx_1-\omega t)}+e^{-i(kx_2-\omega t)}\cdot e^{i(kx_2-\omega t)}+e^{-i(kx_1-\omega t)}
        \cdot e^{i(kx_2-\omega t)}+e^{-i(kx_2-\omega t)}\cdot e^{i(kx_1-\omega t)}\right)=\\ \abs{
        \psi_0}^2e^{-i\omega t}\cdot e^{i\omega t}\left(2+e^{ik}(x_2-x_1)+e^{ik(x_1-x_2)}\right)=\\
        \abs{\psi_0}^2\left(2+e^{ik\Delta x}+e^{-ik\Delta x}\right)=\\ \abs{\psi_0}^2\left(
        2+2\cos(k\Delta x)\right)=\\ \abs{\psi_0}^2\cdot 4\cos^2\left(\dfrac{k\Delta x}{2}\right)
      \end{split}
    \end{align}
    \noindent Sabiendo que por Euler $e^{ik\Delta x}+e^{-ik\Delta x}=2\cos(k\Delta x)$ y que 
    $\cos(2\theta)+1=2\cos^2\theta$

    \noindent En conclusión, la función que se observa en el patrón:
    \begin{equation}
      \boxed{\abs{\psi_0}^2\cdot4\cos^2\left(\dfrac{k\Delta x}{2}\right)}
      \label{eq:PatronInterferencia}
    \end{equation}

    \noindent ¿Por qué aparece este patrón de interferencia y no solo partículas por donde pasan
    los electrones? Por los \textbf{términos de interferencia}, que no se cancelan (si lo 
    hicieran, tendríamos únicamente dos bandas por donde pasan los rayos).
\chapter{La ecuación de Schrödinger I}
\chapter{La ecuación de Schrödinger II}

\end{document}