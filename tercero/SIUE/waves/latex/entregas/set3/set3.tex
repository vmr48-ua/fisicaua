\documentclass{report}
\usepackage[english]{babel}

\input{preamble}
\input{macros}
\input{letterfonts}

\title{\Huge{Waves\\Homework Set 2}}
\author{Víctor Mira Ramírez}
\date{\number\month -\number\day -\number\year}

\begin{document}

\maketitle
\clearpage

%%%%%%%%%%%%%%%%%%%%%%%%%%%%%%%%%%%%%%%%%%%%%%%%%%%%%%%%%%%%%%%%%%%%%%%%%%%%%
\pregunta{Consider the oscillation of a pendulum: The diagram illustrates that the potential energy of the pendulum bob is given by $U(\theta)=mgL(1-\cos\theta)$, where $U(\theta)=0$ at the bottom of the swing. The kinetic energy is $K=\nicefrac{1}{2}mv^2$, where $v=L\dot{\theta}$.
  \begin{enumerate}[label=\alph*.]
    \item Use the full energy method to derive the exact governing differential equation $\ddot{\theta}=-\dfrac{g}{L}\sin\theta$. Show all steps of the necessary logic/math.
    \item In the potential energy function $U(\theta)=mgL(1-\cos\theta)$, make a Maclaurin expansion of $\cos\theta$ out to second order in $\theta$ so that the total energy function has one term quadratic in $\dot{\theta}$ and one term quadratic in $\theta$. Then write the new approximate expression for the total energy.
    \item Use the full energy method to derive the approximate governing differential equation $\ddot{\theta}=-\dfrac{g}{L}\theta$. As you know, that approximate governing differential equation has exact simple harmonic motion as its general solution.
  \end{enumerate}}
%%%%%%%%%%%%%%%%%%%%%%%%%%%%%%%%%%%%%%%%%%%%%%%%%%%%%%%%%%%%%%%%%%%%%%%%%%%%%
\pregunta{Consider the funciton $f(x)=(a+x)^2$, where $a$ is a constant, where $x<a$ and where $x$ and $a$ have the same units.\\

\textbf{Make a binomial expansion of $f(x)=(a+x)^2$}}
%%%%%%%%%%%%%%%%%%%%%%%%%%%%%%%%%%%%%%%%%%%%%%%%%%%%%%%%%%%%%%%%%%%%%%%%%%%%%
\pregunta{}
%%%%%%%%%%%%%%%%%%%%%%%%%%%%%%%%%%%%%%%%%%%%%%%%%%%%%%%%%%%%%%%%%%%%%%%%%%%%%
\pregunta{}

\end{document}
