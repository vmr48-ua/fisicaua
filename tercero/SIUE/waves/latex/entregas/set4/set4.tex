\documentclass{report}
\usepackage[english]{babel}

\input{preamble}
\input{macros}
\input{letterfonts}

\title{\Huge{Waves\\Homework Set 4}}
\author{Víctor Mira Ramírez}
\date{\number\month -\number\day -\number\year}

\begin{document}

\maketitle
\clearpage

%%%%%%%%%%%%%%%%%%%%%%%%%%%%%%%%%%%%%%%%%%%%%%%%%%%%%%%%%%%%%%%%%%%%%%%%%%%%%
\pregunta{Text, exercise 6.4, pages 6.23 and 6.24

    \begin{enumerate}[label=\alph*.]
        \item \textit{Derive} the governing differential equations for the “arbitrary configuration” (i.e., not for a normal mode). Show all steps of your reasoning.\\

            \underline{\textit{Hint}}: Do that by finding, in the arbitrary configuration, the total force exerted by the springs on each mass.
            Note, e.g., that only the left and center springs exert forces on the left mass
        \item Use the “method of searching for normal coordinates” to derive:
            \begin{enumerate}[label=\roman*.]
                \item The normal coordinates. As part of this, be sure to prove that the coordinates that you so identify are indeed normal coordinates. Show all steps of your reasoning.
                \item The mode frequencies. Show all steps of your reasoning.
            \end{enumerate}
        \item Derive the mode configurations: Using your results from part b, in particular, from the general solution for each of the two normal coordinates as a function of time, derive final results for each of $\psi_a(t)$ and $\psi_b(t)$ as explicit functions of time assuming unspecified initial conditions. Show all steps oof your reasoning.
    \end{enumerate}
    
    }

%%%%%%%%%%%%%%%%%%%%%%%%%%%%%%%%%%%%%%%%%%%%%%%%%%%%%%%%%%%%%%%%%%%%%%%%%%%%%
\pregunta{
    \begin{center}
    \textbf{Normal Modes of Diatomic Molecule - Equal mass case}
    \end{center}

    As we saw in class some time ago, it is possible to predict the frequency for vibrational radiation from
    diatomic molecules by mentally replacing the chemical bond by a spring whose equivalent spring constant
    can, at least roughly, be deduced from only simple considerations. In this problem we will reconsider that
    situation from the point of view of normal modes.\\

    As a first step, we recall our model for the system. For this, refer to the figure below.\\

    Let $\psi_a(t)$ and $\psi_b(t)$ be displacements from equilibrium. Note that both are measured positive to the right,
    though either or both can be displaced to the left of its equilibrium position. Take $m_1 = m_2$.

    \begin{enumerate} [label=\alph*.]
        \item Before getting involved in math: Based on work we did a few weeks ago, what do you expect the
        (nonzero) mode frequency to be? Justify your answer in a few or so, English sentences.
        \item Derive the governing differential equations for the “arbitrary” configuration (\textit{i.e.}, not a normal mode) for
        the system in the ($\psi_a,\psi_b$) coordinate system defined above. Show all steps of your derivation.\\

        If you have doubts that you have arrived at the correct governing differential equations for the “arbitrary”
        configuration, see me.
        \item Now find the mode frequencies using the “method of finding normal coordinates.” \underline{Show all steps of the
        work}. As part of your work/answers, the following must be included:
            \begin{enumerate}[label=\roman*.]
                \item Derive the normal coordinates (in terms of $\psi_a(t)$ and $\psi_b(t)$). Clearly identify them and prove that
                they are indeed normal coordinates.
                \item Show that the method of finding normal coordinates implies that one of the normal modes has “zero
                frequency.” What sort of motion of the molecule does that imply? Why?
                \item From the method of normal coordinates, is the other mode frequency what you expected?
                What is the physical meaning of the higher frequency normal coordinate? (I.e., what kind of motion
                internal to the molecule does it represent?)
            \end{enumerate}
    \end{enumerate}
}
%%%%%%%%%%%%%%%%%%%%%%%%%%%%%%%%%%%%%%%%%%%%%%%%%%%%%%%%%%%%%%%%%%%%%%%%%%%%%
\pregunta{\textbf{Normal Modes of Diatomic Molecule via method of normal coordinates - Unequal mass case}}
%%%%%%%%%%%%%%%%%%%%%%%%%%%%%%%%%%%%%%%%%%%%%%%%%%%%%%%%%%%%%%%%%%%%%%%%%%%%%
\pregunta{}
%%%%%%%%%%%%%%%%%%%%%%%%%%%%%%%%%%%%%%%%%%%%%%%%%%%%%%%%%%%%%%%%%%%%%%%%%%%%%
\pregunta{}
\end{document}