\documentclass{report}
\usepackage[english]{babel}

\input{preamble}
\input{macros}
\input{letterfonts}

\title{\Huge{Waves\\Lecture notes}}
\author{}
\date{\number\year}

\begin{document}

\maketitle
\clearpage
\noindent Apuntes de las clases de \textit{Waves} dadas por \textit{Dave Kaplan} y transcritos a \LaTeX
\hspace{0cm} por \textit{Víctor Mira Ramírez} durante el curso 2023-2024 del grado en Física de la \textit{Southern Illinois University}.
\pagebreak
\tableofcontents
\pagebreak

\chapter{Introduction}
  \noindent Waves is a broad topic in Physics. Info in the universe propagates through waves of all kinds: sound, electromagnetic, hydrodynamic, shock, gravity... etc. The purpose of the course is to discuss the universe:
  \begin{enumerate}
    \item What is in it?
    \item What is it itself made of?
  \end{enumerate}
  \noindent The classical view to view it is to consider the universe as an arena in which objects exits. Although not very accurate with the actual phiysics beliefs.
  \noindent What's left in the universe if we empty it from everything? Space-time, filled with charges, mass.\\

  \noindent Mass is an attribute of matter that is able to interact with the universe and penetrate into the Spacetime, pictorically like a \textit{fabric} that can be curved or bent. So we are stepping into gravity territory. 
  What keeps thte Earth orbiting the Sun? Gravity is not a \textit{force}. The Sun's mass curves the Spacetime, this kind of fabric we are talking about. Cataclysims on the universe may generate changes
  in the Spacetime that propagate to us as gravitational waves.\\

  \noindent Attending the topic of charges, all acceleration in Spacetime generates radiation in the form of electromagnetic waves. That is the source of electromagnetic radiation, all light an hence everything we see are electromagnetic waves.
  We are only able to see a limited slice of the electromagnetic spectrum, which may not represent the whole picture. We can only see accelerating charges perpendicular to the line of sight.\\

  \begin{figure}[h]
    \centering
    \includegraphics[width=0.8\textwidth]{fotos/visible_spectrum.png}
  \end{figure}

  \noindent According to Maxwell equations we know that time-changing electric fields generate magnetic fields in the vacuum and viceversa. The speed of this oscilating wave in vacuum is what we call $c=\frac{1}{\sqrt{\varepsilon_0\mu_0}}$
  \clearpage

  \section{A Basic Review of Simple Harmonic Motion}
    \subsection{Spring}
      \noindent Figure a massless spring. We know that springs exert a force contrary to a movement out of equilibrium. That force will be proportional to $kx$ being $k$ the spring constant. Depending on wether you push or pull on a spring you'll get a positive or a negative force. According to Newton's Second Law:
      \begin{equation}
        \vec{a}(t)=\dfrac{\vec{F}(t)}{m}
      \end{equation}
      \noindent If an object moves at a constant velocity through vacuum, it has no acceleration, but if velocity is not constant, what would that non-zero acceleration be? This question raises Newton's Second Law.\\

      \noindent When we pull from a spring with a mass attached and we leave it free, the mass will accelerate with less and less force until it arrives to the equilibrium point of the spring. At that point the force will be zero, but the mass will continue its trajectory according to the Laws of Conservation of Momentum and Energy,
      but this time the force exerted by the spring will be negative, opposing to the movement and stopping the mass at some point (The same as the start one but from the other side). This movement will continue in an oscillatory manner infinitelly. We have our Harmonic Oscillator.
      \begin{equation}
        \vec{a}(t)=\dfrac{-k}{m}\vec{x}(t)
      \end{equation}
      \noindent In our problem, we are describing a cosine function. A sinusoidal function that has some amplitude equal to the separation we initially took from the spring's equillibrium point. Velocity is the slope of position, then it is it's derivative. We can say the same about the acceleration as we can see:
      \begin{equation}
        x(t)=\cos(t)\hspace{1cm}v(t)=\dfrac{d}{dt}(\cos(t))=-\sin(t)\hspace{1cm}a(t)=\dfrac{d}{dt}(-\sin(t))=-\cos(t)
      \end{equation}
      \noindent But how do we know that is the only possible position funciton? It is not, this one only works for $k=m$ for example:
      \begin{equation}
        x(t)=\cos(2t)\hspace{1cm}v(t)=-2\sin(2t)\hspace{1cm}a(t)=-2^2\cos(2t)
      \end{equation}
      Which only works for $\dfrac{k}{m}=2^2$
      \begin{equation}
        x(t)=\cos(3t)\hspace{1cm}v(t)=-3\sin(3t)\hspace{1cm}a(t)=-3^2\cos(3t)
      \end{equation}
      Which only works for $\dfrac{k}{m}=3^2=9$\\

      \noindent We can derive by innduction that $a(t)=\dfrac{-k}{m}x(t)$ with $\omega=\sqrt{\dfrac{k}{m}}$. Then the solution would be:
      \begin{equation}
        x(t)=\cos(\omega t)
      \end{equation}

      \noindent Now we are going to think about what we call amplitude, a factor that will multiply our position function so that it "survives" the derivative and stays on our velocity and acceleration formulas.
      \begin{equation}
        x(t)=A\cos(\omega t)\hspace{1cm}v(t)=-A\omega\sin(\omega t)\hspace{1cm}a(t)=-A\omega^2\cos(\omega t)=-\omega^2 x(t)
      \end{equation}
      \noindent Which indeed satisfies the problem. We have existence, but unicity? We'll return to it in a few sections.
      \clearpage

      \subsection{What is the Physical meaning of $\omega$?}
        We claim that if $T$ is the period, then
        \begin{equation}
          T=\dfrac{2\pi}{\omega}\Longleftrightarrow\omega=\dfrac{2\pi}{T}
        \end{equation}
        But why did we assume the formula of the period to start off.
        \begin{equation}
          \cos(\omega(t+T))=\cos(\omega t)\Longleftrightarrow\cos(\omega t+\omega T)=\cos(\omega t)\Longleftrightarrow \omega T=2\pi\Longleftrightarrow T=\dfrac{2\pi}{\omega}
        \end{equation}
\end{document}
