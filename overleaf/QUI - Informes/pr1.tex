\section{Introducción}  %y he visto que boris puso que tb resumen
    \noindent En esta práctica prepararemos varias disoluciones de  \ce{NaCl} con agua a partir de una disolución inicial. Se nos proporcionarán las especificaciones de dichas sdisoluciones (es decir, su concentración) de formas diferentes: su molaridad, molalidad o porcentaje en masa.
    \\ \\Primeramente, de estas disoluciones obtendremos valores para sus densidades utilizando métodos distintos explicados en el laboratorio (picnómetro y aerómetro) electromagnético. En una segunda parte determinaremos la concentración de ácido cítrico (\ce{C6H8O7}) de una disolución problema, realizando una valoración de la misma usando un indicador (fenolftaleína (\ce{C20H14O4})) y una base (Hidróxido de sodio(\ce{NaOH})).

\section{Inventario}
\noindent Para esta práctica hemos usado:

\begin{multicols}{2}
    \begin{itemize}
        \item Aerómetro de $30 ml$
        \item Aerómetro de $10 ml$
        \item Picnómetro
        \item Pesa electrónica
        \item Bandeja pequeña
        \item Espátula metálica
        \item Bureta
        \item 2 Pipetas
        \item 2 Propietas
        \item 4 Matraces aforados
    \end{itemize}
\end{multicols}

\begin{figure}[H]
    \centering
    \includegraphics[width=.4\textwidth]{prac1/aerómetros.jpg}\hfill
    \includegraphics[width=.4\textwidth]{prac1/picnómetro.jpg}
    \caption{A la izquierda aerómetros y a la derecha un picnómetro}
    \vspace{-1cm}
\end{figure}

\clearpage

\section{Procedimiento}
\subsection{Preparación de disoluciones}
\noindent En esta parte de la práctica vamos a preparar cuatro disoluciones distintas de \ce{NaCl} cada una de ellas con procedimientos distintos. A la primera de ellas la llamaremos A y será una disolución del 20\% en masa. A partir de esta primera disolución haremos las otras 3. La B será una disolución al 5\%, la C 1 molar y la D 1 molal.\\\\
Una vez prepradas, calcularemos la densidad de la disolución A mediante el uso de un picnómetro y obtenemos una densidad de $1.1422 \si{g/ml}$.\\\\
\noindent El valor de la densidad de la disolución A obtenido con el picnómetro se consiguió gracias esta expresión:
\[\rho = \rho_{\ce{H2O}}\cdot\frac{m_{\text{pic + dis}}-m_{\text{pic}}}{m_{\text{pic + \ce{H2O}}}-m_{\text{pic}}}\]
\noindent Finalmente, medimos la densidad de las cuatro disoluciones obtenidas mediante el método del aerómetro. Para este méetodo, hemos de tener en cuenta que el resultado nos quedará en grados Baumé, los cuales convertiremos en \si{g/ml} con la siguiente fórmula:
\[Be = 145-\frac{145}{\rho}\]
\noindent En la siguiente tabla se recogen las medidas de la densidad realizadas con el aerómetro para cada disolución.\\
\begin{table}[H]
\centering
\begin{tabular}{ccc}
\rowcolor[HTML]{9698ED} 
\textbf{Disolución} & \textbf{Aerómetro} & \textbf{g/ml} \\
\rowcolor[HTML]{DAE8FC} 
\textbf{A} & $18.3$ & $1.444$ \\
\textbf{B} & $5$ & $1.036$ \\
\rowcolor[HTML]{DAE8FC} 
\textbf{C} & $5.6$ & $1.040$ \\
\textbf{D} & $5.5$ & $1.039$
\end{tabular}
\caption{Densidades}
\label{densidades}
\end{table}
\vspace{0.2cm}
\noindent Observamos como el método del picnómetro y el del aerómetro se aproximan notoriamente. Concluimos que usar el aerómetro para medidas aproximadas es más eficiente gracias a una inversión temporal menor en la realización de las medidas.

\clearpage
\subsection{Determinación de la concentración de ácido cítrico}
\noindent En primer lugar, pipeteamos 1 \si{mL} de la disolución problema de ácido cítrico (X) y se vierte en un matraz Erlenmeyer limpio. A continuación, se homogeneiza la bureta con una pequeña cantidad de disolución patrón (P) de \ce{NaOH} 0.10M. Seguidamente, añadimos tres gotas de fenolftaleína en el matraz.\\\\ 
En el momento en el que el líquido del matraz adquiera una coloración diferente (en este caso rosácea) y permanente se cierra por completo la llave de la bureta y anotamos el volumen de la disolución de \ce{NaOH} (P) gastado en el proceso en el siguiente cuadro:\\

\begin{table}[H]
\centering
\begin{tabular}{cc}
\rowcolor[HTML]{9698ED} 
\textbf{Muestra} & \textbf{V(\si{ml})} \\
\rowcolor[HTML]{DAE8FC} 
1 & $10.2$ \\
2 & $10.2$ \\
\rowcolor[HTML]{DAE8FC} 
3 & $10.2$ \\
 &  \\
\rowcolor[HTML]{FCFF2F} 
Media & $10.2$
\end{tabular}
\caption{Valoración}
\label{valoracin}
\end{table}
\vspace{0.4cm}

\noindent Esta es la reacción que está sucediendo en la valoración del ácido cítrico:
\[\ce{C3H4OH(COOH)3 + 3NaOH -> C3H4OH(COONa)3 + 3H2O}\]
\clearpage

\section{Cuestiones}
\vspace{0.4cm}
\noindent\textcolor{BlueViolet}{\textbf{\textit{a) Calcule la concentración de cada una de las disoluciones A, B, C y D en las 10 formas descritas en la introducción y resuma los resultados en una tabla.}}}\\\\
\noindent En esta cuestión se nos pide calcular la concentración de nuestras disoluciones siguiendo los diversos métodos que se exponen en el guión. A partir de los datos obtenidos del \textquote{\textit{CRC Handbook of Chemistry and Physics}} (es decir, que $\rho_{\text{agua}} = 0.9982\ \si{g/ml}$) quedan los cálculos de los diez métodos resumidos en la siguiente tabla:\\

\begin{table}[H]
\renewcommand{\arraystretch}{1.9}
\centering
\newcolumntype{A}{ >{}m{4cm} }
\newcolumntype{B}{ >{\centering\arraybackslash} m{2.4cm} }
\newcolumntype{C}{ >{\centering\arraybackslash} m{1.3cm} }
\resizebox{\textwidth}{!}{
\begin{tabular}{A B C C C C}
 &  & \cellcolor[HTML]{9698ED}\textbf{A} & \cellcolor[HTML]{9698ED}\textbf{B} & \cellcolor[HTML]{9698ED}\textbf{C} & \cellcolor[HTML]{9698ED}\textbf{D} \\
\rowcolor[HTML]{ECF4FF} 
\cellcolor[HTML]{9698ED}\textbf{Molaridad \small{(\si{M})}} & \cellcolor[HTML]{9698ED}\textbf{$\frac{n_{\text{soluto}}}{\si{L}_{\text{disolución}}}$} & $4$  & $0.5$ & $1$ &  $0.9$\\
\cellcolor[HTML]{DAE8FC}\textbf{Molalidad \small{(\si{m})}} & \cellcolor[HTML]{DAE8FC}\textbf{$\frac{n_{\text{soluto}}}{\si{kg}_{\text{disolvente}}}$} & $4.3$ & $0.6$ & $1$ & $1$ \\
\rowcolor[HTML]{ECF4FF} 
\cellcolor[HTML]{9698ED}\textbf{Fracción molar} & \cellcolor[HTML]{9698ED}\textbf{$\frac{n_{\text{soluto}}}{n_{\text{totales}}}$} & $0.074$ & $0.0015$ & $0.0181$ & $0.016$ \\
\cellcolor[HTML]{DAE8FC}\textbf{Razón molar} & \cellcolor[HTML]{DAE8FC}\textbf{$\frac{n_{\text{soluto}}}{n_{\text{disolvente}}}$} & $0.08$ &  $0.015$ &  $0.0184$ & $0.017$ \\
\rowcolor[HTML]{ECF4FF} 
\cellcolor[HTML]{9698ED}\textbf{Fracción másica} & \cellcolor[HTML]{9698ED}\textbf{$\frac{\si{g}_{\text{soluto}}}{\si{g}_{\text{disolución}}}$} & $0.2$ & $0.05$ & $0.06$ & $0.05$ \\
\cellcolor[HTML]{DAE8FC}\textbf{Razón másica} & \cellcolor[HTML]{DAE8FC}\textbf{$\frac{\si{g}_{\text{soluto}}}{\si{g}_{\text{disolvente}}}$} & $0.25$ &$0.53$ & $0.06$ & $0.059$ \\
\rowcolor[HTML]{ECF4FF} 
\cellcolor[HTML]{9698ED}\textbf{Porcentaje en peso} & \cellcolor[HTML]{9698ED}\textbf{$\frac{\si{g}_{\text{soluto}}}{\si{g}_{\text{disolución}}} \cdot 100$} & $20\%$ & $5\%$ & $6\%$ & $5\%$ \\
\cellcolor[HTML]{DAE8FC}\textbf{Gramos por litro \small{(\si{g/L})}} & \cellcolor[HTML]{DAE8FC}\textbf{$\frac{\si{g}_{\text{soluto}}}{\si{L}_{\text{disolución}}}$} & $250$ & $52$ & $58$ & $0.058$ \\
\rowcolor[HTML]{ECF4FF} 
\cellcolor[HTML]{9698ED}\textbf{Partes por millón} & \cellcolor[HTML]{9698ED}\textbf{$\frac{\si{g}_{\text{soluto}}}{\si{g}_{\text{disolución}}} \cdot 10^6$} & $2\cdot10^5$ & $5\cdot10^4$ & $6\cdot10^4$ & $4.2\cdot10^4$ \\
\cellcolor[HTML]{DAE8FC}\textbf{Partes por billón} & \cellcolor[HTML]{DAE8FC}\textbf{$\frac{\si{g}_{\text{soluto}}}{\si{g}_{\text{disolución}}} \cdot 10^9$} & $2\cdot10^8$ & $5\cdot10^7$ & $6\cdot10^7$ & $5.2\cdot10^7$
\end{tabular}
}
\end{table}
\vspace*{-2cm}

\clearpage

\noindent\textcolor{BlueViolet}{\textbf{\textit{b) Se quiere preparar una disolución de carbonato sódico al 8\% en peso, partiendo de la sal decahidratada. Calcule las cantidades de sal hidratada y de agua que son necesarias.}}}\\

\noindent Primero ajustamos la ecuación que se nos pide:
\[\ce{20H2O\cdot NaCl + H2CO3 -> Na2CO3 + 2HCl + 20H2O}\]
\noindent Suponiendo que se generan 100\si{g} de \ce{Na2CO3} impuros, obtenemos la formación de 8\si{g} de \ce{Na2CO3} puro. Con el siguiente factor de conversión concluimos los moles necesarios para la obtención de 8g puros:
\[ 8\si{g}\ \ce{Na2CO3} \cdot \frac{1\ce{mol}\ \ce{Na2CO3}}{2\cdot23\si{g}\ \ce{Na}\ +\ 3\cdot16\si{g}\ \ce{O}} \cdot \frac{2\si{mol}\ \ce{H2O}\cdot\ce{NaCl}}{1\si{mol}\ \ce{Na2CO3}} = 0.151\ \si{mol}\ \ce{H2O}\cdot\ce{NaCl}\]\\\\\\
\noindent\textcolor{BlueViolet}{\textbf{\textit{c) Se quiere preparar 100 mL de una disolución de hidróxido sódico 0,2 M. Calcule la cantidad de hidróxido sódico y de agua que son necesaria para preparar la disolución.}}}\\

\noindent En esta cuestión se nos pide calcular la cantidad de hidróxido de sodio (\ce{NaOH}) y agua (\ce{H2O}) necesarios para obtener una disolución $0.2 M$. A partir de la fórmula de la cuestión 1 ($M = \frac{n_{soluto}}{\si{L}_{disolución}}$) haremos el siguiente factor de conversión y obtendremos la respuesta:

\[100\si{ml} \ \ce{NaOH} \cdot \frac{1\si{L}}{1000\si{ml}} \cdot \frac{0.2 \si{mol} \ \ce{NaOH}}{1 \si{L}\ \text{disolución}} = 0.02 \si{mol} \ \ce{NaOH}\]\\\\\\
\noindent\textcolor{BlueViolet}{\textbf{\textit{d) Calcule, a partir del resultado de la valoración la concentración de ácido cítrico problema.}}}\\

\noindent Primero, como sabemos la molaridad de la disolución que tenemos de NaOH,
calculamos los moles que tenemos que serán 0,0011 moles de NaOH y como por
estequiometria de la reacción que se produce, sabemos que la relación es 3 mol
NaOH/mol C6H8O7, llegamos a que tenemos 0,0033 moles de C6H8O7 y por tanto la
molaridad de la disolución que tenemos es 0,33 M.