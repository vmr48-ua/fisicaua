\documentclass{report}
\usepackage[spanish]{babel}
\addto\captionsspanish{
  \renewcommand{\contentsname}%
    {Índice}%
}

\input{preamble}
\input{macros}
\input{letterfonts}

\title{\Huge{Análisis de\\ varias variables 2}}
\author{\huge{Víctor Mira Ramírez}}
\date{\today}

\begin{document}

\maketitle
\newpage
\tableofcontents
\pagebreak

\chapter{Formas diferenciales}
  \section*{Introducción}
    \noindent Vamos a ver en este capítulo un concupto muy importante en matemáticas, que son las formas diferenciales. Como hicimos en \textit{Análisis de varias variables 1}, empezaremos hablando de las formas diferenciales en 2 dimensiones, e iremos poco a poco ampliando el campo de nuestro estudio a dimensiones superiores.
  \section{Definiciones básicas}
    \definicion{Diferencial}{
      Sea $f$ una función definida en un entorno del punto $M_0 = \left(x_0,y_0\right)$ tal que $f$ admite derivadas parciales en un entorno $\mcV\left(M_0\right)$. Se llama \textbf{diferencial de $f$ en $M_0$} a la aplicación lineal definida de $\bbR^2$ en $\bbR$ denotado:
      \begin{align*}
        L\left(x_0,y_0\right) \colon &\bbR\times\bbR \to \bbR\\
        &\phantomarrow{\bbR\times\bbR}{\left(h,k\right)} h \frac{\partial h}{\partial x}\left(x_0,y_0\right) + k \frac{\partial k}{\partial y}\left(x_0,y_0\right)
      \end{align*}
      }
    \comentario{
      Para el caso particular $g\left(x,y\right)=x$, entonces $\frac{\partial g}{\partial x}\left(x,y\right)=1$ y $\frac{\partial g}{\partial y}\left(x,y\right)=0$ lo que nos da $dg\left(x,y\right)=dx\left(h,k\right)=h$
      $\left(\text{rotación }dx=h\right)$\ \ \ \ \ \ \ \ \ $g\left(x,y\right)=y$, entonces $\frac{\partial g}{\partial x}\left(x,y\right)=0$ y $\frac{\partial g}{\partial y}\left(x,y\right)=1$ lo que nos da $dg\left(x,y\right)=dy\left(h,k\right)=k$
      Esto nos da: $df\left(dx,dy\right)=\frac{\partial f}{\partial x}\left(x,y\right)dx+\frac{\partial f}{\partial y}\left(x,y\right)dy$
      \begin{itemize}
        \item Escribir $df\left(dx,dy\right)$ es un abuso de notación que debe evitarse, escribiendo en su lugar $df$ únicamente.
        \item La diferencial puede existir sin que la función sea diferenciable \\($f \in \mathcal{C}_1 \implies f \text{ diferenciable}$, pero $f \text{ diferenciable} \notimplies f \in \mathcal{C}_1$)
      \end{itemize}
    }
    \comentario{
      Sea $f \in \mathcal{C}_1\left(\mathcal{V}\left(M_0\right)\right)$ una función de una variable,\\
      \[f'\left(x_0+h\right)=f\left(x_0\right)+f'\left(x_0\right)h + o\left(h\right) \Longleftrightarrow f\left(x_0+h\right)=f\left(x_0\right)+df_{x_0}\left(h\right) + o\left(h\right)\]\\
      Como tenemos que $h = dx$,$df_{x_0}=f'\left(x_0\right)dx \Longleftrightarrow f'\left(x_0\right)=\frac{df_{x_0}}{dx}
      \left(\frac{dx}{dt}=x'\left(t\right)\Longleftrightarrow dx = x'\left(t\right)dt\right)$
    }
    \definicion{1-forma diferencial}{
      Sea $\mathcal{U}$ un abierto de $\bbR$ y sea $\left(h,k\right) \in \mathcal{U}$ se llama \textbf{1-Forma diferencial} $w$ a la aplicación
      \begin{align*}
        w \colon &\bbR^2 \longrightarrow \bbR\\
        & \left(h,k\right) \longmapsto w=P\left(x,y\right)h + Q\left(x,y\right)k
      \end{align*}
      donde $P$ y $Q$ son funciones "suficientemente regulares".
    }
    \comentario{
      Si $P$ y $Q$ son derivadas parciales de $f$, entonces la 1-forma diferencial w es la diferencial de $f$
    }
    \definicion{Producto exterior}{
      Sean $dx$ y $dy$ dos 1-formas diferenciales, se llama \textbf{producto exterior}
      a la aplicación definida por $dx\wedge dy$ ("$dx$ exterior $dy$"):
      \begin{align*}
        w \colon &\bbR^2\times\bbR^2 \longrightarrow \bbR\\
        & \left(h_1,k_1\right)\times\left(h_2,k_2\right) \longmapsto h_1k_2 - h_2k_1
      \end{align*}
    }
    \comentario{Un determinante es el producto exterior de dos 1-formas diferenciales}
    \teorema{}{
      \begin{itemize}
        \item $dx\wedge dx = 0$
        \item $dx\wedge dy = -dy\wedge dx$
      \end{itemize}
    }
    \noindent A consecuencia de este teorema se obtiene la llamada "\textit{Regla de Sarrus}", cuya demostración es inmediata.
    \definicion{2-forma diferencial}{
      Sea $\mathcal{U}$ un abierto de $\bbR^2$ y sea $\left(x,y\right) \in \mathcal{U}$, se llama \textbf{2-forma diferencial} $w$ a la aplicación
      \begin{align*}
        w \colon &\bbR^2 \longrightarrow \mathcal{B}_i\left(\bbR^2\times\bbR^2,\bbR\right)\\
        & \left(x,y\right) \longmapsto f\left(x,y\right)dx\wedge dy
      \end{align*}
      donde $\mathcal{B}_i\left(\bbR^2\times\bbR^2,\bbR\right)$ es el conjunto de las apicaciones bilineales.
    }
    \comentario{
      $\forall H_1\left(h_1,k_1\right),H_1'\left(h_1',k_1'\right),H_2\left(h_2,k_2\right)$ y sea $\lambda \in \bbR \colon$\\
      $dx\wedge dy\left(H_1+\lambda H_1',H_2\right)=dx\wedge dy\left(\left(h_1+\lambda h_1',k_1+\lambda k_1'\right),\left(h_2,k_2\right)\right)=dx\wedge dyk_2\left(h_1+\lambda h_1'\right)-dx\wedge dyh_2\left(k_1+\lambda k_1'\right)=$\\
      $dx\wedge dy\left(h_1k_2-h_2k_1\right)+\lambda dx\wedge dy\left(h_1'k_2 - h_2k_1'\right)=dx\wedge dy\left(H_1,H_2\right)+\lambda dx\wedge dy \left(H_1',H_2\right) \implies$ \\$dx\wedge dy$ es lineal respecto a la primera variable. Idem con la segunda.\\
      Esto implica que $dx\wedge dy$ es bilineal, y lo es tambien $f\left(x,y\right)dx\wedge dy$.\\ En conclusión, $f\left(x,y\right)dx\wedge dy \in \mathcal{B}_i\left(\bbR^2\times \bbR^2,\bbR\right)$
    }
    \definicion{Derivada exterior}{
      Sea $w$ una 1-forma diferencial definida en $\mathcal{U}$ abierto de $\bbR^2$. Se llama derivada exterior de $w$ y la denotamos $dw$ a la 2-forma diferencial definida por:
      $$dw = \left(\frac{\partial Q}{\partial x}\left(x,y\right)-\frac{\partial P}{\partial y}\left(x,y\right)\right)dx\wedge dy$$
    }
    \vspace{1cm}
  \section{Formas diferenciales cerradas y exactas}
    \definicion{Forma diferencial cerrada}{
      Sea $w$ una 1-forma diferencial de clase $\mathcal{C}_1$ en un abierto $\mathcal{U}$ de $\bbR^2$ se dice que $w$ es cerrada si $dw = 0$
    }
    \comentario{
      Tenemos $w = P\left(x,y\right)dx + Q\left(x,y\right)dy$, si $w \in \mathcal{C}_1$, las derivadas parciales de $P$ y $Q$ existen, y tenemos\\
      $dw = \left(\frac{\partial Q}{\partial x}-\frac{\partial P}{\partial y}\right)dx\wedge dy$
      \ \ \ \ \ \ \ \ \ \ \ \ \ \ \ \ \ \ \ \ \ \ \ \ \ \ \ \ \ \ \ \ \ \ \ \ \ \ \
      Si $w = 0 \Longleftrightarrow \frac{\partial Q}{\partial x}\left(x,y\right) = \frac{\partial P}{\partial y}\left(x,y\right) \forall\left(x,y\right)\in\mathcal{U}$
    }
    \definicion{Forma diferencial exacta}{
      Sea $w$ una 1-forma diferenicial de clase $\mathcal{C}_1$ en un abierto de
      $\mathcal{U}$ de $\bbR^2$, se dice que $w$ es \textbf{exacta} si existe
      una función $f\colon\mathcal{U} \to \bbR$ de clase $\mathcal{C}_1$ tal que
      $df = w$
    }
    \comentario{
      Ya sabemos que si $f\in\mathcal{C}_1\left(\mathcal{U}\right)$, entonces
      $df = \frac{\partial f}{\partial x}dx + \frac{\partial f}{\partial y}dy$.
      (Si $\frac{\partial f}{\partial x} = P \text{ y }
      \frac{\partial f}{\partial y} = Q$ tenemos la estructura de una
      1-forma diferencial). Si $w = df$, tenemos que
      $P=\frac{\partial f}{\partial x}\left(x,y\right)$ y
      $Q=\frac{\partial f}{\partial y}\left(x,y\right)$\\ \\
      Esto significa que para demostrar que $w$ es exacta hay que encontrar $f$
      tal que $df = w$, es decir, buscamos $f\left(x,y\right)$ que sea solución
      del sistema $P=\frac{\partial f}{\partial x}$,
      $Q=\frac{\partial f}{\partial y}$.\\ \\
      Como $f \in\mathcal{C}_1\left(\mathcal{U}\right)$, entonces
      $\frac{\partial f}{\partial x}$ y $\frac{\partial f}{\partial y}$
      son continuas. También sabemos que $w \in\mathcal{C}_1\left(\mathcal{U}
      \right)$, lo que nos dice que $P,Q\in\mathcal{C}_1\left(\mathcal{U}\right)
      \implies f\in\mathcal{C}_2\left(\mathcal{U}\right)$
    }
    \teorema{}{
      Si $w \in \mathcal{C}_1\left(\mathcal{U}\right) \implies
      f\in \mathcal{C}_2\left(\mathcal{U}\right)$
    }
    \comentario{
      Al ser de clase $\mathcal{C}_1$, podemos determinar las derivadas parciales
      de $\frac{\partial f}{\partial x}$ y $\frac{\partial f}{\partial y}$. Tenemos:
      $$\frac{\partial^2 f}{\partial y\partial x}=\frac{\partial P}{\partial y}
        \hspace{2cm}
        \frac{\partial^2 f}{\partial x\partial y}=\frac{\partial Q}{\partial x}$$
      Y con el \textit{Lema de Schwartz} $\left(
        \frac{\partial^2 f}{\partial y\partial x}=
        \frac{\partial^2 f}{\partial x\partial y}\right)$, entonces sabemos que\\
      $$ \frac{\partial P}{\partial y}\left(x,y\right)=\frac{\partial Q}{\partial x}
      \left(x,y\right),\ \forall\left(x,y\right)\in\mathcal{U}$$
    }
    \teorema{}{
      Sea $w$ una 1-forma diferencial de clase $\mathcal{C}_1\left(\mathcal{U}
      \right)$ y exacta en $\mathcal{U}$, entonces $w$ es cerrada.
    }
    \comentario{
      El recíproco del teorema anterior es falso, salvo si damos a $\mathcal{U}$
      una cierta geometría. Por ello, vamos a volver a hablar de topología.
    }
    \definicion{Conjunto estrellado}{
      Sea $\mathcal{U}$ un abierto de $\bbR^2$, se dice que $\mathcal{U}$ es
      estrellado si $\exists a\in\mathcal{U}$ tal que $\forall b\in\mathcal{U}$
      tenemos $\left[a,b\right]\subset \mathcal{U}$
    }
    \ejemplo{}{
      \begin{enumerate}
        \item Todo convexo de $\bbR^2$ es un estrellado.
        \item Si $\mathcal{U}\in\bbR^2\setminus\left\{\left(0,0\right)\right\}$,
        $\mathcal{U}$ no es estrellado.
        \item $\left[a,b\right]\in\bbR^2=\left\{ta + \left(1-t\right)b,t\in\left[
        0,1\right]\right\}$ (Parametrización de un segmento)
      \end{enumerate}
    }
    \teorema{Poincaré}{
      Si $w$ es una 1-forma diferencial de clase $\mathcal{C}_1$ en un abierto
      estrellado de $\mathcal{U}$ de $\bbR^2$ tal que $dw = 0$ $\left(\text{i.e }
      w\text{ es cerrada}\right)$ entonces $w$ es exacta.
    }
    \noindent Una aplicación en física puede ser, por ejemplo, en mecánica de
    fluidos, ya que si en $\left(0,0\right)$ hay un obstáculo y hay un flujo
    alrededor de un cilindro (definido en un entorno de $\left(0,0\right)$)
    entonces no se pueden usar los teoremas de la mecánica de fluidos.\\ \\
    La demostración se hará con integrales de línea (curvilíneas o de camino).
    \ejemplo{Sea $w=x\ dx+y\ dy$ y $\mathcal{U}=\bbR^2$, ¿La forma diferencial
    $w$ es cerrada?}{
      Claro que sí, vamos a demostrarlo. Tenemos $P\left(x,y\right)=x$,
      $Q\left(x,y\right)=y$ entonces $\frac{\partial P}{\partial y}\left(x,y
      \right)=0=\frac{\partial Q}{\partial y}\left(x,y\right)$, lo que implica
      que $dw=0$, es decir, que $w$ es cerrada. \\ \\Además, como $\bbR^2$ es
      estrellado, tenemos gracias al \textit{Teorema de Poincaré} que $w$ es
      exacta.
    }
    \ejemplo{Usando la anterior forma diferencial, encuentra la función $f$
    tal que $df=w$}{
      Tenemos $w=x\ dx+y\ dy = \frac{\partial f}{\partial x}dx +
      \frac{\partial f}{\partial y}dy$, lo que nos da: $$\frac{\partial f}
      {\partial x}=x \hspace{1cm} \frac{\partial f }{\partial y}=y$$ De aquí,
      obtenemos que $f\left(x,y\right)=\frac{x^2}{2}+g\left(y\right)$, y
      derivando respecto a $y\colon$\\$\frac{\partial f}{\partial y}=g'\left(
      y\right)=y \implies f\left(x,y\right)=\frac{1}{2}\left(x^2+y^2\right)+C$
      y tenemos $w=df$.
    }
    \comentario{
      En $\bbR^3$, una 1-forma diferencial $w$ se escribe $w=P\left(x,y,z\right)
      dx + Q\left(x,y,z\right)dy + R\left(x,y,z\right)dz$ donde $P,Q,R$ son tres
      funciones definidas en un abierto $\mathcal{U}$ de $\bbR^3$. Si además
      $\mathcal{U}$ es un estrellado de $\bbR^3$, entonces $w$ es exacta.
      $$\frac{\partial Q}{\partial x}=\frac{\partial P}{\partial y}\hspace{1cm}
        \frac{\partial R}{\partial x}=\frac{\partial P}{\partial z}\hspace{1cm}
        \frac{\partial R}{\partial y}=\frac{\partial Q}{\partial z}$$
    }
    \clearpage
    \teorema{}{
      Sea $w$ una 1-forma diferencial de clase $\mathcal{C}_1\left(\mathcal{U}
      \right)$, donde $\mathcal{U}$ es un abierto de $\bbR^2$, suponemos el
      cambio de variable:$$x=f\left(u,v\right)\hspace{1cm}y=g\left(u,v\right)
      \hspace{1cm}\text{con}f,g\in\mathcal{C}_1$$ $$w=P\left(x,y\right)
      dx+Q\left(x,y\right)dy = P_1\left(u,v\right)du + Q_1\left(u,v\right)dv$$
      $$\text{con } P_1\left(u,v\right)=P\frac{\partial f}{\partial u}+
      Q\frac{\partial f}{\partial u} \hspace{1cm} Q_1\left(u,v\right)=
      P\frac{\partial f}{\partial v}+Q\frac{\partial f}{\partial v}$$
    }
    \ejemplo{}{
      Sea $w=x\ dx+y\ dy$, tenemos el cambio de variable:
      \begin{align*}
        \centering
        \begin{tabular}{c c c}
          $\begin{cases}\begin{aligned}
              x=r\cos{\theta}\\
              y=r\sin\theta
            \end{aligned}\end{cases}$&
          $\implies$&
          $\begin{cases}\begin{aligned}
              dx=\cos\theta\ dr-r\sin\theta\ d\theta\\
              dy=\sin\theta\ dr+r\cos\theta\ d\theta
            \end{aligned}\end{cases}$
        \end{tabular}
      \end{align*}
      y por tanto, $w=d\left(\frac{1}{2}\left(x^2+y^2\right)\right)=d\left(
      \frac{r^2}{2}\right)=r\ dr$
      }
\chapter{Campo de gradiente, divergencia y rotacional}
  \vspace{-0.5cm}
  \section{Campo escalar, campo vectorial}
    \definicion{Campo de gradiente}{
      Sea $\ovl{V}$ un campo vectorial con componentes $P$ y $Q$ de clase
      $\mcC_1$ en $\mcD \subset \bbR^2$. Se dice que $\ovl{V}$ es un campo
      de gradiente si existe un campo escalar $\varphi$ de clase $\mcC_1$ en
      $\mcD$ tal que:
      \begin{align*}
        \begin{tabular}{r l}
          $\varphi\colon \mcD\subset\bbR^2$ &$\longrightarrow \bbR$\\
          $M\left(x,y\right)$ &$\longmapsto \varphi\left(M\right)=\varphi\left(x,y\right)$
        \end{tabular}&
        \hspace{1cm}\ovl{V}=\ovl{grad}\ \varphi=\ovl{\nabla}\varphi&
        \text{i.e.}&
        \ovl{V}=
        \begin{cases}\begin{aligned}
            P\left(x,y\right)=\frac{\partial\varphi}{\partial x\left(x,y\right)}\\
            Q\left(x,y\right)=\frac{\partial\varphi}{\partial y\left(x,y\right)}
        \end{aligned}\end{cases}
      \end{align*}
    }
    \teorema{}{
      Sea $\ovl{V}$ un campo vectorial con componentes $P$ y $Q$ de clase
      $\mcC_1\left(\mcD\right)\subset\bbR^2$:\\ $\ovl{V}$ campo de gradiente
      $\Longleftrightarrow$ la 1-forma diferencial $w=P\left(x,y\right)dx+Q
      \left(x,y\right)dy$ es exacta en $\mcD$, i.e. $w=df$.\\ Como consecuencia,
      sabemos que si $\ovl{V}$ es un campo de gradiente $\implies\forall\left(
      x,y\right)\in\mcD , \frac{\partial P}{\partial y}\left(x,y\right)=
      \frac{\partial Q}{\partial x}\left(x,y\right)$
    }
    \comentario{
      \vspace{-0.2cm}
      \begin{enumerate}
        \item Si $\mcD$ es un abierto estrellado, entonces $\frac{\partial P}{\partial y}
              =\frac{\partial Q}{\partial x}\Longleftrightarrow w \text{ es cerrada }
              \overset{\text{Poincaré}}{\Longleftrightarrow} w\text{ es exacta }
              \Longleftrightarrow \ovl{V}=\ovl{\nabla}\varphi$
        \item Si $\mcD$ es un abierto 'simplemente conexo' de $\bbR^3$,
              entonces $dw=0 \Longleftrightarrow \exists\varphi\in\mcC_1\left(\mcD\right)$
              tal que $d\varphi = w$\\ i.e. $w$ cerrada $\Longleftrightarrow w$ exacta.
              \hspace{3.3cm} $\mcD$ simplemente conexo $\Longleftrightarrow\mcD$ no tiene huecos
      \end{enumerate}
    }
    \teorema{}{
      \begin{itemize}
        \item Los abiertos estrellados son simplemente conexos.
        \item En $\bbR^3$, un campo vectorial $\ovl{V}\left(P,Q,R\right)$ con
              $P,Q,R\in\mcC_1\left(\mcD\right)$ donde $\mcD\in\bbR^3$,
              $$\ovl{V}=\ovl{\nabla}\varphi \Longleftrightarrow w=Pdx+Qdy+Rdz
              \text{ es exacta.}$$
        \item \begin{tabular}{c c c}
                $\begin{cases}\begin{aligned}
                  \text{$\mcD$ estrellado}\\
                  \text{ó}\\
                  \text{$\mcD$ simplemente conexo}
                \end{aligned}\end{cases}$&
                $\Longleftrightarrow$&
                $\begin{cases}\begin{aligned}
                  \frac{\partial R}{\partial y}=\frac{\partial Q}{\partial z}\\
                  \frac{\partial P}{\partial z}=\frac{\partial R}{\partial x}\\
                  \frac{\partial Q}{\partial x}=\frac{\partial P}{\partial y}
                \end{aligned}\end{cases}$
              \end{tabular}
      \end{itemize}
    }
    \definicion{Rotacional}{
      Sea $\ovl{V}\left(P,Q,R\right)$ un campo vectorial de clase
      $\mcC_1\left(\mcD\in\bbR^3\right)$, llamamos rotacional de $\ovl{V}$ a:
      $$\ovl{rot}\ovl{V}=\ovl{\nabla}\times\ovl{V}=\left(
      \frac{\partial R}{\partial y} - \frac{\partial Q}{\partial z},
      \frac{\partial P}{\partial z} - \frac{\partial R}{\partial x},
      \frac{\partial Q}{\partial x} - \frac{\partial P}{\partial y}\right)$$
    }
    \teorema{}{
      Sea $\ovl{V}\left(P,Q,R\right)$ de clase $\mcC_1\left(\mcD\in\bbR^3\right)$,
      si $\ovl{V}$ es un campo de gradiente, entonces \hspace{0.5cm}
      $\ovl{\nabla}\times\ovl{V}=\ovl{0}$
    }
    \comentario{
      Si $\mcD$ es simplemente conexo (sin huecos) entonces el recíproco es
      cierto, $\ovl{rot}\ovl{V}=\ovl{0}\Rightarrow\ovl{V}=\ovl{\nabla}\varphi$
      donde $\varphi\in\mcC_1\left(\mcD\right)$
    }
    \definicion{Divergencia}{
      Sea $\ovl{V}\left(P,Q,R\right)$ un campo vectorial de clase
      $\mcC_1\left(\mcD\in\bbR^3\right)$, llamamos divergencia de $\ovl{V}$ al
      escalar: $$\ovl{div}\ovl{V}=\ovl{\nabla}\cdot\ovl{V}=
        \frac{\partial P}{\partial x} + \frac{\partial Q}{\partial y} +
        \frac{\partial R}{\partial z}$$
    }
    \teorema{}{
      Sea $\ovl{V}\left(P,Q,R\right)$ un campo vectorial de clase
      $\mcC_2\left(\mcD\in\bbR^3\right)$, entonces \hspace{0.5cm}
      $\nabla\cdot\nabla\times\ovl{V}=0$
    }
    \teorema{}{
      Si $\mcD$ es simplemente conexo y $\nabla\cdot\ovl{V}=0 \implies$
      $\ovl{V}=\ovl{\nabla}\times\ovl{A}$ (donde $\ovl{A}$ es un campo
      vectorial definido en $\mcD$)
    }
    \comentario{
      El campo $\ovl{A}$ no es único, de la misma manera que un campo escalar
      $\varphi$ asociado a un campo de vectores.\\Si $\ovl{V} = \ovl{\nabla}
      \varphi = \ovl{\nabla}\left(\varphi+\text{cte}\right)$ De la misma manera,
      tenemos que si $\ovl{A}f=\ovl{A}+\ovl{\nabla}f$, entonces $\ovl{\nabla}
      \times\ovl{A}f=\ovl{\nabla}\times\left(\ovl{A}+\ovl{\nabla}f\right)=
      \ovl{\nabla}\times\ovl{A}+\ovl{\nabla}\times\left(\ovl{\nabla}f\right)$.
      Es decir, si $\ovl{V}=\ovl{\nabla}\times\ovl{A}$, existe $f$ tal que
      $\ovl{V}=\ovl{\nabla}\times\left(\ovl{A}+\ovl{\nabla}f\right)$
    }
    \definicion{Laplaciano Cartesiano}{
      Sea $\varphi$ un campo de clase $\mcC_2$ en $\bbR^2$, llamamos \textbf{
      Laplaciano de $\varphi$} al campo escalar $\Delta\varphi=\frac{\partial^2
      \varphi}{\partial x^2}+\frac{\partial^2\varphi}{\partial y^2}+\frac{\partial^2
      \varphi}{\partial z^2}$
    }
    \comentario{
      Podemos decir que $\Delta\varphi$ se escribe como
      $$\Delta\varphi=\frac{\partial}{\partial x}\left(\frac{\partial\varphi}
      {\partial x}\right)+\frac{\partial}{\partial y}\left(\frac{\partial\varphi}
      {\partial y}\right)+\frac{\partial}{\partial z}\left(\frac{\partial\varphi}
      {\partial z}\right)\Longleftrightarrow\Delta\varphi = \ovl{div}\left(
      \ovl{grad}\varphi\right)=\ovl{\nabla}\cdot\ovl{\nabla}\varphi$$
    }
    \teorema{}{
      Sea $\ovl{V}$ un campo de gradiente tal que $\ovl{\nabla}\cdot\ovl{V}=0$,
      tomando $\varphi$ un campo escalar $\varphi\in\mcC_2\left(\mcD\right)$
      tal que $\ovl{V}=\ovl{\nabla}\varphi$, entonces $\ovl{\nabla}\cdot
      \ovl{\nabla}\varphi=\Delta\varphi=0$
    }
  \section{Aplicaciones a la física}
    \vspace{-0.5cm}
    \begin{align*}
      \begin{tabular}{r l}
        $\varphi\colon \bbR^3$ &$\longrightarrow \bbR$\\
        $\ovl{M}$ &$\longmapsto \varphi\left(M\right)$
      \end{tabular}&
      \hspace{1cm}\text{ý}&
      \hspace{-0.2cm}
      \begin{tabular}{r l}
        $\varphi\colon \bbR^3$ &$\longrightarrow \bbR^3$\\
        $\ovl{M}$ &$\longmapsto \ovl{\varphi\left(M\right)}$
      \end{tabular}&
      \text{donde $\ovl{\varphi\left(M\right)}\ $}\left(
        \begin{tabular}{c}
          $\varphi_x\left(M\right)$\\
          $\varphi_y\left(M\right)$\\
          $\varphi_z\left(M\right)$
        \end{tabular}
      \right)\text{ vectorial}
    \end{align*}
      \subsection*{Operador Nabla (operador diferencial)}
        $\ovl{\nabla}=\left(
        \begin{tabular}{c}
          $\frac{\partial}{\partial x}$\\
          $\frac{\partial}{\partial y}$\\
          $\frac{\partial}{\partial z}$
        \end{tabular}\right)
        \text{ en la base canónica de $\bbR^3$}$
        \subsubsection*{Gradiente}
          $\ovl{\text{grad}}\varphi=\ovl{\nabla}\varphi=\left(
          \begin{tabular}{c}
            $\frac{\partial}{\partial x}$\\
            $\frac{\partial}{\partial y}$\\
            $\frac{\partial}{\partial z}$
          \end{tabular}\right)
          \text{ (Campos escalares)}$
        \subsubsection*{Divergencia}
          $\text{div}\ovl{\varphi}=\ovl{\nabla}\cdot\varphi=
          \frac{\partial\varphi_x}{\partial x} +\frac{\partial\varphi_y}{\partial y}+
          \frac{\partial\varphi_z}{\partial z}
          \text{(Producto escalar)}$
        \subsubsection*{Rotacional}
          $\ovl{\text{rot}}\ovl{\varphi}=\ovl{\nabla}\times\varphi=\left(
          \begin{tabular}{c}
            $\frac{\partial}{\partial x}$\\
            $\frac{\partial}{\partial y}$\\
            $\frac{\partial}{\partial z}$
          \end{tabular}\right)\times\left(
          \begin{tabular}{c}
            $\varphi_x$\\
            $\varphi_y$\\
            $\varphi_z$
          \end{tabular}\right)=\left(
            \begin{tabular}{c}
              $\frac{\partial\varphi_z}{\partial y}-\frac{\partial\varphi_y}{\partial z}$\\
              $\frac{\partial\varphi_x}{\partial z}-\frac{\partial\varphi_z}{\partial x}$\\
              $\frac{\partial\varphi_y}{\partial x}-\frac{\partial\varphi_x}{\partial y}$
            \end{tabular}\right)$
        \subsubsection*{Laplaciano Escalar}
          $\Delta\varphi=\ovl{\nabla}\cdot\ovl{\nabla}\varphi=\ovl{\nabla}
          \ovl{\nabla}\varphi=\ovl{\nabla}^2\varphi$ (ecuación de difusión)
        \subsubsection*{Laplaciano Vectorial}
        $\ovl{\Delta}\ovl{\varphi}=\left(\Delta\varphi_x,\Delta\varphi_y,\Delta
        \varphi_z\right)$\vspace{0.2cm}\\
        (Vector cuyas componentes son Laplacianos). Se usa en mecánica de fluidos,
        (Ecuación de Navier-Stokes), en electromagnetismo (ecuación de d'Alembert)
      \subsection*{Significado del gradiente}
        \subsubsection{Leyes en física}
          \setlength{\tabcolsep}{1.4cm}
          \hspace{-1.3cm}
          \begin{tabular}{l l}
            Ley de Holm local (potencial)      & Ley de Fourier (temperatura)\\
            \multicolumn{1}{c}{$\ovl{f}=-\gamma\ovl{\text{grad}}V$}&
            \multicolumn{1}{c}{$\ovl{f}=-\lambda\ovl{\text{grad}}T$}
          \end{tabular}
        \subsubsection{Relaciones fundamentales}
          \setlength\extrarowheight{8pt}
          \setlength{\tabcolsep}{0.9cm}
          \hspace{-0.8cm}
          \begin{tabular}{ l  l }
            $\ovl{\nabla}(\ovl{A}+\ovl{B})=\ovl{\nabla}A+\ovl{\nabla}B$&
            $\ovl{\nabla}\times(\ovl{A}+\ovl{B})=\ovl{\nabla}\times\ovl{A}
              +\ovl{\nabla}\times\ovl{B}$\\
            $\ovl{\nabla}\cdot\varphi\ovl{A}=\ovl{\nabla}\varphi\cdot\ovl{A}+\varphi
              \ovl{\nabla}\cdot\ovl{A}$&
            $\ovl{\nabla}\times(\varphi\ovl{A})=\ovl{A}\times\ovl{\nabla}\varphi+
              \varphi(\ovl{\nabla}\times\ovl{A})$\\
            $\ovl{\nabla}\cdot(\ovl{A}\times\ovl{B})=\ovl{B}\cdot(\ovl{\nabla}\times
              \ovl{A})-\ovl{A}\cdot(\ovl{\nabla}\times\ovl{B})$&
            $\Delta\ovl{A}=\ovl{\nabla}\ovl{\nabla}\cdot A -
              \ovl{\nabla}\times\ovl{\nabla}\times A\hspace{0.1cm}\Longleftrightarrow
              \hspace{0.1cm}\Delta\ovl{A}=\ovl{\nabla}^2A-\ovl{\nabla}\times
              \ovl{\nabla}\times A$
          \end{tabular}
\chapter{Integrales de línea}
  \vspace{-0.5cm}
  \section{Curvas paramétricas}
    \definicion{Curva paramétrica}{
      Sea $I$ un intervalo de $\bbR$, $f$ y $g$ dos funciones defindas en $I$
      "suficientemente regulares".\\ Sea $C=\left\{M\left(x,y\right)\in\bbR^2
      \colon\exists t\in I,M\left(t\right)=\left(x\left(t\right),y\left(t\right)
      \right)\right\}$.Llamamos a $C$ una \textbf{curva paramétrica} en $\bbR^2$.
      Tenemos $x=f\left(t\right)$, $y=g\left(t\right)$
    }
    \ejemplo{}{
      Tenemos $t\subset I\in\bbR\left(a_1,a_2,b_1,b_2\right)\in\bbR^4\times\bbR^3$
      dados
      \setlength\extrarowheight{0pt}
      $\begin{cases}\begin{aligned}
        x=a_1t+b_1\\
        y=a_2t+b_2
      \end{aligned}\end{cases}$ tenemos $t=\frac{x-b_1}{a_1}$ lo que nos da $y=a_2\left(\frac
      {x-b_1}{a_1}\right)+b_2\Longleftrightarrow y=\frac{a_2}{a_1}x+b_2-\frac
      {a_2b_1}{a_1}$ (ecuación de una recta) i.e. $y = F\left(x\right)$
    }
    \ejemplo{}{
      \setlength\extrarowheight{0pt}
      $\begin{cases}\begin{aligned}
        x=\cos t\\
        y=\sin t
      \end{aligned}\end{cases}$ con $t\in\left[0,2\pi\right]$ tenemos $x^2+y^2=1$, la
      parametrización de un círcula en el plano.
    }
    \definicion{Curva paramétrica tangente}{
      Sea $C=\left\{M\left(x,y\right)\in\bbR^2\colon\exists t\in I,M\left(t\right)=
      \left(f\left(t\right),g\left(t\right)\right)\right\}$ con $f$,$g$ derivables
      en $I$.\\Entonces, la curva paramétrica con $t_0\in I$\\
      \setlength\extrarowheight{0pt}
      $$\begin{cases}\begin{aligned}
        x=f'\left(t_0\right)t + f\left(t_0\right)\\
        y=g'\left(t_0\right)t + g\left(t_0\right)
      \end{aligned}\end{cases}$$\\
      es la tangente a $C$ en $M\left(t_0\right)=\left(f'\left(t_0\right),
      g'\left(t_0\right)\right)\neq\left(0,0\right)$
    }
    \comentario{
      Como $f'\left(t_0\right)\neq 0$, gracias a que $x=f'\left(t_0\right)t +
      f\left(t_0\right)$, $t=\frac{x-f\left(t_0\right)}{f'\left(t_0\right)}$,
      entonces si la ponemos en $y=g'\left(t_0\right)t + g\left(t_0\right)$,
      $y=g'\left(t_0\right)\left(\frac{x-f\left(t_0\right)}{f'\left(t_0\right)}
      \right)+g\left(t_0\right)=\left[\frac{g'\left(t_0\right)}{f'\left(t_0\right)}
      x-\frac{g'\left(t_0\right)f\left(t_0\right)}{f'\left(t_0\right)}+g\left(t_0
      \right)\right]=\frac{g'\left(t_0\right)}{f'\left(t_0\right)}\left(x-f\left(
      t_0\right)\right)+g\left(t_0\right)=\frac{g'\left(t_0\right)}
      {f'\left(t_0\right)}\left(x-x_0\right)+g\left(t_0\right)$\\
      Si $y=F\left(x\right)\implies g\left(t\right)=F\left(f\left(t\right)\right)$
      \hspace{1cm}$g'\left(t_0\right)=F'\left(f\left(t_0\right)\cdot f'\left(
      t_0\right)\right)\Longleftrightarrow F'\left(x_0\right)=\frac{g'\left(t_0
      \right)}{f'\left(t_0\right)}$ y finalmente, $\left[y=F'\left(x_0\right)
      \left(x-x_0\right)+F\left(x_0\right)\right] \text{ ecuación de una tangente [...]}$\\
      \newpage
      \textbf{Interpretación vectorial:}\\
      Considerando la parametrización de la definición $\left(x_0=f\left(t_0\right),
      y_0=g\left(t_0\right)\right)$ y sean $M\left(x,y\right),M_0\left(x_0,y_0\right)
      \implies \ovl{T}=\left(f'\left(t_0\right),g'\left(t_0\right)\right)$\vspace{0.2cm}\\
      \setlength\extrarowheight{0pt}
      $\begin{cases}\begin{aligned}
        x=f'\left(t_0\right)t+x_0\\
        y=g'\left(t_0\right)t+y_0
      \end{aligned}\end{cases}\Longleftrightarrow\hspace{0.2cm}
      \begin{cases}\begin{aligned}
        x-x_0=f'\left(t_0\right)t\\
        y-y_0=g'\left(t_0\right)t
      \end{aligned}\end{cases}\Longleftrightarrow\hspace{0.2cm} M_0\ovl{M}=t\ovl{T}$, entonces $\ovl{T}$ es
      tangente a $C$ en $M_0$.
    }
  \section{Integrales de línea}
    \subsection*{Introducción}
      \vspace{-0.5cm}
      \begin{wrapfigure}{l}{.5\textwidth}
        \includegraphics[width=.4\textwidth]{integraldelinea.png}
      \end{wrapfigure}
      \hfill{} \\ \\ \\ \\
      \noindent$C=\{M\in\bbR^2\colon\exists t\in\left[a,b\right],M\left(f\left(t\right),
      g\left(t\right)\right)\}$ con $f$, $g$ derivables en $\left[a,b\right]$, $A\left(f
      \left(a\right),g\left(a\right)\right)$ y $B\left(f\left(b\right),g\left(b\right)
      \right)$.\\ \\Consideremos $\ovl{V}\left(P\left(x,y\right),Q\left(x,y\right)
      \right)$ y sea $w$ la 1-forma diferencial, i.e. $w=P\left(x,y\right)dx + Q\left(x,y
      \right)dy$\\
    \subsection*{}
    \vspace{-1cm}
    \definicion{Integral de línea}{
      $\gamma_{\ovl{AB}}=\int_{\ovl{AB}}\ovl{V}\left(M\right)\cdot\ovl{dM}=\int_{\ovl{AB}}
      \left(P\left(x,y\right)dx+Q\left(x,y\right)dy\right)=\int_{\ovl{AB}}w$ y tenemos
      $$\gamma_{\ovl{AB}} = \int_{\ovl{AB}}w = \int_{a}^{b}\left[P\left(f\left(t\right),g
      \left(t\right)\right)f'\left(t\right)+Q\left(f\left(t\right),g\left(t\right)g'\left(t
      \right)\right)\right]dt \rightarrow \text{¡Integral de Riemann!}$$ Donde $dx=f'
      \left(t\right)dt$,   $dy=g'\left(t\right)dt$
    }
    \ejemplo{Sea $w=\frac{y}{x^2+y^2}dx-\frac{x}{x^2+y^2}dy,\left(x,y\right)=\left(0,0
    \right)$, necesistamos una curva parametrizada.}{
      Sea\setlength\extrarowheight{0pt}
      $C = \begin{cases}\begin{aligned}
        x=\cos\theta\\
        y=\sin\theta
      \end{aligned}\end{cases} \text{con $\theta\in\left[0,2\pi\right]$}$, denotamos $C^+$ a la curva
      en el sentido trigonométrico.\\Nos piden calcular
      $\gamma_{C^+}=\int_{C^+}w$. \\Volvemos a la definición: $\int_{C^+}w=\int_{0}^{2\pi}
      \left[\sin\theta\left(-\sin\theta\right)-\cos\theta\left(\cos\theta\right)\right]
      d\theta=-\int_{0}^{2\pi d\theta}=-2\pi$ \par
      \vspace{0.2cm}Donde $P\left(f\left(\theta\right),g\left(\theta\right)\right) =
      \sin\theta$ y $Q\left(f\left(\theta\right),g\left(\theta\right)\right) =
      -\cos\theta$. Y también,$f'\left(\theta\right) = -\sin\theta$ y $g'\left(\theta
      \right) = \cos\theta$
    }
    \comentario{
      $\gamma_{\ovl{AB}}=\int_{\ovl{AB}}\ovl{V}\left(M\right)dM\left(t\right)=\int_{a}^{b}
      \ovl{V}\left(M\left(t\right)\right)\cdot\ovl{T}\left(t\right)\cdot dt$ \hspace{1cm}
      Donde $\ovl{V}\left(M\left(t\right)\right)$ puede ser un campo de fuerzas.
    }
    \ejemplo{Sea $w=xy\ dx+y^2\ dy+dz$ una 1-forma diferencial en $\bbR^3$ y sea $C$ la
    curva orientada en $\bbR^3$ con la parametrización $\ovl{V}\left(t\right)=\left(
    t^2,t^3,1\right)$ con $t \in \left[0,1\right]$. Calcular $\int_{C}w$}{
      $\int_{C}w=\int_{0}^{1}\left[t^5\left(2t\right)+t^6\left(3t^2\right)+0\right]dt=
      \frac{13}{21}$
    }
    \clearpage
    \teorema{}{
      Si $\ovl{V}$ es un \textbf{campo de gradiente} (i.e. $\ovl{V}=\ovl{\text{grad}}
      \varphi=\ovl{\nabla}\ovl{V}$), $\gamma_{\ovl{AB}} =\int_{\ovl{AB}}\ovl{V}\left(
      M\right)\cdot\ovl{dM}=\int_{\ovl{AB}}\ovl{\nabla}\varphi\left(M\right)\cdot
      \ovl{dM} = \int_{\ovl{AB}}d\varphi=\varphi\left(B\right)-\varphi\left(A\right)$
    }
    \comentario{
      \begin{itemize}
        \item La circulación de un campo de gradiente no depende del camino, solamente
              de los valores del campo escalar $\varphi$ definido por $\ovl{V}=\ovl{\nabla}
              \varphi$ en los extremos del camino $\ovl{AB}$.
        \item $\gamma_{\ovl{AB}}=0$ si $A$ y $B$ están en la misma curva de nivel de
              $\varphi$
      \end{itemize}
    }
    \teorema{}{
      \hspace{-0.9cm}
      \setlength{\tabcolsep}{17pt}
      \begin{tabular}{l l}
        $\int_{\ovl{AB}}\left(w_1+w_2\right)=\int_{\ovl{AB}}w_1+\int_{\ovl{AB}}w_2$&
        $\left(\ovl{V_1}\left(M\right)+\ovl{V_2}\left(M\right)\right)\ovl{dM}=
          \int_{\ovl{AB}}\ovl{V_1}\left(M\right)dM +\int_{\ovl{AB}}\ovl{V_2}\left(
          M\right)dM$\\
        $\forall\lambda\in\bbR\int_{\ovl{AB}}\lambda w_1=\lambda\int_{\ovl{AB}}w_1$&
        $\forall\lambda\in\bbR\int_{\ovl{AB}}\lambda\ovl{V_1}\left(M\right)dM=
          \lambda\int_{\ovl{AB}}\ovl{V_1}\left(M\right)dM$ \\
        Si $C\in\ovl{AB}$, entonces $\int_{\ovl{AB}}w=\int_{\ovl{AC}}w+
          \int_{\ovl{CB}}w$&
        $\int_{\ovl{AB}}\ovl{V_1}\left(M\right)dM=
          \int_{\ovl{AC}}\ovl{V_1}\left(M\right)dM+\int_{\ovl{CB}}\ovl{V_1}
          \left(M\right)dM$\\
        \multicolumn{2}{l}{$\int_{\ovl{AB}}w=-\int_{\ovl{BA}}w$ (Hay que cambiar la
          parametrización para mostrarlo)}
      \end{tabular}
    }
    \corolario{}{
      Si $\ovl{AB}$ es una curva cerrada, entonces $A=B$, y si $\ovl{V}$ es un campo
      de gradiente, entonces $\int_{\ovl{AB}}\ovl{V}\left(M\right)\ovl{dM}=0$
    }
    \definicion{Campo conservativo}{
      Se dice que un campo vectorial es conservativo si la circulación del campo
      vectorial es nula en toda curva cerrada.
    }
    \teorema{}{
      Sean $A$ y $B$ dos puntos del plano, la circulación de un campo conservativo no
      depende del camino para ir de $A$ a $B$.
    }
    \teorema{}{
      Si $\ovl{V}$ es un campo de gradiente, entonces $\ovl{V}$ es un campo conservativo.
    }
    \ejercicio{Sea $\ovl{V}\left(M\right)=\frac{3x^2+y^2}{y^2}\hat{x}-\frac{2x^3}{y^3}
    \hat{y}$\\\begin{itemize}
      \item Sean $A\left(-1,1\right)$ y $B\left(1,1\right)$ calcular $\gamma_{\ovl{AB}}$
            con $\ovl{AB}=\left[A\ B\right]$ (segmento).
      \item Idem con $\ovl{AB}\equiv$ semicírculo de centro $\Omega\left(0,1\right)$
    \end{itemize}
    }
    {
      Ver TD2
    }
    \teorema{}{
      Si $\ovl{V}$ es un campo conservativo, entonces $\ovl{V}$ es un campo de gradiente.
    }
    \comentario{
      Destacar que ahora tenemos la doble implicación entre campo conservativo y de gradiente.
    }
  \section{Longitud de una curva}
    \definicion{Longitud de una curva}{
      Sea $\gamma\colon\left[a,b\right]\longrightarrow\bbR^n$ una curva paramétrica de
      clase $\mcC_1$, la \textbf{longitud de $\gamma$} viene dada por la expresión:
      $$L\left(\gamma\right)=\int_{a}^{b}\norm{\gamma '\left(t\right)}dt$$
    }
    \comentario{
      Si $\gamma\left(t\right)\setlength\extrarowheight{0pt}\begin{pmatrix}x_1(t)\\
      \vdots\\x_n(t)\end{pmatrix}$ en coordenadas cartesianas, entonces
      $L\left(\gamma\right)=\int_{a}^{b}\sqrt{x_1'(t)^2+\hdots x_2'(t)^2}dt$
    }
    \ejercicio{Halla la longitud de arco de la siguiente curva: $\gamma\left(t\right)
    =\left(3t,3t^2,2t^3\right)$ entre los puntos $\left(0,0,0\right)$ y $\left(3,3,2
    \right)$}{
      $L\left(\gamma\right)=\int_{\left(0,0,0\right)}^{\left(3,3,2\right)}\sqrt{
      3^2+36t^2+36t^4}dt=\int_{0}^{1}\sqrt{9+36t^2+36t^4}dt=3\int_{0}^{1}\sqrt{
      4t^4+4t^2+1}dt=3\int_{0}^{1}\left(2t^2+1\right)dt=\left[2t^3+3t\right]^1_0=5$
    }
    \ejercicio{Halla la longitud de arco de la siguiente curva: $\gamma\left(t\right)
    =\setlength\extrarowheight{0pt}\begin{cases}\begin{aligned}\left(\cos(t),\sin(t),3t\right)\text{
    si $0\leq t\leq \pi$}\\\left(-1,-t+\pi,3t\right)\text{ \ \ si $\pi\leq t \leq 2\pi$}
    \end{aligned}\end{cases}$}{
      Tenemos $\gamma_1'\left(t\right)=\setlength\extrarowheight{0pt}\begin{pmatrix}
      -\sin t\\\cos t\\3\end{pmatrix}\ $ y $\ \gamma_2'\left(t\right)=
      \setlength\extrarowheight{0pt}\begin{pmatrix}0\\-1\\3\end{pmatrix}$,
      $\norm{\gamma_1\left(t\right)}=\sqrt{\sin^2t+\cos^2t+3^2}=\sqrt{10}\ $ y
      $\ \norm{\gamma_2\left(t\right)}=\sqrt{10}$, \\$L\left(\gamma_1\left(t\right)
      \right)=\int_{0}^{\pi}\sqrt[]{10}=\pi\ \sqrt[]{10}\hspace{1cm}
      L\left(\gamma_2\left(t\right)\right)=\pi\ \sqrt[]{10}\implies
      L\left(\gamma\left(t\right)\right)=L\left(\gamma_1\left(t\right)\right)+
      L\left(\gamma_2\left(t\right)\right)=\pi\ \sqrt[]{10}+\pi\ \sqrt[]{10}=
      2\pi\ \sqrt[]{10}$
    }
\chapter{Integrales dobles}
  \section{Introducción}
    \noindent Sea $\mcD=\left[a,b\right]\times\left[c,d\right]$ (con $a<b$ y $c<d$)
    un rectángulo. Como se hizo con la integral simple, vamos a subdividir
    los intervalos de los ejes (abcisas y ordenadas). Sean las particiones de
    $\left[a,b\right]$ y $\left[c,d\right]$\\ \setlength\extrarowheight{0pt}
    $\begin{cases}\begin{aligned}a=x_0<x_1<\hdots<x_n=b\\c=y_0<y_1<\hdots<y_m=d\end{aligned}\end{cases}$
    Obtenemos así las mallas, o Rectángulos Elementales:
    $R_{ij}=\left[x_{i-1},x_i\right]\times\left[y_{j-1},y_j\right]$
    \begin{wrapfigure}{l}{.4\textwidth}
      \hspace{-0.3cm}
      \includegraphics[width=.4\textwidth]{integralesdobles.png}
    \end{wrapfigure}
    \clearpage
  \section{Teoremas de Fubini}
    \teorema{Fubini en un rectángulo (versión débil)}{
      Sea $R$ un rectángula de $\bbR^2$, $R=\left[a,b\right]\times\left[c,d\right]$
      y sea $f$ una función integrable sobre $R$, tenemos: $$ \iint_{R}f\left(
      x,y\right)dxdy=\int_{a}^{b}\left(\int_{c}^{d}f\left(x,y\right)dy\right)dx=
      \int_{c}^{d}\left(\int_{a}^{b}f\left(x,y\right)dx\right)dy$$ (las integrales
      dobles son permutables)
    }
    \ejemplo{Calcular $I=\iint_A e^{x+y}dxdy$ con $A=\left[0,1\right]\times\left[0,2
    \right]$}{
      $I=\iint_A e^xe^ydxdy=\int_{0}^{1}e^xdx\cdot\int_{0}^{2}e^ydy=\left[e^x
      \right]^1_0\cdot\left[e^y\right]_0^2=\left(e-1\right)\left(e^2-1\right)$
    }
    \ejemplo{Calcular $I=\iint_R f\left(x,y\right)dxdy$ con $R=\left[0,2\right]
    \times\left[0,2\right]$ y $f$ definida en $R$ por $f\left(x,y\right)=16-x^2-2y^2$}{
        $I=\iint_R f\left(x,y\right)dxdy=\int_{0}^{2}\int_{0}^{2}16-x^2+2y^2dxdy=
        \int_{0}^{2}\left[16x-\frac{1}{3}x^3-2y^2x\right]_0^2dy=
        \int_{0}^{2}32-\frac{8}{3}-4y^2dy=\left[32y-\frac{8}{3}y-\frac{4}{3}y^3
        \right]_0^2=64-\frac{16}{3}-\frac{32}{3}=48$
    }
    \teorema{Fubini versión fuerte}{
      \begin{itemize}
        \item Sean $\varphi_1$ y $\varphi_2$ dos funiciones continuas en un rectángulo
              compacto de $\left[a,b\right]$ de $\bbR$ (uniformemente continuas) tales
              que:\\$\forall x\in\left[a,b\right],\varphi_1\left(x\right)\leq\varphi_2
              \left(x\right)$ y sea $\mcD\subset\bbR^2$ tal que $\mcD=\left\{\left(
              x,y\right)\in\bbR^2\colon a\leq x\leq b, \varphi_1\left(x\right)\leq y
              \leq\varphi_2\left(x\right)\right\}$ Si $f\colon\mcD\longrightarrow\bbR$
              es continua en $\mcD$ (entonces $f$ es integrable en $\mcD$) tenemos:
              $$\iint_Df\left(x,y\right)dxdy=\int_{a}^{b}\left(\int_{\varphi_1\left(
              x\right)}^{\varphi_2\left(x\right)}f\left(x,y\right)dy\right)dx$$
        \item Sean $\psi_1$ y $\psi_2$ dos funciones continuas en un rectángulo
              compacto de $\left[c,d\right]$ de $\bbR$ tales que:\\
              $\forall x\in\left[c,d\right],\psi_1\left(y\right)\leq\psi_2
              \left(y\right)$ y sea $\mcD'\subset\bbR^2$ tal que $\mcD'=\left\{\left(
              x,y\right)\in\bbR^2\colon c\leq y\leq d, \psi_1\left(y\right)\leq x
              \leq\psi_2\left(y\right)\right\}$ Si $f\colon\mcD'\longrightarrow\bbR$
              es continua en $\mcD'$ tenemos:
              $$\iint_{D'}f\left(x,y\right)dxdy=\int_{c}^{d}\left(\int_{\psi_1\left(
              y\right)}^{\psi_2\left(y\right)}f\left(x,y\right)dx\right)dy$$ (la
              integral se hace de forma horizontal).
      \end{itemize}
    }
    \ejemplo{Calcular $\iint_\mcD f\left(x,y\right)dxdy$ con $\mcD=\left\{\left(x,y
    \right)\in\bbR^2\colon 0\leq x \leq1, x^2\leq y\leq x\right\}$ (no es un rectángulo),
    sobre la función $f\left(x,y\right)=x+y$}{
      \begin{wrapfigure}{r}{.26\textwidth}
        \vspace{-1.65cm}
        \includegraphics[width=.26\textwidth]{ejemplo4.1.3.png}
      \end{wrapfigure}
      \vspace{0.6cm}
      $\iint_\mcD f\left(x,y\right)dxdy = \int_{0}^{1}\int_{x^2}^{x}f\left(x,y\right)
      dydx=\int_{0}^{1}\left[xy+\frac{y^2}{2}\right]^x_{x^2}dx=\vspace{0.2cm}\\
      \int_{0}^{1}\left(x^2+\frac{x^2}{2}-x^3-\frac{x^4}{2}\right)dx=\frac{3}{20}$
      \vspace{0.6cm}
    }
    \ejemplo{Calcular $\iint_\mcD e^xdxdy$ con $\mcD=\left\{\left(x,y\right)\in\bbR^2
    \colon 0\leq x\leq \log\left(y\right), 1\leq y\leq 2\right\}$}{
      \begin{wrapfigure}{r}{.26\textwidth}
        \vspace{-1.25cm}
        \includegraphics[width=.26\textwidth]{ejemplo4.1.4.png}
      \end{wrapfigure}
      \vspace{0.6cm}
      Como $f\left(x,y\right)\in\mcC_0\left(\bbR^2\right)\Rightarrow\text{Podemos
      aplicar el \textit{Teorema de Fubini}}$\vspace*{0.2cm}\\\setlength\extrarowheight{0pt}
      $\begin{cases}\begin{aligned}
        \iint_{\mcD}e^xdxdy=\int_1^2\int_0^{\log y}e^xdxdy=\int_{1}^{2}
          \left(y-1\right)dy=\left[\frac{y^2}{2}-y\right]_1^2=\frac{1}{2}\\
        \iint_{\mcD}e^xdxdy=\int_0^{\log y}\int_1^2e^xdydx=\int_{0}^{\log 2}
        e^xdydx=\int_0^{\log 2}e^x\left(2-e^x\right)dx=\frac{1}{2}
      \end{aligned}\end{cases}$
      \vspace{0.2cm}
    }
  \section{Teorema de Green-Riemann}
  \section{Superficies parametrizadas}
    \subsection{Introducción}
      \definicion{Plano tangente}{
        Si una superficie $S$ es suave (i.e. $\ovl{T_u}\times\ovl{T_v}\neq\ovl{0}$)
        definimos \textbf{el plano tangente de $S$} en $\Phi\left(u_0,v_0\right)$ como el plano
        determinado por $\ovl{T_u}$ y $\ovl{T_v}$ y con vector normal $\ovl{n}=\ovl{T_u}\times\ovl{T_v}$\\
        Una ecuación del plano tangente a $S$ en $\left(x_0,y_0,z_0\right)$ es:
        \begin{equation}\label{eq:leaf}
          \left(x-x_0,y-y_0,z-z_0\right)\cdot\ovl{n}=0
        \end{equation}
        con $\ovl{n}$ evaluado en $\left(u_0,v_0\right)$\\Si $\ovl{n}=\left(n_1,n_2,n_3\right)$
        entonces \refeq{eq:leaf} se escribe:
        \begin{equation*}
          n_1\left(x-x_0\right)+n_2\left(y-y_0\right)+n_3\left(z-z_0\right)=0
        \end{equation*}
      }
      \ejemplo{Sea $\Phi\colon\bbR^2\rightarrow\bbR^3$ dada por
        $\begin{cases}\begin{aligned}
          x=u\cos v\\
          y=u\sin v\\
          z=u^2+v^2
        \end{aligned}\end{cases}$ ¿Existe un plano tangente? \\Hallar el plano tangente en $\Phi\left(0,1\right)$
        }{
          $\ovl{T_u}=\cos\left(v\right)\ovl{i}+\sin\left(v\right)\ovl{j}+2u\ovl{k}$\\
          $\ovl{T_v}=-\sin\left(v\right)\ovl{i}+\cos\left(v\right)\ovl{j}+2v\ovl{k}$\\
          El plano tangente en $\Phi\left(u,v\right)$ es el conjunto de vectores que pasen
          por $\Phi\left(u,v\right)$ perpendiculares a $\ovl{T_u}\times\ovl{T_v}$ Tenemos:\\
          $\ovl{T_u}\times\ovl{T_v}\neq\ovl{0}=\setlength\extrarowheight{0pt}\begin{pmatrix}
          -2u^2\cos\left(v\right)+2v\sin\left(v\right)\\-2u^2\sin\left(v\right)-2v\cos\left(
          v\right)\\u\end{pmatrix} \hspace{1.5cm}\ovl{T_u}\times\ovl{T_v}=\ovl{0}\Longleftrightarrow
          \left(u,v\right) = \left(0,0\right)$\\Entonces el plano tangente en $\Phi\left(0,0\right)$
          pero sí en los otros puntos. Por ejemplo en $\Phi\left(0,1\right)=\left(1,0,1\right)$
          tenemos $\ovl{n}=\ovl{T_u}\times\ovl{T_v}=\setlength\extrarowheight{0pt}\begin{pmatrix}
          -2\\0\\1\end{pmatrix}$ así que una ecuación del plano tangente es: $-2\left(x-1\right)
          +\left(z-1\right)=0$ es decir, $z=2x-1$
        }
      \clearpage
    \subsection{Área de una superficie}
      \noindent En esta sección, consideraremos sólo superficies suaves a trozos que sean uniones de imágenes
      de superficies parametrizadas $\Phi_i\colon\mcD_i\longrightarrow\bbR^3$ tales que: (Los 'quizás'
      son a causa de medidas matemáticas que se definen en \textit{Análisis Funcional})
      \begin{enumerate}
        \item $\mcD_i$ es una región elemental del plano.
        \item $\Phi_i$ es de clase $\mcC_1$ y biyectiva, excepto, quizá, en $\partial\mcD$.
        \item La imagen de $\Phi_i$ es suave, excepto, quizá, en un número finito de puntos.
      \end{enumerate}
      \definicion{Área de una superficie}{
        Se define el \textbf{área de una superficie $\mcA\left(S\right)$} de una superficie
        parametrizada por:$$\boxed{\mcA\left(S\right)=\iint_{\mcD}\norm{\ovl{T_u}\times\ovl{T_v}}\ du\ dv}$$
      }
      \comentario{
        En el caso que $S$ fuera unión de superficies $S_i$ su área será la suma de las
        áreas de las $S_i$.
      }
      Entonces tenemos:\\
      $$\mcA\left(A\right)=\iint_\mcD \sqrt{\left(\frac{\partial\left(x,y\right)}{\partial\left(u,v\right)}\right)^2
      +\left(\frac{\partial\left(y,z\right)}{\partial\left(u,v\right)}\right)^2+
      \left(\frac{\partial\left(x,z\right)}{\partial\left(u,v\right)}\right)^2}\ du\ dv
      \hspace{1cm}\text{donde }\frac{\partial\left(x,y\right)}{\partial\left(u,v\right)}=
      \begin{vmatrix}
      \frac{\partial x}{\partial u}& \frac{\partial x}{\partial v}\vspace{2mm}\\
      \frac{\partial y}{\partial u}& \frac{\partial y}{\partial v}
      \end{vmatrix}$$
      \begin{figure}[h]
        \centering
        \vspace{-1mm}
        \includegraphics[width=.8\textwidth]{mireia1.png}
      \end{figure}\\
      Hay que considerar una porción de $\mcD$ y sea $R_{ij}$ el ij-ésimo rectángulo
      con vértices $\left(u_i,v_j\right),\left(u_{i+1},v_j\right),\left(u_i,v_{j+1}\right)
      \left(u_{i+1},v_{j+1}\right)$. Los valores de $\ovl{T_u}$ y $\ovl{T_v}$ en $\left(u_i,v_i\right)$
      serán determinados $\ovl{T_{u_i}}$ y $\ovl{T_{v_j}}$. Podemos pensar los vectores
      $\Delta_u\ovl{T_u}$ y $\Delta_v\ovl{T_v}$ como tangentes a la superficie en
      $\Phi\left(u_i,v_j\right)=\left(x_{ij},y_{ij},z_{ij}\right)$ donde $\Delta_u=
      u_{i+1}-u_i$ y $\Delta_v=v_{j+1}-v_j$\\ \\
      Entonces, estos vectores forman un paralelogramo $P_{ij}$ que está en el plano
      tangente a la superficie en $\left(x_{ij},y_{ij},z_{ij}\right)$. Para $n$ grande,
      (es una subdivisión de $n$ trozos), el área de $P_{ij}$ es una buena aproximación
      al área de $\Phi\left(R_{ij}\right)$. Como el area del paralelogramo generado por
      dos vectores $\ovl{v_1}$ y $\ovl{v_2}$ es $\norm{\ovl{v_1}\times\ovl{v_2}}$ entonces
      $\mcA\left(P_{ij}\right)=\norm{\Delta_u\ovl{T_{uj}}\times\Delta_v\ovl{T_{vj}}}=
      \Delta u\Delta v\cdot\norm{\ovl{T_{ui}}\times\ovl{T_{vj}}}$\\ \\
      Por lo tanto, el área total es $\mcA_n=\sum_{i=0}^{n-1}\sum_{j=0}^{n-1}\mcA\left(
      P_{ij}\right)$ y cuando $n\rightarrow\inf$, $\mcA_n\rightarrow\iint_{\mcD}\norm{
      \ovl{T_u}\times\ovl{T_v}}\ du\ dv$
      \ejemplo{
        Sea $\mcD$ la región determinada por $0\leq\theta\leq 2\pi$ y $0\leq r\leq 1$ y
        sea $\Phi\colon\mcD\longrightarrow\bbR^3$ tal que:\vspace{2mm}\\
        Sea la parametrización del cono $S$:
        $\begin{cases}\begin{aligned}
          x=r\cos\theta\\
          y=r\sin\theta\\
          z=r
        \end{aligned}\end{cases}$ hallar su área de superficie.
      }{
        $\frac{\partial\left(x,y\right)}{\partial\left(r,\theta\right)}=
        \begin{vmatrix}
          \cos\theta&   -r\sin\theta\\
          \sin\theta&    r\cos\theta
        \end{vmatrix}=r$\hspace{0.5cm}$\frac{\partial\left(y,z\right)}{\partial\left(r,\theta\right)}
        =-r\cos\theta$ \hspace{0.1cm} y \hspace{0.1cm} $\frac{\partial\left(x,z\right)}{\partial\left(r,\theta\right)}
        =r\sin\theta \implies$\\ \\$\implies\norm{\ovl{T_r}\times\ovl{T_\theta}}=\sqrt{r^2+r^2\cos^2\theta
        +r^2sin^2\theta}=\sqrt{2}r$\\ \\$\mcA=\int_0^2\pi\int_0^1\sqrt{2}r\ dr\ d\theta =\sqrt{2}\pi$
      }
      \comentario{
        Para confirmar que es el area de $\Phi\left(\mcD\right)$ debemos comprobar
        que $\Phi$ es biyectiva para puntos que no están en la frontera de $\Phi$\\
        Sea $\mcD_0$ el conjunto de $\left(r,\theta\right)$ tales que $0<r<1$ y
        $0<\theta <2\pi$ i.e. $\mcD_0 = \mcD\partial\mcD$\\ Para ver que $\Phi$ es
        biyectiva, suponemos que $\Phi\left(r,\theta\right)=\Phi\left(r',\theta'\right)$
        pero $\left(r,\theta\right),\left(r',\theta'\right)\in \mcD_0$\\
        Tenemos:
        \begin{alignat*}{4}
          & \begin{aligned}
            & \begin{cases}\begin{aligned}
              r\cos\theta=r'\cos\theta'\\
              r\sin\theta=r'\sin\theta'\\
              r=r'\\
            \end{aligned}\end{cases}\\
          \end{aligned}
            && \implies \quad &&
          \begin{aligned}
            \begin{cases}\begin{aligned}
              \cos\theta=\cos\theta'\\
              \sin\theta=\sin\theta'\\
            \end{aligned}\end{cases} \\
          \end{aligned}
            \quad && \implies\theta=\theta' \text{ ó } \theta=\theta'\left(2\pi\right)
        \end{alignat*}
        \\En conclusión,
        $\begin{cases}\begin{aligned}
          r=r'\\
          \theta=\theta'
        \end{aligned}\end{cases}$ y eso nos dice que $\Phi$ es biyectiva.\\
      }
      \ejemplo{
        Una helicoide se define como $\Phi\colon\mcD\rightarrow\bbR^3$ con
        $\begin{cases}\begin{aligned}
          x=r\cos\theta\\
          y=r\sin\theta\\
          z=\theta
        \end{aligned}\end{cases}$ y $\mcD$ es la región donde\vspace{-0.5cm}\\ $0\leq\theta\leq 2\pi$ y
        $0\leq r\leq 1$. Hallar su área.
      }{
        \vspace{0.2cm}
        Calculamos $\frac{\partial\left(x,y\right)}{\partial\left(1,0\right)}=r
        \hspace{0.4cm}\frac{\partial\left(y,z\right)}{\partial\left(1,0\right)}=\sin\theta
        \hspace{0.4cm}\frac{\partial\left(x,z\right)}{\partial\left(1,0\right)}=\cos\theta$\\ \\
        $\mcA = \iint_D\norm{\ovl{T_r}\times\ovl{T_\theta}}\ dr\ d\theta=
        \int_{0}^{2\pi}\int_{\theta}^{1}\sqrt{r^2+1}\ dr\ d\theta=\int_{0}^{2\pi}\ d\theta
        \int_{0}^{1}\sqrt{r^2+1}\ dr=\pi\left(\sqrt{2}+\log\left(1+\sqrt{2}\right)\right)$
      }
      \comentario{
        Si consideramos una superficie $\mcS$ dada con $z=f\left(x,y\right)$ donde
        $\left(x,y\right)\in\mcD$ admite la parametrización
        $\begin{cases}\begin{aligned}
          x=u\\ y=v\\ z=f\left(u,v\right)
        \end{aligned}\end{cases}$ con $\left(u,v\right)\in\mcD$. \\ \\
        Si $f$ es de clase $\mcC_1$, entonces la superficie es suave, y el cálculo del
        área se reduce a:$$\boxed{\mcA\left(\mcS\right)=\iint_\mcD\sqrt{1+\left(\frac{\partial f}
        {\partial x}\right)^2+\left(\frac{\partial f}{\partial y}\right)^2}\ dx\ dy}$$
      }
      \ejemplo{
        Calcular el área de la superficie de la esfera $\mcS$ definida por $x^2+y^2+z^2=1$
        (Indicación: calcular el área del hemisferio superior $\mcS^+$ (i.e. con $z\geq 0$))
      }{
        Tenemos $z=f\left(x,y\right)=\sqrt{1-x^2-y^2}$ con $\left(x^2+y^2\leq 1\right)$
        Sea $\mcD$ la región de $\left(x,y\right)\in\bbR^2/x^2+y^2\leq 1$\\ \\
        $\mcA\left(\mcS^+\right)=\iint_\mcD\sqrt{\frac{x^2}{1-x^2+y^2}+\frac{y^2}{1-x^2-y^2}+1}\ dx\ dy=
        \iint_\mcD\frac{1}{\sqrt{1-x^2-y^2}}\ dx\ dy$\\ \\Aplicamos el
        \textit{Teorema de Fubini}:\\
        $\mcA\left(\mcS^+\right)=\int_{-1}^{1}\int_{-\sqrt{1-x^2}}^{\sqrt{1-x^2}}
        \frac{1}{\sqrt{1-x^2-y^2}}\ dy\ dx=\int_{-1}^{1}\left[\arcsin\left(
        \frac{y}{\left(1-x^2\right)^{\frac{1}{2}}}\right)\right]_{-\sqrt{1-x^2}}^{
        \sqrt{1-x^2}}\ dx=\int_{-1}^{1}\left(\frac{\pi}{2}+\frac{\pi}{2}\right)\ dx=
        2\pi$ \\ \\Por simetría, $\mcA\left(\mcS^-\right)=2\pi$ y finalmente $\mcA\left(
        \mcS^+\right)+\mcA\left(\mcS^-\right)= 4\pi$
      }
  \section{Integrales de funciones escalares sobre superficies}
    \definicion{}{
      Sea $f$ una función continua con valores reales definida en $\mcS$, La integral
      de $f$ sobre $\mcS$ se define: $\iint_{\mcS}f\left(x,y,z\right)\ d\mcS=\iint_\mcS f\ d\mcS$
      (donde $d\mcS$ es un diferencial de superficie)\\ \\
      $\iint_\mcS f\ d\mcS=\iint_{\mcD}f\left(\Phi\left(u,v\right)\right)\norm{\ovl{T_u}
      \times\ovl{T_v}}\ du\ dv$
    }
    \comentario{
      \begin{enumerate}
        \item Si $f\equiv 1$ volvemos a encontrar la fórmula del área
        \item La integral de superficie, como el área de superficie, no depende de la
              parametrización elegida.
        \item Si $\mcS$ es una unión de superficies parametrizadas, si $i=1,\hdots ,N$
              que no se intersecan excepto quizá, a lo largo de curvas que definen sus
              fronteras, entonces: $\iint_\mcS f\ d\mcS=\sum_{i=1}^{N}\iint_{\mcS_i}f\
              d\mcS_i$\\
      \end{enumerate}
    }
    \ejemplo{Consideramos el helicoide $\mcS$ anterior. Sea $f\left(x,y,z\right)=\sqrt{x^2+y^2
    +1}$. Hallar $\iint_\mcS f\ d\mcS$}{
      $\frac{\partial\left(x,y\right)}{\partial\left(r,\theta\right)}=2 \hspace{0.6cm}
      \frac{\partial\left(y,z\right)}{\partial\left(r,\theta\right)}=\sin\theta \hspace{0.6cm}
      \frac{\partial\left(x,z\right)}{\partial\left(r,\theta\right)}=\cos\theta \hspace{1cm}$
      con $r\in[0,1)$, $\theta\in[0,2\pi)$\\ \\
      Tenemos $f\left(r\cos\theta,r\sin\theta,0\right)=\sqrt{r^2+1}$ lo que nos da \\
      $\iint_\mcD f\ d\mcS=\iint_\mcD f\left(\Phi\left(r,\theta\right)\right)\norm{\ovl{T_r}
      \times\ovl{T_\theta}}\ dr\ d\theta= \int_{0}^{2\pi}\int_{0}^{1}\sqrt{r^2+1}\sqrt{r^2+1}\ dr\ d
      \theta=\int_{0}^{2\pi}\frac{4}{3}\ d\theta=\frac{8}{3}\pi$
    }
    \comentario{
      \vspace{-0.25cm}Si $z=g\left(x,y\right)$ con $g\in\mcC_1$ tenemos:\hspace{0.2cm}
      $\boxed{\iint_\mcS f\ d\mcS=\iint_\mcD f\left(x,y,g\left(x,y\right)\right)\sqrt{1+\left(
      \frac{\partial g}{\partial x}\right)^2+\left(\frac{\partial g}{\partial y}\right)^2}
      \ dx\ dy}$
    }
    \ejemplo{Sea $\mcS$ una superficie definida por $z=x^2+y$ donde $\mcD$ es la región
      caracterizada por $0\leq x\leq 1$ y $-1\leq y\leq 1$. Calcular $\iint_\mcD x\ d\mcS$}{
      Tenemos $\iint_\mcS xd\mcS=\iint_\mcD x\sqrt{1+4x^2+1}\ dxdy=\int_{-1}^{1}\left(
      \int_{0}^{1}x\sqrt{4x^2+2}dx\right)dy=\frac{1}{8}\int_{-1}^{1}\left(\int_{0}^{1}
      8x\left(4x^2+2\right)^\frac{1}{2}dx\right)dy=\frac{2}{3}\frac{1}{8}\int_{-1}^{1}
      \left[\left(2+4x^2\right)^\frac{3}{2}\right]_0^1dy=\frac{1}{12}\int_{-1}^{1}\left(
      6^\frac{3}{2}-2^\frac{3}{2}\right)dy=\sqrt{2}\left(\sqrt{3}-\frac{1}{3}\right)$
    }
    \clearpage
    \comentario{
      Si $z=g\left(x,y\right)$\\
      \begin{wrapfigure}{r}{.3\textwidth}
        \vspace{-0.7cm}
        \centering
        \includegraphics[width=.3\textwidth]{mireia2.png}
      \end{wrapfigure}
      $\Phi\left(x,y,z\right)=z-g\left(x,y\right)=0$\\ 
      Un vector normal a $\Phi\text{ es }\ovl{n}=\ovl{\nabla}\Phi$
      $\text{es decir, }\boxed{\ovl{n}=-\frac{\partial g}{\partial x}\ovl{i}-\frac{\partial g}{\partial y}\ovl{j}+\ovl{k}}$
      \vspace{0.2cm}\\$\ovl{n},\ovl{k}=\norm{\ovl{n}}\cdot\norm{\ovl{k}}\cdot
      \cos(\ovl{n},\ovl{k}) \Leftrightarrow\cos\theta=
      \frac{\ovl{n}\cdot\ovl{k}}{\norm{\ovl{n}}\cdot\norm{\ovl{k}}}=
      \frac{\ovl{n}\cdot\ovl{k}}{\norm{\ovl{n}}}=\frac{1}{\norm{\ovl{n}}} 
      \Leftrightarrow\norm{\ovl{n}}=\frac{1}{\cos\theta}\\
      \text{(sabiendo que }\ovl{k}\text{ es unitario y es }\left(0,0,1\right)\text{)}\vspace{0.2cm}\\
      \text{Al final obtenemos, } \boxed{\iint_\mcS f\ d\mcS=\iint_\mcD\frac{f\left(x,y,
      g\left(x,y\right)\right)}{\cos\theta}}\ dxdy$
    }
    
    \ejemplo{Calcular $\iint_\mcS x\ d\mcS$ donde $\mcS$ es el triángulo con vértices
    $\left(1,0,0\right),\left(0,1,0\right),\left(0,0,1\right)$ (indicación: encontrar un
    vector normal $\ovl{n}$ unitario, tendriámos $\ovl{n}\cdot\ovl{k}=\cos\theta$)}{
      Esta superficie es el plano dado por $x+y+z=1$. Un vector normal $\ovl{n}$ a
      este plano tiene coordenadas $\left(1,1,1\right)$. Tenemos $\norm{\ovl{n}}=\sqrt{3}$
      lo que nos da un vector normal unitario a este plano $\ovl{n}=\frac{\ovl{n}}{\norm{\ovl{n}}}
      \left(\frac{1}{\sqrt{3}},\frac{1}{\sqrt{3}},\frac{1}{\sqrt{3}}\right) \text{esto va vertical}$.
      Con la fórmula del comentario anterior, tenemos $\iint_\mcS x\ d\mcS=\iint_\mcD \frac
      {x}{\ovl{n}\cdot\ovl{k}}\ dxdy=\iint_\mcD \frac{x}{\frac{1}{\sqrt{3}}}$ donde 
      $\ovl{n}\cdot\ovl{k}=\norm{\ovl{n}}\norm{\ovl{k}}\cos\theta\Leftrightarrow
      \ovl{n}\cdot\ovl{k}=\cos\theta$ porque $\norm{\ovl{n}}\norm{\ovl{k}}=1$ por unitarios.\\
      $\iint_\mcD \frac{x}{\frac{1}{\sqrt{3}}}=\sqrt{3}\int_{0}^{1}\int_{0}^{1-x}x\ dydx=
      \frac{\sqrt{3}}{6}$      
    }
  \section{Integrales de superficie de funciones vectoriales}
    \definicion{Superficie orientada}{
      Una superficie suave $\mcS$ es una \textbf{superficie orientada} si existe una función
      normal unitaria $\ovl{n}$ definida en cada punto $\left(x,y,z\right)$ sobre la superficie.
      El campo vectorial $\ovl{n}\left(x,y,z\right)$ recibe el nombre de \textbf{orientación} 
      de $\mcS$.\\ \\ Puesto que una normal unitaria a $\mcS$ puede ser $\ovl{n}\left(x,y,z\right)
      $ o $-\ovl{n}\left(x,y,z\right)$, una superficie \textbf{orientada} tiene dos orientaciones.
    }
    \comentario{
      Tenemos $\ovl{n}=\frac{\ovl{T_u}\times\ovl{T_v}}{\norm{\ovl{T_u}\times\ovl{T_v}}}$
      Una superficie $\mcS$ definida por $z=g\left(x,y\right)$ tiene una orientación 
      hacia arriba cuando las normales unitarias están dirigidas hacia arriba.
    }
    \definicion{Superficie cerrada}{
      Una superficie \textbf{cerrada} se define como la frontera de un sólido finito.
    }
    \ejemplo{}{La superficie de una esfera es una superficie cerrada}
    \comentario{
      \begin{enumerate}
        \item Si una superficie \textbf{suave} $\mcS$ está definida por $f\left(x,y,z\right)=0$
              entonces la normal es \textbf{unitaria} $\ovl{n}=\frac{\ovl{\nabla}f}
              {\norm{\ovl{\nabla}f}}$
        \item Si $f$ está definida de forma explícita. $z=g\left(x,y\right)$ podemos escribir
      \end{enumerate}
    }
\chapter{Integrales Triples}
    \section{Introducción}
    \noindent Como lo hicimos con la integral doble, vamos a definir la integral de una función en el
    paralelepípedo (rectángulo). $V=\left[a_1,b_1\right]\times\left[a_2,b_2\right]\times\left[a_3,b_3\right]$.\\
    \noindent Sean $N_1,N_2,N_3$ tres enteros dados:
    $\begin{cases}\begin{aligned}
      \vspace{2pt}
      x_i-x_{i-1}=\frac{b_1-a_1}{N_1}\\
      \vspace{2pt}
      y_j-y_{j-1}=\frac{b_2-a_2}{N_2}\\
      \vspace{2pt}
      z_k-z_{k-1}=\frac{b_3-a_3}{N_3}
    \end{aligned}\end{cases}$\\
    lo que nos da \textbf{paralelepípedos elementales} $w_{ijk}=\left[x_{i-1},x_i\right]\times
    \left[y_{j-1},y_j\right]\times\left[z_{k-1},z_k\right]$
    \definicion{Integral Triple}{
      Sea $\mcI_\mcV\left(f\right)=\sum_{i=1}^{N_1}\sum_{j=1}^{N_2}\sum_{k=1}^{N_3}
      f\left(x_i,y_j,z_k\right)\left(x_i-x_{i-1}\right)\left(y_j-y_{j-1}\right)\left(z_k-z_{k-1}\right)$
      si $N_1,N_2$ y $N_3$ se van al infinito, entonces $\mcI_\mcV\left(f\right)$ admite un 
      límite en $\bbR$. Este límite se llama \textbf{Integral Triple} de $f$ en $\mcV$y se escribe:
      $$\mcI=\iiint_\mcV f\left(x,y,z,\right)\ dx\ dy\ dz$$
    }
    \comentario{
      Como lo hicimos con la integral doble, la integral triple se puede calcular gracias a
      tres integrales simples.
    }
    \teorema{Cálculo directo}{
      \begin{enumerate}
        \item Si $\mcV$ es de la forma siguiente: $$\mcV = \left\{\left(x,y,z\right)
        \in\bbR^3\colon a\leq x\leq b\colon g­_1(x)\leq y\leq g_2(x)\colon h_1(x,y)\leq z
        \leq h_2(x,y)\right\}$$
        $$\text{y entonces, }\iiint_mcV f\left(x,y,z\right)\ dx\ dy\ dz=
        \int_{a}^{b}\left(\int_{g_1(x)}^{g_2(x)}\left(\int_{h_{1}(x,y)}^{h_2(x,y)}f\left(x,y,z\right)\ dz\right)\ dy\right)\ dx$$
        
        \item Si $\mcV$ es de la forma siguiente: $$\mcV = \left\{\left(x,y,z\right)
        \in\bbR^3\colon a\leq x\leq b\colon g­_1(x)\leq z\leq g_2(x)\colon h_1(x,z)\leq y
        \leq h_2(x,z)\right\}$$
        $$\text{entonces, }\iiint_mcV f\left(x,y,z\right)\ dx\ dy\ dz=
        \int_{a}^{b}\left(\int_{g_1(x)}^{g_2(x)}\left(\int_{h_{1}(x,z)}^{h_2(x,z)}f\left(x,y,z\right)\ dy\right)\ dz\right)\ dx$$     
      
        \item \textit{Idem.}
      \end{enumerate}
    }
  \section{Cambio de variables}
    \noindent Consideramos $\mcI=\iiint_\mcV f\left(x,y,z\right)\ dx\ dy\ dz$ y supongamos que tenemos
    el cambio de variable siguiente:\\
    \begin{tabular}{l}
      \hspace{-1cm}
      $\begin{cases}\begin{aligned}
        &x=a\left(u,v,w\right)\\
        &y=b\left(u,v,w\right)\\
        &z=c\left(u,v,z\right)
      \end{aligned}\end{cases}$\\
    \end{tabular}
    con $\left(u,v,w\right)\in\Omega\text{ biyección de }\mcV$\\
    Podemos, entonces, definir una nueva función $g$ tal que: $g\left(u,v,w\right)=
    f\left(a\left(u,v,w\right),b\left(u,v,w\right),c\left(u,v,w\right)\right)$ lo que nos
    permite escribir:
    \teorema{}{
      $$\iiint_\mcV f\left(x,y,z\right)\ dx\ dy\ dz=\iiint_\Omega g\left(u,v,w\right)
      \begin{vmatrix}J_g\left(u,v,w\right)\end{vmatrix}\ du\ dv\ dw$$
    }
    \ejemplo{Calcular $\mcI=\iiint_\mcV xyz\left(1-x-y-z\right)\ dx\ dy\ dz$ con $\mcV$
    delimitado por los planos\\ $\begin{cases}\begin{aligned}&x=0\\ &y=0\\ &z=0\\ &x+y+z=1\end{aligned}\end{cases}$ con el cambio de variable propuesto:
    $\begin{cases}\begin{aligned}&x=u\left(1-u\right)\\ &y=uv\left(1-w\right)\\ &z=uvw\end{aligned}\end{cases}$}{
      \begin{wrapfigure}[1]{r}{.3\textwidth}
        \vspace{-1cm}
        \begin{center}
            \includegraphics[width=.3\textwidth]{mireia4.png}
        \end{center}
      \end{wrapfigure}
      Tenemos $g\left(u,v,w\right)=\left(u-uv,uv-uvw,uvw\right)$\vspace{0.4cm}\\ 
      y además:
      $\begin{cases}\begin{aligned} &u=x+y+z\\ &v=\frac{y+z}{x+y+z}\\ &w=\frac{z}{y+z}\end{aligned}\end{cases}$\vspace{0.4cm}\\
      $\begin{vmatrix}J_g\left(u,v,w\right)\end{vmatrix}=
      \begin{vmatrix}
        1-v&                -u&                 0\\
        v\left(1-w\right)&  u\left(1-w\right)&  -uv\\
        vw&                 uw&                 uv
      \end{vmatrix}=u^2v$\\ 

      \vspace{0.2cm}$g\left(u,v,w\right)=f\left(u\left(1-v\right),uv\left(1-w\right),uvw\right)=u^3v^2w\left(1-u\right)\left(1-v\right)\left(1-w\right)$\\

      \vspace{0.2cm}$\mcI=\iiint u^5v^3w\left(1-u\right)\left(1-v\right)\left(1-w\right)dudvdw=\int_{0}^{1}u^5\left(1-v\right)dv\int_{0}^{1}w\left(1-w\right)dw=$
      $\left(\frac16-\frac17\right)\left(\frac14-\frac15\right)\left(\frac12-\frac13\right)$
    }
    \subsection{Algunos cambios de variable}
      \subsubsection{Cambio en coordenadas cilíndricas}
        \vspace{0.2cm}
        \begin{wrapfigure}[5]{l}{.27\textwidth}
            \vspace{-1.4cm}
            \begin{center}
                \includegraphics[width=.35\textwidth]{mireia3.png}
            \end{center}
        \end{wrapfigure}
        $\begin{cases}\begin{aligned} &x=r\cos\theta\\ &y=r\sin\theta\\ &z=z\end{aligned}\end{cases}$ con $r\geq 0,\ \theta\in\left[0,2\pi\right]$
        $\hspace{0.5cm}g\left(r,\theta,z\right)=\left(r\cos\theta,r\sin\theta,z\right)$\\
        
        \vspace{0.2cm}$\begin{vmatrix}J_g\left(r.\theta,z\right)\end{vmatrix}=
        \begin{vmatrix}\frac{\partial\left(x,y,z\right)}{\partial\left(r,\theta,z\right)}\end{vmatrix}=
        \begin{vmatrix}
        \cos\theta& -r\sin\theta& 0\\
        \sin\theta& r\sin\theta & 0\\
        0         & 0           & 1
        \end{vmatrix}=r$
        $\iiint_V \left(x,y,z\right)\ dx\ dy\ dz=$\vspace{0.4cm}\\
        \hspace*{1cm}$=\iiint_\Omega g\left(r\cos\theta,r\sin\theta,z\right)r\ dr\ d\theta\ dz$

        \clearpage
        \comentario{
          Si $\mcV=\left\{\left(x,y,z\right)\in\bbR^3/(x,y)\in\mcD,\varphi_1(x,y)\leq z \leq\varphi_2(x,Y)\right\}$
          la proyección de $\mcV$ sobre ($Oxy$) es el dominino $\mcD$ tal que:
          $\mcD=\left\{(r,\theta)\in\bbR^2,\alpha<\theta<\beta, h_1(\theta)\leq r \leq h_2(\theta)\right\}$
          Entonces si $f$ es una función continua en $\mcV$ tenemos:
          $$\iiint_\mcV f(x,y,z)dxdydz=\iint_\mcD\left(\int_{\varphi_1(x,y)}^{\varphi_2(x,y)}f(x,y,z)dz\right)dxdy=
          \int_{\alpha}^{\beta}\int_{h_1(\theta)}^{h_1(\theta)}\int_{\varphi_1(x,y)}^{\varphi_2(x,y)}f\left(r\cos\theta,r\sin\theta,z\right)r\ dzdrd\theta$$
        }
        \ejemplo{Calcular $\iiint_\mcV \left(x^2+y^2+1\right)\ dxdydz$\\ donde
        $\mcV=\left\{(x,y,z)\in\bbR^3,x^2+y^2\leq 1\text{ y }0\leq z \leq 2\right\}$}{
          Vamos a utilizar las coordenadas cilíndricas $\begin{cases}\begin{aligned} &x=r\cos\theta\\ &y=r\sin\theta\\ &z=z\end{aligned}\end{cases}$
          e integramos en $\Delta = [0,1]\times[0,2\pi]\times[0,2]$.\\

          \vspace{0.4cm}Tenemos $f(r\cos\theta,r\sin\theta,z)=r^2+1$ lo que nos da $\iiint_\mcV f\left(x,y,z\right)\ dxdydz=
          \iiint_\Delta (1+r^2)r\ drd\theta dz=\int_{0}^{\pi}dz\int_{0}^{2\pi}d\theta\int_{0}^{1}(1+r^2)r\ dr=4\pi\left[\frac{r^2}{2}+\frac{r^4}{4}\right]_0^1
          =4\pi\left(\frac12+\frac14\right)=3\pi$
        }
      \subsection{Coordenadas Esféricas}
        $\begin{cases}\begin{aligned}
          &x = \rho\sin\varphi\cos\theta\\
          &y = \rho\sin\varphi\sin\theta \\ 
          &z = \rho\cos\varphi\\
          &\rho^2 = x^2+y^2+z^2
        \end{aligned}\end{cases}$ con $\rho\geq 0$, $\theta\in[0,2\pi]$, $\varphi\in[0,\pi]$\\

        \vspace{0.4cm}El Jacobiano será $\boxed{\rho^2\sin\varphi}$
        \comentario{
          Si ponemos 
          $\begin{cases}\begin{aligned}
            &r=\rho\sin\varphi\\
            &\theta=\theta\\
            &z=\rho\cos\varphi
          \end{aligned}\end{cases}$ entonces podemos transformar las coordenadas esféricas en cilíndricas $\left(r,\theta,z\right)$
        }
        \ejemplo{Sea $\mcB$ la bola unidad y $a>1$, calcular $\mcI = \iiint_\mcB\frac{dx\ dy\ dz}{\sqrt{x^2+y^2+(z-1)^2}}$}{
          Pasamos a coordenadas esféricas: $\begin{cases}\begin{aligned}
            &x = \rho\sin\varphi\cos\theta\\
            &y = \rho\sin\varphi\sin\theta \\ 
            &z = \rho\cos\varphi
          \end{aligned}\end{cases}$ y obtenemos\vspace{0.2cm} \\
          $\mcI=\int_{0}^{1}\int_{-\pi}^{\pi}\int_{0}^{2\pi}\frac{\rho^2\sin\varphi}{\sqrt{\rho^2+r^2-2a\rho\cos\varphi}}d\varphi d\theta d\rho$
          \hspace{0.5cm}Hacemos el cambio de variable: 
          $\begin{cases}\begin{aligned} &t=\rho^2+a^2-2a\rho\cos\varphi\\ &dt=2a\rho\sin\varphi\ d\varphi\end{aligned}\end{cases}$
          
          \vspace{0.4cm}y \hspace{0.5cm}$\mcI = 2\pi\int_{0}^{1}\rho^2\left(\int_{(a-\rho)^2}^{(a+\rho)^2}\frac{dt}{2a\rho\sqrt{t}}\right)d\rho = \frac{4\pi}{a}\int_{0}^{1}\rho^2\ d\rho=\frac{4\pi}{3a}$
        }
\end{document}

